\documentclass{article}%
\usepackage[T1]{fontenc}%
\usepackage[utf8]{inputenc}%
\usepackage{lmodern}%
\usepackage{textcomp}%
\usepackage{lastpage}%
%
\title{Grade 1 Math Workbook}%
\author{Akshai Srinivasan, Teja Koripella, Skye Tyrrell, Angellou Sutharsan}%
\date{}%
%
\begin{document}%
\normalsize%
\maketitle%
\vfill%
\begin{center}%
ISBN: 9798848749724%
\linebreak%
\copyright%
MathMaestro.org%
\end{center}%
\newpage%
\section{Table of Contents}%
\label{sec:TableofContents}%
Problems...........................................................................................Page 3%
\newline%
Comparing Two Digit Numbers%
.%
.%
.%
.%
.%
.%
.%
.%
.%
.%
.%
.%
.%
.%
.%
.%
.%
.%
.%
.%
.%
.%
.%
.%
.%
.%
.%
.%
.%
.%
.%
.%
.%
.%
.%
.%
.%
.%
.%
.%
.%
.%
.%
.%
.%
.%
.%
.%
.%
.%
.%
.%
.%
.%
.%
.%
Page 4{-}15%
\newline%
Adding Numbers Within 20%
.%
.%
.%
.%
.%
.%
.%
.%
.%
.%
.%
.%
.%
.%
.%
.%
.%
.%
.%
.%
.%
.%
.%
.%
.%
.%
.%
.%
.%
.%
.%
.%
.%
.%
.%
.%
.%
.%
.%
.%
.%
.%
.%
.%
.%
.%
.%
.%
.%
.%
.%
.%
.%
.%
.%
.%
.%
.%
.%
.%
.%
.%
Page 16{-}27%
\newline%
Subtracting Numbers Within 20%
.%
.%
.%
.%
.%
.%
.%
.%
.%
.%
.%
.%
.%
.%
.%
.%
.%
.%
.%
.%
.%
.%
.%
.%
.%
.%
.%
.%
.%
.%
.%
.%
.%
.%
.%
.%
.%
.%
.%
.%
.%
.%
.%
.%
.%
.%
.%
.%
.%
.%
.%
.%
.%
Page 28{-}39%
\newline%
Finding the Missing Number%
.%
.%
.%
.%
.%
.%
.%
.%
.%
.%
.%
.%
.%
.%
.%
.%
.%
.%
.%
.%
.%
.%
.%
.%
.%
.%
.%
.%
.%
.%
.%
.%
.%
.%
.%
.%
.%
.%
.%
.%
.%
.%
.%
.%
.%
.%
.%
.%
.%
.%
.%
.%
.%
.%
.%
.%
.%
.%
Page 40{-}51%
\newline%
Adding Three Numbers%
.%
.%
.%
.%
.%
.%
.%
.%
.%
.%
.%
.%
.%
.%
.%
.%
.%
.%
.%
.%
.%
.%
.%
.%
.%
.%
.%
.%
.%
.%
.%
.%
.%
.%
.%
.%
.%
.%
.%
.%
.%
.%
.%
.%
.%
.%
.%
.%
.%
.%
.%
.%
.%
.%
.%
.%
.%
.%
.%
.%
.%
.%
.%
.%
.%
.%
.%
.%
.%
Page 52{-}63%
\newline%
Before and After%
.%
.%
.%
.%
.%
.%
.%
.%
.%
.%
.%
.%
.%
.%
.%
.%
.%
.%
.%
.%
.%
.%
.%
.%
.%
.%
.%
.%
.%
.%
.%
.%
.%
.%
.%
.%
.%
.%
.%
.%
.%
.%
.%
.%
.%
.%
.%
.%
.%
.%
.%
.%
.%
.%
.%
.%
.%
.%
.%
.%
.%
.%
.%
.%
.%
.%
.%
.%
.%
.%
.%
.%
.%
.%
.%
.%
Page 64{-}75%
\newline%
Basic Place Value%
.%
.%
.%
.%
.%
.%
.%
.%
.%
.%
.%
.%
.%
.%
.%
.%
.%
.%
.%
.%
.%
.%
.%
.%
.%
.%
.%
.%
.%
.%
.%
.%
.%
.%
.%
.%
.%
.%
.%
.%
.%
.%
.%
.%
.%
.%
.%
.%
.%
.%
.%
.%
.%
.%
.%
.%
.%
.%
.%
.%
.%
.%
.%
.%
.%
.%
.%
.%
.%
.%
.%
.%
.%
.%
Page 76{-}87%
\newline%
Breaking Apart Numbers%
.%
.%
.%
.%
.%
.%
.%
.%
.%
.%
.%
.%
.%
.%
.%
.%
.%
.%
.%
.%
.%
.%
.%
.%
.%
.%
.%
.%
.%
.%
.%
.%
.%
.%
.%
.%
.%
.%
.%
.%
.%
.%
.%
.%
.%
.%
.%
.%
.%
.%
.%
.%
.%
.%
.%
.%
.%
.%
.%
.%
.%
.%
.%
.%
.%
Page 88{-}101%
\newline%
Solutions......................................................................................Page 102%
\newline%
Comparing Two Digit Numbers Solutions%
.%
.%
.%
.%
.%
.%
.%
.%
.%
.%
.%
.%
.%
.%
.%
.%
.%
.%
.%
.%
.%
.%
.%
.%
.%
.%
.%
.%
.%
.%
.%
.%
.%
.%
.%
.%
.%
.%
Page 103{-}106%
\newline%
Adding Numbers Within 20 Solutions%
.%
.%
.%
.%
.%
.%
.%
.%
.%
.%
.%
.%
.%
.%
.%
.%
.%
.%
.%
.%
.%
.%
.%
.%
.%
.%
.%
.%
.%
.%
.%
.%
.%
.%
.%
.%
.%
.%
.%
.%
.%
.%
.%
Page 107{-}110%
\newline%
Subtracting Numbers Within 20 Solutions%
.%
.%
.%
.%
.%
.%
.%
.%
.%
.%
.%
.%
.%
.%
.%
.%
.%
.%
.%
.%
.%
.%
.%
.%
.%
.%
.%
.%
.%
.%
.%
.%
.%
.%
Page 111{-}114%
\newline%
Finding the Missing Number Solutions%
.%
.%
.%
.%
.%
.%
.%
.%
.%
.%
.%
.%
.%
.%
.%
.%
.%
.%
.%
.%
.%
.%
.%
.%
.%
.%
.%
.%
.%
.%
.%
.%
.%
.%
.%
.%
.%
.%
.%
.%
Page 115{-}118%
\newline%
Adding Three Numbers Solutions%
.%
.%
.%
.%
.%
.%
.%
.%
.%
.%
.%
.%
.%
.%
.%
.%
.%
.%
.%
.%
.%
.%
.%
.%
.%
.%
.%
.%
.%
.%
.%
.%
.%
.%
.%
.%
.%
.%
.%
.%
.%
.%
.%
.%
.%
.%
.%
.%
.%
.%
.%
Page 119{-}122%
\newline%
Before and After Solutions%
.%
.%
.%
.%
.%
.%
.%
.%
.%
.%
.%
.%
.%
.%
.%
.%
.%
.%
.%
.%
.%
.%
.%
.%
.%
.%
.%
.%
.%
.%
.%
.%
.%
.%
.%
.%
.%
.%
.%
.%
.%
.%
.%
.%
.%
.%
.%
.%
.%
.%
.%
.%
.%
.%
.%
.%
.%
.%
Page 123{-}126%
\newline%
Basic Place Value Solutions%
.%
.%
.%
.%
.%
.%
.%
.%
.%
.%
.%
.%
.%
.%
.%
.%
.%
.%
.%
.%
.%
.%
.%
.%
.%
.%
.%
.%
.%
.%
.%
.%
.%
.%
.%
.%
.%
.%
.%
.%
.%
.%
.%
.%
.%
.%
.%
.%
.%
.%
.%
.%
.%
.%
.%
.%
Page 127{-}130%
\newline%
Breaking Apart Numbers Solutions%
.%
.%
.%
.%
.%
.%
.%
.%
.%
.%
.%
.%
.%
.%
.%
.%
.%
.%
.%
.%
.%
.%
.%
.%
.%
.%
.%
.%
.%
.%
.%
.%
.%
.%
.%
.%
.%
.%
.%
.%
.%
.%
.%
.%
.%
.%
.%
Page 131{-}134%
\newline%
\newpage

%
\huge%
\vspace*{\fill}%
\begin{center}%
Problems%
\end{center}%
\vspace*{\fill}%
\pagebreak%
\normalsize%
\large%
\begin{center}%
\textbf{Comparing Two Digit Numbers- Worksheet 1}%
\newline%
\end{center} \normalsize%
1) Fill in the blank with <, >, or =: 76 \_\_\_\_\_\_\_ 25%
\newline%
\newline%
\newline%
2) Fill in the blank with <, >, or =: 28 \_\_\_\_\_\_\_ 71%
\newline%
\newline%
\newline%
3) Fill in the blank with <, >, or =: 71 \_\_\_\_\_\_\_ 92%
\newline%
\newline%
\newline%
4) Fill in the blank with <, >, or =: 70 \_\_\_\_\_\_\_ 64%
\newline%
\newline%
\newline%
5) Fill in the blank with <, >, or =: 69 \_\_\_\_\_\_\_ 15%
\newline%
\newline%
\newline%
6) Fill in the blank with <, >, or =: 1 \_\_\_\_\_\_\_ 6%
\newline%
\newline%
\newline%
7) Fill in the blank with <, >, or =: 58 \_\_\_\_\_\_\_ 12%
\newline%
\newline%
\newline%
8) Fill in the blank with <, >, or =: 70 \_\_\_\_\_\_\_ 20%
\newline%
\newline%
\newline%
9) Fill in the blank with <, >, or =: 32 \_\_\_\_\_\_\_ 22%
\newline%
\newline%
\newline%
10) Fill in the blank with <, >, or =: 39 \_\_\_\_\_\_\_ 16%
\newline%
\newline%
\newline%
11) Fill in the blank with <, >, or =: 59 \_\_\_\_\_\_\_ 82%
\newline%
\newline%
\newline%
12) Fill in the blank with <, >, or =: 62 \_\_\_\_\_\_\_ 60%
\newline%
\newline%
\newline%
13) Fill in the blank with <, >, or =: 11 \_\_\_\_\_\_\_ 39%
\newline%
\newline%
\newline%
14) Fill in the blank with <, >, or =: 16 \_\_\_\_\_\_\_ 39%
\newline%
\newline%
\newline%
15) Fill in the blank with <, >, or =: 72 \_\_\_\_\_\_\_ 88%
\newline%
\newline%
\newline%
16) Fill in the blank with <, >, or =: 75 \_\_\_\_\_\_\_ 31%
\newline%
\newline%
\newline%
17) Fill in the blank with <, >, or =: 60 \_\_\_\_\_\_\_ 82%
\newline%
\newline%
\newline%
18) Fill in the blank with <, >, or =: 89 \_\_\_\_\_\_\_ 11%
\newline%
\newline%
\newline%
19) Fill in the blank with <, >, or =: 50 \_\_\_\_\_\_\_ 24%
\newline%
\newline%
\newline%
20) Fill in the blank with <, >, or =: 84 \_\_\_\_\_\_\_ 61%
\newline%
\newline%
\newline%
21) Fill in the blank with <, >, or =: 19 \_\_\_\_\_\_\_ 31%
\newline%
\newline%
\newline%
22) Fill in the blank with <, >, or =: 23 \_\_\_\_\_\_\_ 87%
\newline%
\newline%
\newline%
23) Fill in the blank with <, >, or =: 68 \_\_\_\_\_\_\_ 51%
\newline%
\newline%
\newline%
24) Fill in the blank with <, >, or =: 31 \_\_\_\_\_\_\_ 56%
\newline%
\newline%
\newline%
25) Fill in the blank with <, >, or =: 15 \_\_\_\_\_\_\_ 77%
\newline%
\newline%
\newline%
26) Michael has 43 pencils. Rachel has 38 pencils. Who has more pencils?%
\newline%
\newline%
\newline%
27) Rachel has 96 toys. Sam has 57 toys. Who has more toys?%
\newline%
\newline%
\newline%
28) Laura has 8 pens. Michael has 15 pens. Who has more pens?%
\newline%
\newline%
\newline%
29) Sam has 9 pencils. Rachel has 6 pencils. Who has more pencils?%
\newline%
\newline%
\newline%
30) Laura has 31 dollars. Michael has 17 dollars. Who has more dollars?%
\newline%
\newline%
\newline%
\pagebreak%
\large%
\begin{center}%
\textbf{Comparing Two Digit Numbers- Worksheet 2}%
\newline%
\end{center} \normalsize%
1) Fill in the blank with <, >, or =: 17 \_\_\_\_\_\_\_ 97%
\newline%
\newline%
\newline%
2) Fill in the blank with <, >, or =: 7 \_\_\_\_\_\_\_ 44%
\newline%
\newline%
\newline%
3) Fill in the blank with <, >, or =: 92 \_\_\_\_\_\_\_ 84%
\newline%
\newline%
\newline%
4) Fill in the blank with <, >, or =: 99 \_\_\_\_\_\_\_ 3%
\newline%
\newline%
\newline%
5) Fill in the blank with <, >, or =: 88 \_\_\_\_\_\_\_ 44%
\newline%
\newline%
\newline%
6) Fill in the blank with <, >, or =: 94 \_\_\_\_\_\_\_ 69%
\newline%
\newline%
\newline%
7) Fill in the blank with <, >, or =: 7 \_\_\_\_\_\_\_ 87%
\newline%
\newline%
\newline%
8) Fill in the blank with <, >, or =: 53 \_\_\_\_\_\_\_ 85%
\newline%
\newline%
\newline%
9) Fill in the blank with <, >, or =: 26 \_\_\_\_\_\_\_ 45%
\newline%
\newline%
\newline%
10) Fill in the blank with <, >, or =: 20 \_\_\_\_\_\_\_ 32%
\newline%
\newline%
\newline%
11) Fill in the blank with <, >, or =: 76 \_\_\_\_\_\_\_ 66%
\newline%
\newline%
\newline%
12) Fill in the blank with <, >, or =: 85 \_\_\_\_\_\_\_ 5%
\newline%
\newline%
\newline%
13) Fill in the blank with <, >, or =: 67 \_\_\_\_\_\_\_ 70%
\newline%
\newline%
\newline%
14) Fill in the blank with <, >, or =: 5 \_\_\_\_\_\_\_ 28%
\newline%
\newline%
\newline%
15) Fill in the blank with <, >, or =: 36 \_\_\_\_\_\_\_ 90%
\newline%
\newline%
\newline%
16) Fill in the blank with <, >, or =: 63 \_\_\_\_\_\_\_ 19%
\newline%
\newline%
\newline%
17) Fill in the blank with <, >, or =: 98 \_\_\_\_\_\_\_ 10%
\newline%
\newline%
\newline%
18) Fill in the blank with <, >, or =: 20 \_\_\_\_\_\_\_ 42%
\newline%
\newline%
\newline%
19) Fill in the blank with <, >, or =: 47 \_\_\_\_\_\_\_ 97%
\newline%
\newline%
\newline%
20) Fill in the blank with <, >, or =: 72 \_\_\_\_\_\_\_ 82%
\newline%
\newline%
\newline%
21) Fill in the blank with <, >, or =: 57 \_\_\_\_\_\_\_ 63%
\newline%
\newline%
\newline%
22) Fill in the blank with <, >, or =: 12 \_\_\_\_\_\_\_ 38%
\newline%
\newline%
\newline%
23) Fill in the blank with <, >, or =: 45 \_\_\_\_\_\_\_ 53%
\newline%
\newline%
\newline%
24) Fill in the blank with <, >, or =: 78 \_\_\_\_\_\_\_ 27%
\newline%
\newline%
\newline%
25) Fill in the blank with <, >, or =: 91 \_\_\_\_\_\_\_ 43%
\newline%
\newline%
\newline%
26) Sam has 95 papers. Rachel has 25 papers. Who has more papers?%
\newline%
\newline%
\newline%
27) Michael has 74 papers. Laura has 41 papers. Who has more papers?%
\newline%
\newline%
\newline%
28) Rachel has 53 pencils. Bob has 37 pencils. Who has more pencils?%
\newline%
\newline%
\newline%
29) Sally has 86 papers. Laura has 76 papers. Who has more papers?%
\newline%
\newline%
\newline%
30) Bob has 38 bottles. Laura has 62 bottles. Who has more bottles?%
\newline%
\newline%
\newline%
\pagebreak%
\large%
\begin{center}%
\textbf{Comparing Two Digit Numbers- Worksheet 3}%
\newline%
\end{center} \normalsize%
1) Fill in the blank with <, >, or =: 52 \_\_\_\_\_\_\_ 53%
\newline%
\newline%
\newline%
2) Fill in the blank with <, >, or =: 65 \_\_\_\_\_\_\_ 12%
\newline%
\newline%
\newline%
3) Fill in the blank with <, >, or =: 23 \_\_\_\_\_\_\_ 33%
\newline%
\newline%
\newline%
4) Fill in the blank with <, >, or =: 57 \_\_\_\_\_\_\_ 44%
\newline%
\newline%
\newline%
5) Fill in the blank with <, >, or =: 43 \_\_\_\_\_\_\_ 62%
\newline%
\newline%
\newline%
6) Fill in the blank with <, >, or =: 42 \_\_\_\_\_\_\_ 46%
\newline%
\newline%
\newline%
7) Fill in the blank with <, >, or =: 34 \_\_\_\_\_\_\_ 63%
\newline%
\newline%
\newline%
8) Fill in the blank with <, >, or =: 95 \_\_\_\_\_\_\_ 56%
\newline%
\newline%
\newline%
9) Fill in the blank with <, >, or =: 58 \_\_\_\_\_\_\_ 74%
\newline%
\newline%
\newline%
10) Fill in the blank with <, >, or =: 18 \_\_\_\_\_\_\_ 19%
\newline%
\newline%
\newline%
11) Fill in the blank with <, >, or =: 70 \_\_\_\_\_\_\_ 57%
\newline%
\newline%
\newline%
12) Fill in the blank with <, >, or =: 31 \_\_\_\_\_\_\_ 28%
\newline%
\newline%
\newline%
13) Fill in the blank with <, >, or =: 78 \_\_\_\_\_\_\_ 21%
\newline%
\newline%
\newline%
14) Fill in the blank with <, >, or =: 33 \_\_\_\_\_\_\_ 59%
\newline%
\newline%
\newline%
15) Fill in the blank with <, >, or =: 66 \_\_\_\_\_\_\_ 25%
\newline%
\newline%
\newline%
16) Fill in the blank with <, >, or =: 88 \_\_\_\_\_\_\_ 93%
\newline%
\newline%
\newline%
17) Fill in the blank with <, >, or =: 87 \_\_\_\_\_\_\_ 33%
\newline%
\newline%
\newline%
18) Fill in the blank with <, >, or =: 38 \_\_\_\_\_\_\_ 80%
\newline%
\newline%
\newline%
19) Fill in the blank with <, >, or =: 94 \_\_\_\_\_\_\_ 16%
\newline%
\newline%
\newline%
20) Fill in the blank with <, >, or =: 99 \_\_\_\_\_\_\_ 63%
\newline%
\newline%
\newline%
21) Fill in the blank with <, >, or =: 74 \_\_\_\_\_\_\_ 34%
\newline%
\newline%
\newline%
22) Fill in the blank with <, >, or =: 25 \_\_\_\_\_\_\_ 87%
\newline%
\newline%
\newline%
23) Fill in the blank with <, >, or =: 41 \_\_\_\_\_\_\_ 1%
\newline%
\newline%
\newline%
24) Fill in the blank with <, >, or =: 82 \_\_\_\_\_\_\_ 16%
\newline%
\newline%
\newline%
25) Fill in the blank with <, >, or =: 16 \_\_\_\_\_\_\_ 86%
\newline%
\newline%
\newline%
26) Rachel has 99 marbles. Alex has 99 marbles. Who has more marbles?%
\newline%
\newline%
\newline%
27) Rachel has 17 toys. James has 93 toys. Who has more toys?%
\newline%
\newline%
\newline%
28) Sam has 92 bottles. Bob has 9 bottles. Who has more bottles?%
\newline%
\newline%
\newline%
29) Rachel has 16 pencils. Bob has 31 pencils. Who has more pencils?%
\newline%
\newline%
\newline%
30) Alex has 90 toys. James has 87 toys. Who has more toys?%
\newline%
\newline%
\newline%
\pagebreak%
\large%
\begin{center}%
\textbf{Comparing Two Digit Numbers- Worksheet 4}%
\newline%
\end{center} \normalsize%
1) Fill in the blank with <, >, or =: 34 \_\_\_\_\_\_\_ 12%
\newline%
\newline%
\newline%
2) Fill in the blank with <, >, or =: 35 \_\_\_\_\_\_\_ 22%
\newline%
\newline%
\newline%
3) Fill in the blank with <, >, or =: 43 \_\_\_\_\_\_\_ 25%
\newline%
\newline%
\newline%
4) Fill in the blank with <, >, or =: 49 \_\_\_\_\_\_\_ 91%
\newline%
\newline%
\newline%
5) Fill in the blank with <, >, or =: 16 \_\_\_\_\_\_\_ 91%
\newline%
\newline%
\newline%
6) Fill in the blank with <, >, or =: 28 \_\_\_\_\_\_\_ 97%
\newline%
\newline%
\newline%
7) Fill in the blank with <, >, or =: 49 \_\_\_\_\_\_\_ 36%
\newline%
\newline%
\newline%
8) Fill in the blank with <, >, or =: 92 \_\_\_\_\_\_\_ 94%
\newline%
\newline%
\newline%
9) Fill in the blank with <, >, or =: 60 \_\_\_\_\_\_\_ 79%
\newline%
\newline%
\newline%
10) Fill in the blank with <, >, or =: 48 \_\_\_\_\_\_\_ 83%
\newline%
\newline%
\newline%
11) Fill in the blank with <, >, or =: 86 \_\_\_\_\_\_\_ 93%
\newline%
\newline%
\newline%
12) Fill in the blank with <, >, or =: 69 \_\_\_\_\_\_\_ 80%
\newline%
\newline%
\newline%
13) Fill in the blank with <, >, or =: 39 \_\_\_\_\_\_\_ 9%
\newline%
\newline%
\newline%
14) Fill in the blank with <, >, or =: 9 \_\_\_\_\_\_\_ 39%
\newline%
\newline%
\newline%
15) Fill in the blank with <, >, or =: 53 \_\_\_\_\_\_\_ 58%
\newline%
\newline%
\newline%
16) Fill in the blank with <, >, or =: 56 \_\_\_\_\_\_\_ 24%
\newline%
\newline%
\newline%
17) Fill in the blank with <, >, or =: 97 \_\_\_\_\_\_\_ 50%
\newline%
\newline%
\newline%
18) Fill in the blank with <, >, or =: 86 \_\_\_\_\_\_\_ 64%
\newline%
\newline%
\newline%
19) Fill in the blank with <, >, or =: 1 \_\_\_\_\_\_\_ 87%
\newline%
\newline%
\newline%
20) Fill in the blank with <, >, or =: 28 \_\_\_\_\_\_\_ 44%
\newline%
\newline%
\newline%
21) Fill in the blank with <, >, or =: 44 \_\_\_\_\_\_\_ 83%
\newline%
\newline%
\newline%
22) Fill in the blank with <, >, or =: 57 \_\_\_\_\_\_\_ 42%
\newline%
\newline%
\newline%
23) Fill in the blank with <, >, or =: 51 \_\_\_\_\_\_\_ 76%
\newline%
\newline%
\newline%
24) Fill in the blank with <, >, or =: 26 \_\_\_\_\_\_\_ 62%
\newline%
\newline%
\newline%
25) Fill in the blank with <, >, or =: 14 \_\_\_\_\_\_\_ 33%
\newline%
\newline%
\newline%
26) Sally has 81 dollars. Sam has 96 dollars. Who has more dollars?%
\newline%
\newline%
\newline%
27) Alex has 32 toys. Bob has 50 toys. Who has more toys?%
\newline%
\newline%
\newline%
28) Alex has 56 marbles. Bob has 18 marbles. Who has more marbles?%
\newline%
\newline%
\newline%
29) Alex has 19 books. Sally has 35 books. Who has more books?%
\newline%
\newline%
\newline%
30) Sam has 35 books. James has 16 books. Who has more books?%
\newline%
\newline%
\newline%
\pagebreak%
\large%
\begin{center}%
\textbf{Adding Numbers Within 20- Worksheet 1}%
\newline%
\end{center} \normalsize%
1) 11 + 11 = \_\_\_\_\_\_\_%
\newline%
\newline%
\newline%
2) 19 + 19 = \_\_\_\_\_\_\_%
\newline%
\newline%
\newline%
3) 59 + 64 = \_\_\_\_\_\_\_%
\newline%
\newline%
\newline%
4) 20 + 20 = \_\_\_\_\_\_\_%
\newline%
\newline%
\newline%
5) 4 + 4 = \_\_\_\_\_\_\_%
\newline%
\newline%
\newline%
6) 12 + 29 = \_\_\_\_\_\_\_%
\newline%
\newline%
\newline%
7) 18 + 18 = \_\_\_\_\_\_\_%
\newline%
\newline%
\newline%
8) 57 + 64 = \_\_\_\_\_\_\_%
\newline%
\newline%
\newline%
9) 8 + 8 = \_\_\_\_\_\_\_%
\newline%
\newline%
\newline%
10) 6 + 6 = \_\_\_\_\_\_\_%
\newline%
\newline%
\newline%
11) 16 + 16 = \_\_\_\_\_\_\_%
\newline%
\newline%
\newline%
12) 28 + 48 = \_\_\_\_\_\_\_%
\newline%
\newline%
\newline%
13) 33 + 44 = \_\_\_\_\_\_\_%
\newline%
\newline%
\newline%
14) 31 + 34 = \_\_\_\_\_\_\_%
\newline%
\newline%
\newline%
15) 7 + 7 = \_\_\_\_\_\_\_%
\newline%
\newline%
\newline%
16) 3 + 6 = \_\_\_\_\_\_\_%
\newline%
\newline%
\newline%
17) 6 + 10 = \_\_\_\_\_\_\_%
\newline%
\newline%
\newline%
18) 14 + 20 = \_\_\_\_\_\_\_%
\newline%
\newline%
\newline%
19) 37 + 56 = \_\_\_\_\_\_\_%
\newline%
\newline%
\newline%
20) 26 + 43 = \_\_\_\_\_\_\_%
\newline%
\newline%
\newline%
21) 17 + 18 = \_\_\_\_\_\_\_%
\newline%
\newline%
\newline%
22) 13 + 13 = \_\_\_\_\_\_\_%
\newline%
\newline%
\newline%
23) 51 + 52 = \_\_\_\_\_\_\_%
\newline%
\newline%
\newline%
24) 9 + 19 = \_\_\_\_\_\_\_%
\newline%
\newline%
\newline%
25) 22 + 23 = \_\_\_\_\_\_\_%
\newline%
\newline%
\newline%
26) James has 17 dollars. Laura has 17 more dollars than James. How many dollars does Laura have?%
\newline%
\newline%
\newline%
27) Alex has 48 pens. Sally has 58 more pens than Alex. How many pens does Sally have?%
\newline%
\newline%
\newline%
28) Michael has 58 papers. Sally has 64 more papers than Michael. How many papers does Sally have?%
\newline%
\newline%
\newline%
29) James has 61 dollars. Alex has 71 more dollars than James. How many dollars does Alex have?%
\newline%
\newline%
\newline%
30) Alex has 5 pencils. James has 18 more pencils than Alex. How many pencils does James have?%
\newline%
\newline%
\newline%
\pagebreak%
\large%
\begin{center}%
\textbf{Adding Numbers Within 20- Worksheet 2}%
\newline%
\end{center} \normalsize%
1) 32 + 35 = \_\_\_\_\_\_\_%
\newline%
\newline%
\newline%
2) 11 + 17 = \_\_\_\_\_\_\_%
\newline%
\newline%
\newline%
3) 19 + 19 = \_\_\_\_\_\_\_%
\newline%
\newline%
\newline%
4) 38 + 57 = \_\_\_\_\_\_\_%
\newline%
\newline%
\newline%
5) 17 + 17 = \_\_\_\_\_\_\_%
\newline%
\newline%
\newline%
6) 11 + 11 = \_\_\_\_\_\_\_%
\newline%
\newline%
\newline%
7) 7 + 26 = \_\_\_\_\_\_\_%
\newline%
\newline%
\newline%
8) 28 + 41 = \_\_\_\_\_\_\_%
\newline%
\newline%
\newline%
9) 4 + 8 = \_\_\_\_\_\_\_%
\newline%
\newline%
\newline%
10) 48 + 64 = \_\_\_\_\_\_\_%
\newline%
\newline%
\newline%
11) 5 + 5 = \_\_\_\_\_\_\_%
\newline%
\newline%
\newline%
12) 32 + 42 = \_\_\_\_\_\_\_%
\newline%
\newline%
\newline%
13) 14 + 28 = \_\_\_\_\_\_\_%
\newline%
\newline%
\newline%
14) 15 + 19 = \_\_\_\_\_\_\_%
\newline%
\newline%
\newline%
15) 69 + 80 = \_\_\_\_\_\_\_%
\newline%
\newline%
\newline%
16) 18 + 34 = \_\_\_\_\_\_\_%
\newline%
\newline%
\newline%
17) 4 + 21 = \_\_\_\_\_\_\_%
\newline%
\newline%
\newline%
18) 56 + 75 = \_\_\_\_\_\_\_%
\newline%
\newline%
\newline%
19) 6 + 6 = \_\_\_\_\_\_\_%
\newline%
\newline%
\newline%
20) 2 + 2 = \_\_\_\_\_\_\_%
\newline%
\newline%
\newline%
21) 56 + 66 = \_\_\_\_\_\_\_%
\newline%
\newline%
\newline%
22) 53 + 65 = \_\_\_\_\_\_\_%
\newline%
\newline%
\newline%
23) 45 + 65 = \_\_\_\_\_\_\_%
\newline%
\newline%
\newline%
24) 20 + 20 = \_\_\_\_\_\_\_%
\newline%
\newline%
\newline%
25) 51 + 68 = \_\_\_\_\_\_\_%
\newline%
\newline%
\newline%
26) Bob has 10 books. Michael has 10 more books than Bob. How many books does Michael have?%
\newline%
\newline%
\newline%
27) Sam has 18 papers. Sally has 18 more papers than Sam. How many papers does Sally have?%
\newline%
\newline%
\newline%
28) Rachel has 58 toys. Bob has 64 more toys than Rachel. How many toys does Bob have?%
\newline%
\newline%
\newline%
29) Rachel has 1 pencils. Sally has 1 more pencils than Rachel. How many pencils does Sally have?%
\newline%
\newline%
\newline%
30) Alex has 29 toys. Michael has 46 more toys than Alex. How many toys does Michael have?%
\newline%
\newline%
\newline%
\pagebreak%
\large%
\begin{center}%
\textbf{Adding Numbers Within 20- Worksheet 3}%
\newline%
\end{center} \normalsize%
1) 39 + 52 = \_\_\_\_\_\_\_%
\newline%
\newline%
\newline%
2) 14 + 14 = \_\_\_\_\_\_\_%
\newline%
\newline%
\newline%
3) 12 + 30 = \_\_\_\_\_\_\_%
\newline%
\newline%
\newline%
4) 2 + 17 = \_\_\_\_\_\_\_%
\newline%
\newline%
\newline%
5) 46 + 54 = \_\_\_\_\_\_\_%
\newline%
\newline%
\newline%
6) 44 + 61 = \_\_\_\_\_\_\_%
\newline%
\newline%
\newline%
7) 20 + 20 = \_\_\_\_\_\_\_%
\newline%
\newline%
\newline%
8) 36 + 48 = \_\_\_\_\_\_\_%
\newline%
\newline%
\newline%
9) 16 + 16 = \_\_\_\_\_\_\_%
\newline%
\newline%
\newline%
10) 52 + 71 = \_\_\_\_\_\_\_%
\newline%
\newline%
\newline%
11) 2 + 2 = \_\_\_\_\_\_\_%
\newline%
\newline%
\newline%
12) 44 + 56 = \_\_\_\_\_\_\_%
\newline%
\newline%
\newline%
13) 1 + 1 = \_\_\_\_\_\_\_%
\newline%
\newline%
\newline%
14) 19 + 19 = \_\_\_\_\_\_\_%
\newline%
\newline%
\newline%
15) 2 + 8 = \_\_\_\_\_\_\_%
\newline%
\newline%
\newline%
16) 43 + 53 = \_\_\_\_\_\_\_%
\newline%
\newline%
\newline%
17) 7 + 7 = \_\_\_\_\_\_\_%
\newline%
\newline%
\newline%
18) 58 + 60 = \_\_\_\_\_\_\_%
\newline%
\newline%
\newline%
19) 9 + 9 = \_\_\_\_\_\_\_%
\newline%
\newline%
\newline%
20) 69 + 73 = \_\_\_\_\_\_\_%
\newline%
\newline%
\newline%
21) 12 + 22 = \_\_\_\_\_\_\_%
\newline%
\newline%
\newline%
22) 6 + 21 = \_\_\_\_\_\_\_%
\newline%
\newline%
\newline%
23) 40 + 47 = \_\_\_\_\_\_\_%
\newline%
\newline%
\newline%
24) 8 + 8 = \_\_\_\_\_\_\_%
\newline%
\newline%
\newline%
25) 4 + 4 = \_\_\_\_\_\_\_%
\newline%
\newline%
\newline%
26) Bob has 13 pencils. Alex has 19 more pencils than Bob. How many pencils does Alex have?%
\newline%
\newline%
\newline%
27) Sally has 47 dollars. Michael has 64 more dollars than Sally. How many dollars does Michael have?%
\newline%
\newline%
\newline%
28) Sam has 13 marbles. Alex has 13 more marbles than Sam. How many marbles does Alex have?%
\newline%
\newline%
\newline%
29) Sally has 18 pens. Rachel has 18 more pens than Sally. How many pens does Rachel have?%
\newline%
\newline%
\newline%
30) James has 6 dollars. Laura has 6 more dollars than James. How many dollars does Laura have?%
\newline%
\newline%
\newline%
\pagebreak%
\large%
\begin{center}%
\textbf{Adding Numbers Within 20- Worksheet 4}%
\newline%
\end{center} \normalsize%
1) 44 + 45 = \_\_\_\_\_\_\_%
\newline%
\newline%
\newline%
2) 22 + 39 = \_\_\_\_\_\_\_%
\newline%
\newline%
\newline%
3) 70 + 88 = \_\_\_\_\_\_\_%
\newline%
\newline%
\newline%
4) 12 + 12 = \_\_\_\_\_\_\_%
\newline%
\newline%
\newline%
5) 28 + 44 = \_\_\_\_\_\_\_%
\newline%
\newline%
\newline%
6) 1 + 1 = \_\_\_\_\_\_\_%
\newline%
\newline%
\newline%
7) 61 + 71 = \_\_\_\_\_\_\_%
\newline%
\newline%
\newline%
8) 9 + 9 = \_\_\_\_\_\_\_%
\newline%
\newline%
\newline%
9) 10 + 13 = \_\_\_\_\_\_\_%
\newline%
\newline%
\newline%
10) 20 + 20 = \_\_\_\_\_\_\_%
\newline%
\newline%
\newline%
11) 45 + 55 = \_\_\_\_\_\_\_%
\newline%
\newline%
\newline%
12) 5 + 5 = \_\_\_\_\_\_\_%
\newline%
\newline%
\newline%
13) 67 + 74 = \_\_\_\_\_\_\_%
\newline%
\newline%
\newline%
14) 15 + 15 = \_\_\_\_\_\_\_%
\newline%
\newline%
\newline%
15) 47 + 51 = \_\_\_\_\_\_\_%
\newline%
\newline%
\newline%
16) 8 + 8 = \_\_\_\_\_\_\_%
\newline%
\newline%
\newline%
17) 2 + 2 = \_\_\_\_\_\_\_%
\newline%
\newline%
\newline%
18) 19 + 35 = \_\_\_\_\_\_\_%
\newline%
\newline%
\newline%
19) 68 + 71 = \_\_\_\_\_\_\_%
\newline%
\newline%
\newline%
20) 5 + 8 = \_\_\_\_\_\_\_%
\newline%
\newline%
\newline%
21) 10 + 14 = \_\_\_\_\_\_\_%
\newline%
\newline%
\newline%
22) 12 + 13 = \_\_\_\_\_\_\_%
\newline%
\newline%
\newline%
23) 10 + 10 = \_\_\_\_\_\_\_%
\newline%
\newline%
\newline%
24) 45 + 51 = \_\_\_\_\_\_\_%
\newline%
\newline%
\newline%
25) 19 + 25 = \_\_\_\_\_\_\_%
\newline%
\newline%
\newline%
26) James has 14 papers. Bob has 14 more papers than James. How many papers does Bob have?%
\newline%
\newline%
\newline%
27) James has 27 pencils. Laura has 46 more pencils than James. How many pencils does Laura have?%
\newline%
\newline%
\newline%
28) Laura has 6 pens. Michael has 6 more pens than Laura. How many pens does Michael have?%
\newline%
\newline%
\newline%
29) Sally has 7 pens. Bob has 7 more pens than Sally. How many pens does Bob have?%
\newline%
\newline%
\newline%
30) Alex has 19 books. Michael has 19 more books than Alex. How many books does Michael have?%
\newline%
\newline%
\newline%
\pagebreak%
\large%
\begin{center}%
\textbf{Subtracting Numbers Within 20- Worksheet 1}%
\newline%
\end{center} \normalsize%
1) 22 {-} 5 = \_\_\_\_\_\_\_%
\newline%
\newline%
\newline%
2) 22 {-} 17 = \_\_\_\_\_\_\_%
\newline%
\newline%
\newline%
3) 67 {-} 4 = \_\_\_\_\_\_\_%
\newline%
\newline%
\newline%
4) 28 {-} 17 = \_\_\_\_\_\_\_%
\newline%
\newline%
\newline%
5) 23 {-} 8 = \_\_\_\_\_\_\_%
\newline%
\newline%
\newline%
6) 32 {-} 3 = \_\_\_\_\_\_\_%
\newline%
\newline%
\newline%
7) 81 {-} 10 = \_\_\_\_\_\_\_%
\newline%
\newline%
\newline%
8) 62 {-} 3 = \_\_\_\_\_\_\_%
\newline%
\newline%
\newline%
9) 21 {-} 7 = \_\_\_\_\_\_\_%
\newline%
\newline%
\newline%
10) 28 {-} 14 = \_\_\_\_\_\_\_%
\newline%
\newline%
\newline%
11) 31 {-} 5 = \_\_\_\_\_\_\_%
\newline%
\newline%
\newline%
12) 80 {-} 2 = \_\_\_\_\_\_\_%
\newline%
\newline%
\newline%
13) 54 {-} 17 = \_\_\_\_\_\_\_%
\newline%
\newline%
\newline%
14) 52 {-} 18 = \_\_\_\_\_\_\_%
\newline%
\newline%
\newline%
15) 63 {-} 18 = \_\_\_\_\_\_\_%
\newline%
\newline%
\newline%
16) 23 {-} 2 = \_\_\_\_\_\_\_%
\newline%
\newline%
\newline%
17) 74 {-} 19 = \_\_\_\_\_\_\_%
\newline%
\newline%
\newline%
18) 86 {-} 11 = \_\_\_\_\_\_\_%
\newline%
\newline%
\newline%
19) 70 {-} 14 = \_\_\_\_\_\_\_%
\newline%
\newline%
\newline%
20) 41 {-} 18 = \_\_\_\_\_\_\_%
\newline%
\newline%
\newline%
21) 93 {-} 1 = \_\_\_\_\_\_\_%
\newline%
\newline%
\newline%
22) 66 {-} 5 = \_\_\_\_\_\_\_%
\newline%
\newline%
\newline%
23) 55 {-} 3 = \_\_\_\_\_\_\_%
\newline%
\newline%
\newline%
24) 90 {-} 18 = \_\_\_\_\_\_\_%
\newline%
\newline%
\newline%
25) 27 {-} 13 = \_\_\_\_\_\_\_%
\newline%
\newline%
\newline%
26) Laura has 67 pens. Michael has 2 less pens than Laura. How many pens does Michael have?%
\newline%
\newline%
\newline%
27) Sam has 64 papers. Alex has 10 less papers than Sam. How many papers does Alex have?%
\newline%
\newline%
\newline%
28) Laura has 80 dollars. Bob has 15 less dollars than Laura. How many dollars does Bob have?%
\newline%
\newline%
\newline%
29) James has 76 pens. Sally has 19 less pens than James. How many pens does Sally have?%
\newline%
\newline%
\newline%
30) James has 36 marbles. Sam has 4 less marbles than James. How many marbles does Sam have?%
\newline%
\newline%
\newline%
\pagebreak%
\large%
\begin{center}%
\textbf{Subtracting Numbers Within 20- Worksheet 2}%
\newline%
\end{center} \normalsize%
1) 61 {-} 19 = \_\_\_\_\_\_\_%
\newline%
\newline%
\newline%
2) 84 {-} 13 = \_\_\_\_\_\_\_%
\newline%
\newline%
\newline%
3) 62 {-} 5 = \_\_\_\_\_\_\_%
\newline%
\newline%
\newline%
4) 96 {-} 15 = \_\_\_\_\_\_\_%
\newline%
\newline%
\newline%
5) 25 {-} 10 = \_\_\_\_\_\_\_%
\newline%
\newline%
\newline%
6) 82 {-} 18 = \_\_\_\_\_\_\_%
\newline%
\newline%
\newline%
7) 32 {-} 17 = \_\_\_\_\_\_\_%
\newline%
\newline%
\newline%
8) 64 {-} 7 = \_\_\_\_\_\_\_%
\newline%
\newline%
\newline%
9) 24 {-} 13 = \_\_\_\_\_\_\_%
\newline%
\newline%
\newline%
10) 40 {-} 18 = \_\_\_\_\_\_\_%
\newline%
\newline%
\newline%
11) 97 {-} 1 = \_\_\_\_\_\_\_%
\newline%
\newline%
\newline%
12) 28 {-} 2 = \_\_\_\_\_\_\_%
\newline%
\newline%
\newline%
13) 81 {-} 10 = \_\_\_\_\_\_\_%
\newline%
\newline%
\newline%
14) 88 {-} 8 = \_\_\_\_\_\_\_%
\newline%
\newline%
\newline%
15) 71 {-} 6 = \_\_\_\_\_\_\_%
\newline%
\newline%
\newline%
16) 64 {-} 20 = \_\_\_\_\_\_\_%
\newline%
\newline%
\newline%
17) 97 {-} 3 = \_\_\_\_\_\_\_%
\newline%
\newline%
\newline%
18) 21 {-} 9 = \_\_\_\_\_\_\_%
\newline%
\newline%
\newline%
19) 21 {-} 13 = \_\_\_\_\_\_\_%
\newline%
\newline%
\newline%
20) 26 {-} 4 = \_\_\_\_\_\_\_%
\newline%
\newline%
\newline%
21) 49 {-} 9 = \_\_\_\_\_\_\_%
\newline%
\newline%
\newline%
22) 67 {-} 9 = \_\_\_\_\_\_\_%
\newline%
\newline%
\newline%
23) 46 {-} 11 = \_\_\_\_\_\_\_%
\newline%
\newline%
\newline%
24) 39 {-} 20 = \_\_\_\_\_\_\_%
\newline%
\newline%
\newline%
25) 47 {-} 8 = \_\_\_\_\_\_\_%
\newline%
\newline%
\newline%
26) Bob has 22 toys. Rachel has 6 less toys than Bob. How many toys does Rachel have?%
\newline%
\newline%
\newline%
27) Sam has 94 dollars. Alex has 4 less dollars than Sam. How many dollars does Alex have?%
\newline%
\newline%
\newline%
28) Michael has 50 pens. James has 6 less pens than Michael. How many pens does James have?%
\newline%
\newline%
\newline%
29) Laura has 53 toys. Sally has 10 less toys than Laura. How many toys does Sally have?%
\newline%
\newline%
\newline%
30) Bob has 48 marbles. Sam has 19 less marbles than Bob. How many marbles does Sam have?%
\newline%
\newline%
\newline%
\pagebreak%
\large%
\begin{center}%
\textbf{Subtracting Numbers Within 20- Worksheet 3}%
\newline%
\end{center} \normalsize%
1) 37 {-} 1 = \_\_\_\_\_\_\_%
\newline%
\newline%
\newline%
2) 27 {-} 7 = \_\_\_\_\_\_\_%
\newline%
\newline%
\newline%
3) 83 {-} 1 = \_\_\_\_\_\_\_%
\newline%
\newline%
\newline%
4) 46 {-} 5 = \_\_\_\_\_\_\_%
\newline%
\newline%
\newline%
5) 50 {-} 9 = \_\_\_\_\_\_\_%
\newline%
\newline%
\newline%
6) 90 {-} 2 = \_\_\_\_\_\_\_%
\newline%
\newline%
\newline%
7) 93 {-} 17 = \_\_\_\_\_\_\_%
\newline%
\newline%
\newline%
8) 42 {-} 9 = \_\_\_\_\_\_\_%
\newline%
\newline%
\newline%
9) 56 {-} 3 = \_\_\_\_\_\_\_%
\newline%
\newline%
\newline%
10) 31 {-} 15 = \_\_\_\_\_\_\_%
\newline%
\newline%
\newline%
11) 39 {-} 6 = \_\_\_\_\_\_\_%
\newline%
\newline%
\newline%
12) 50 {-} 11 = \_\_\_\_\_\_\_%
\newline%
\newline%
\newline%
13) 93 {-} 18 = \_\_\_\_\_\_\_%
\newline%
\newline%
\newline%
14) 28 {-} 6 = \_\_\_\_\_\_\_%
\newline%
\newline%
\newline%
15) 51 {-} 4 = \_\_\_\_\_\_\_%
\newline%
\newline%
\newline%
16) 89 {-} 10 = \_\_\_\_\_\_\_%
\newline%
\newline%
\newline%
17) 36 {-} 10 = \_\_\_\_\_\_\_%
\newline%
\newline%
\newline%
18) 87 {-} 3 = \_\_\_\_\_\_\_%
\newline%
\newline%
\newline%
19) 71 {-} 18 = \_\_\_\_\_\_\_%
\newline%
\newline%
\newline%
20) 87 {-} 15 = \_\_\_\_\_\_\_%
\newline%
\newline%
\newline%
21) 50 {-} 20 = \_\_\_\_\_\_\_%
\newline%
\newline%
\newline%
22) 69 {-} 1 = \_\_\_\_\_\_\_%
\newline%
\newline%
\newline%
23) 32 {-} 16 = \_\_\_\_\_\_\_%
\newline%
\newline%
\newline%
24) 66 {-} 11 = \_\_\_\_\_\_\_%
\newline%
\newline%
\newline%
25) 88 {-} 4 = \_\_\_\_\_\_\_%
\newline%
\newline%
\newline%
26) Alex has 73 pencils. Laura has 10 less pencils than Alex. How many pencils does Laura have?%
\newline%
\newline%
\newline%
27) Sally has 51 papers. Sam has 19 less papers than Sally. How many papers does Sam have?%
\newline%
\newline%
\newline%
28) Bob has 49 pens. James has 11 less pens than Bob. How many pens does James have?%
\newline%
\newline%
\newline%
29) Alex has 28 papers. Rachel has 14 less papers than Alex. How many papers does Rachel have?%
\newline%
\newline%
\newline%
30) Alex has 21 pens. James has 4 less pens than Alex. How many pens does James have?%
\newline%
\newline%
\newline%
\pagebreak%
\large%
\begin{center}%
\textbf{Subtracting Numbers Within 20- Worksheet 4}%
\newline%
\end{center} \normalsize%
1) 79 {-} 15 = \_\_\_\_\_\_\_%
\newline%
\newline%
\newline%
2) 30 {-} 10 = \_\_\_\_\_\_\_%
\newline%
\newline%
\newline%
3) 57 {-} 16 = \_\_\_\_\_\_\_%
\newline%
\newline%
\newline%
4) 40 {-} 16 = \_\_\_\_\_\_\_%
\newline%
\newline%
\newline%
5) 44 {-} 18 = \_\_\_\_\_\_\_%
\newline%
\newline%
\newline%
6) 60 {-} 6 = \_\_\_\_\_\_\_%
\newline%
\newline%
\newline%
7) 41 {-} 12 = \_\_\_\_\_\_\_%
\newline%
\newline%
\newline%
8) 33 {-} 19 = \_\_\_\_\_\_\_%
\newline%
\newline%
\newline%
9) 84 {-} 5 = \_\_\_\_\_\_\_%
\newline%
\newline%
\newline%
10) 39 {-} 17 = \_\_\_\_\_\_\_%
\newline%
\newline%
\newline%
11) 59 {-} 5 = \_\_\_\_\_\_\_%
\newline%
\newline%
\newline%
12) 27 {-} 17 = \_\_\_\_\_\_\_%
\newline%
\newline%
\newline%
13) 74 {-} 6 = \_\_\_\_\_\_\_%
\newline%
\newline%
\newline%
14) 61 {-} 11 = \_\_\_\_\_\_\_%
\newline%
\newline%
\newline%
15) 97 {-} 6 = \_\_\_\_\_\_\_%
\newline%
\newline%
\newline%
16) 78 {-} 1 = \_\_\_\_\_\_\_%
\newline%
\newline%
\newline%
17) 25 {-} 12 = \_\_\_\_\_\_\_%
\newline%
\newline%
\newline%
18) 66 {-} 4 = \_\_\_\_\_\_\_%
\newline%
\newline%
\newline%
19) 42 {-} 12 = \_\_\_\_\_\_\_%
\newline%
\newline%
\newline%
20) 25 {-} 6 = \_\_\_\_\_\_\_%
\newline%
\newline%
\newline%
21) 60 {-} 15 = \_\_\_\_\_\_\_%
\newline%
\newline%
\newline%
22) 79 {-} 16 = \_\_\_\_\_\_\_%
\newline%
\newline%
\newline%
23) 55 {-} 15 = \_\_\_\_\_\_\_%
\newline%
\newline%
\newline%
24) 93 {-} 1 = \_\_\_\_\_\_\_%
\newline%
\newline%
\newline%
25) 69 {-} 5 = \_\_\_\_\_\_\_%
\newline%
\newline%
\newline%
26) Michael has 45 books. Sally has 17 less books than Michael. How many books does Sally have?%
\newline%
\newline%
\newline%
27) Bob has 66 pencils. Rachel has 8 less pencils than Bob. How many pencils does Rachel have?%
\newline%
\newline%
\newline%
28) Rachel has 90 bottles. Alex has 2 less bottles than Rachel. How many bottles does Alex have?%
\newline%
\newline%
\newline%
29) Rachel has 56 books. Laura has 7 less books than Rachel. How many books does Laura have?%
\newline%
\newline%
\newline%
30) Sally has 75 bottles. Laura has 15 less bottles than Sally. How many bottles does Laura have?%
\newline%
\newline%
\newline%
\pagebreak%
\large%
\begin{center}%
\textbf{Finding the Missing Number- Worksheet 1}%
\newline%
\end{center} \normalsize%
1) Fill in the blank: 36 + \_\_\_\_\_\_\_\_ = 62%
\newline%
\newline%
\newline%
2) Fill in the blank: 25 + \_\_\_\_\_\_\_\_ = 60%
\newline%
\newline%
\newline%
3) Fill in the blank: 13 + \_\_\_\_\_\_\_\_ = 88%
\newline%
\newline%
\newline%
4) Fill in the blank: 40 + \_\_\_\_\_\_\_\_ = 67%
\newline%
\newline%
\newline%
5) Fill in the blank: 12 + \_\_\_\_\_\_\_\_ = 47%
\newline%
\newline%
\newline%
6) Fill in the blank: 16 + \_\_\_\_\_\_\_\_ = 81%
\newline%
\newline%
\newline%
7) Fill in the blank: 36 + \_\_\_\_\_\_\_\_ = 51%
\newline%
\newline%
\newline%
8) Fill in the blank: 25 + \_\_\_\_\_\_\_\_ = 66%
\newline%
\newline%
\newline%
9) Fill in the blank: 42 + \_\_\_\_\_\_\_\_ = 88%
\newline%
\newline%
\newline%
10) Fill in the blank: 33 + \_\_\_\_\_\_\_\_ = 64%
\newline%
\newline%
\newline%
11) Fill in the blank: 26 + \_\_\_\_\_\_\_\_ = 85%
\newline%
\newline%
\newline%
12) Fill in the blank: 39 + \_\_\_\_\_\_\_\_ = 64%
\newline%
\newline%
\newline%
13) Fill in the blank: 28 + \_\_\_\_\_\_\_\_ = 46%
\newline%
\newline%
\newline%
14) Fill in the blank: 43 + \_\_\_\_\_\_\_\_ = 58%
\newline%
\newline%
\newline%
15) Fill in the blank: 12 + \_\_\_\_\_\_\_\_ = 86%
\newline%
\newline%
\newline%
16) Fill in the blank: 7 + \_\_\_\_\_\_\_\_ = 67%
\newline%
\newline%
\newline%
17) Fill in the blank: 15 + \_\_\_\_\_\_\_\_ = 88%
\newline%
\newline%
\newline%
18) Fill in the blank: 4 + \_\_\_\_\_\_\_\_ = 54%
\newline%
\newline%
\newline%
19) Fill in the blank: 12 + \_\_\_\_\_\_\_\_ = 72%
\newline%
\newline%
\newline%
20) Fill in the blank: 7 + \_\_\_\_\_\_\_\_ = 78%
\newline%
\newline%
\newline%
21) Fill in the blank: 33 + \_\_\_\_\_\_\_\_ = 66%
\newline%
\newline%
\newline%
22) Fill in the blank: 26 + \_\_\_\_\_\_\_\_ = 70%
\newline%
\newline%
\newline%
23) Fill in the blank: 34 + \_\_\_\_\_\_\_\_ = 56%
\newline%
\newline%
\newline%
24) Fill in the blank: 28 + \_\_\_\_\_\_\_\_ = 50%
\newline%
\newline%
\newline%
25) Fill in the blank: 40 + \_\_\_\_\_\_\_\_ = 76%
\newline%
\newline%
\newline%
26) Michael has 21 toys. Bob has an unknown number of toys. In total, they have 75 toys. How many toys does Bob have?%
\newline%
\newline%
\newline%
27) Rachel has 33 toys. Sam has an unknown number of toys. In total, they have 58 toys. How many toys does Sam have?%
\newline%
\newline%
\newline%
28) Sally has 31 papers. Laura has an unknown number of papers. In total, they have 87 papers. How many papers does Laura have?%
\newline%
\newline%
\newline%
29) Rachel has 36 marbles. Sally has an unknown number of marbles. In total, they have 90 marbles. How many marbles does Sally have?%
\newline%
\newline%
\newline%
30) Bob has 41 toys. Laura has an unknown number of toys. In total, they have 54 toys. How many toys does Laura have?%
\newline%
\newline%
\newline%
\pagebreak%
\large%
\begin{center}%
\textbf{Finding the Missing Number- Worksheet 2}%
\newline%
\end{center} \normalsize%
1) Fill in the blank: 25 + \_\_\_\_\_\_\_\_ = 58%
\newline%
\newline%
\newline%
2) Fill in the blank: 2 + \_\_\_\_\_\_\_\_ = 62%
\newline%
\newline%
\newline%
3) Fill in the blank: 30 + \_\_\_\_\_\_\_\_ = 59%
\newline%
\newline%
\newline%
4) Fill in the blank: 16 + \_\_\_\_\_\_\_\_ = 77%
\newline%
\newline%
\newline%
5) Fill in the blank: 10 + \_\_\_\_\_\_\_\_ = 47%
\newline%
\newline%
\newline%
6) Fill in the blank: 41 + \_\_\_\_\_\_\_\_ = 49%
\newline%
\newline%
\newline%
7) Fill in the blank: 29 + \_\_\_\_\_\_\_\_ = 50%
\newline%
\newline%
\newline%
8) Fill in the blank: 9 + \_\_\_\_\_\_\_\_ = 60%
\newline%
\newline%
\newline%
9) Fill in the blank: 26 + \_\_\_\_\_\_\_\_ = 76%
\newline%
\newline%
\newline%
10) Fill in the blank: 5 + \_\_\_\_\_\_\_\_ = 84%
\newline%
\newline%
\newline%
11) Fill in the blank: 38 + \_\_\_\_\_\_\_\_ = 72%
\newline%
\newline%
\newline%
12) Fill in the blank: 45 + \_\_\_\_\_\_\_\_ = 47%
\newline%
\newline%
\newline%
13) Fill in the blank: 14 + \_\_\_\_\_\_\_\_ = 88%
\newline%
\newline%
\newline%
14) Fill in the blank: 26 + \_\_\_\_\_\_\_\_ = 60%
\newline%
\newline%
\newline%
15) Fill in the blank: 38 + \_\_\_\_\_\_\_\_ = 65%
\newline%
\newline%
\newline%
16) Fill in the blank: 42 + \_\_\_\_\_\_\_\_ = 90%
\newline%
\newline%
\newline%
17) Fill in the blank: 31 + \_\_\_\_\_\_\_\_ = 78%
\newline%
\newline%
\newline%
18) Fill in the blank: 32 + \_\_\_\_\_\_\_\_ = 54%
\newline%
\newline%
\newline%
19) Fill in the blank: 37 + \_\_\_\_\_\_\_\_ = 49%
\newline%
\newline%
\newline%
20) Fill in the blank: 17 + \_\_\_\_\_\_\_\_ = 66%
\newline%
\newline%
\newline%
21) Fill in the blank: 37 + \_\_\_\_\_\_\_\_ = 57%
\newline%
\newline%
\newline%
22) Fill in the blank: 23 + \_\_\_\_\_\_\_\_ = 59%
\newline%
\newline%
\newline%
23) Fill in the blank: 10 + \_\_\_\_\_\_\_\_ = 71%
\newline%
\newline%
\newline%
24) Fill in the blank: 6 + \_\_\_\_\_\_\_\_ = 58%
\newline%
\newline%
\newline%
25) Fill in the blank: 7 + \_\_\_\_\_\_\_\_ = 89%
\newline%
\newline%
\newline%
26) Michael has 30 marbles. Sam has an unknown number of marbles. In total, they have 88 marbles. How many marbles does Sam have?%
\newline%
\newline%
\newline%
27) Michael has 42 pencils. Bob has an unknown number of pencils. In total, they have 66 pencils. How many pencils does Bob have?%
\newline%
\newline%
\newline%
28) Sam has 7 dollars. Bob has an unknown number of dollars. In total, they have 78 dollars. How many dollars does Bob have?%
\newline%
\newline%
\newline%
29) Bob has 27 bottles. Laura has an unknown number of bottles. In total, they have 72 bottles. How many bottles does Laura have?%
\newline%
\newline%
\newline%
30) Sally has 40 pencils. Rachel has an unknown number of pencils. In total, they have 88 pencils. How many pencils does Rachel have?%
\newline%
\newline%
\newline%
\pagebreak%
\large%
\begin{center}%
\textbf{Finding the Missing Number- Worksheet 3}%
\newline%
\end{center} \normalsize%
1) Fill in the blank: 31 + \_\_\_\_\_\_\_\_ = 71%
\newline%
\newline%
\newline%
2) Fill in the blank: 26 + \_\_\_\_\_\_\_\_ = 66%
\newline%
\newline%
\newline%
3) Fill in the blank: 10 + \_\_\_\_\_\_\_\_ = 69%
\newline%
\newline%
\newline%
4) Fill in the blank: 37 + \_\_\_\_\_\_\_\_ = 53%
\newline%
\newline%
\newline%
5) Fill in the blank: 20 + \_\_\_\_\_\_\_\_ = 49%
\newline%
\newline%
\newline%
6) Fill in the blank: 20 + \_\_\_\_\_\_\_\_ = 66%
\newline%
\newline%
\newline%
7) Fill in the blank: 14 + \_\_\_\_\_\_\_\_ = 70%
\newline%
\newline%
\newline%
8) Fill in the blank: 23 + \_\_\_\_\_\_\_\_ = 51%
\newline%
\newline%
\newline%
9) Fill in the blank: 3 + \_\_\_\_\_\_\_\_ = 72%
\newline%
\newline%
\newline%
10) Fill in the blank: 44 + \_\_\_\_\_\_\_\_ = 51%
\newline%
\newline%
\newline%
11) Fill in the blank: 3 + \_\_\_\_\_\_\_\_ = 54%
\newline%
\newline%
\newline%
12) Fill in the blank: 40 + \_\_\_\_\_\_\_\_ = 90%
\newline%
\newline%
\newline%
13) Fill in the blank: 28 + \_\_\_\_\_\_\_\_ = 64%
\newline%
\newline%
\newline%
14) Fill in the blank: 17 + \_\_\_\_\_\_\_\_ = 46%
\newline%
\newline%
\newline%
15) Fill in the blank: 8 + \_\_\_\_\_\_\_\_ = 90%
\newline%
\newline%
\newline%
16) Fill in the blank: 18 + \_\_\_\_\_\_\_\_ = 58%
\newline%
\newline%
\newline%
17) Fill in the blank: 31 + \_\_\_\_\_\_\_\_ = 68%
\newline%
\newline%
\newline%
18) Fill in the blank: 8 + \_\_\_\_\_\_\_\_ = 76%
\newline%
\newline%
\newline%
19) Fill in the blank: 27 + \_\_\_\_\_\_\_\_ = 71%
\newline%
\newline%
\newline%
20) Fill in the blank: 18 + \_\_\_\_\_\_\_\_ = 54%
\newline%
\newline%
\newline%
21) Fill in the blank: 28 + \_\_\_\_\_\_\_\_ = 85%
\newline%
\newline%
\newline%
22) Fill in the blank: 45 + \_\_\_\_\_\_\_\_ = 47%
\newline%
\newline%
\newline%
23) Fill in the blank: 34 + \_\_\_\_\_\_\_\_ = 60%
\newline%
\newline%
\newline%
24) Fill in the blank: 33 + \_\_\_\_\_\_\_\_ = 74%
\newline%
\newline%
\newline%
25) Fill in the blank: 31 + \_\_\_\_\_\_\_\_ = 86%
\newline%
\newline%
\newline%
26) Rachel has 6 dollars. James has an unknown number of dollars. In total, they have 89 dollars. How many dollars does James have?%
\newline%
\newline%
\newline%
27) Rachel has 40 bottles. Laura has an unknown number of bottles. In total, they have 69 bottles. How many bottles does Laura have?%
\newline%
\newline%
\newline%
28) Laura has 23 bottles. Rachel has an unknown number of bottles. In total, they have 71 bottles. How many bottles does Rachel have?%
\newline%
\newline%
\newline%
29) Bob has 17 marbles. Rachel has an unknown number of marbles. In total, they have 87 marbles. How many marbles does Rachel have?%
\newline%
\newline%
\newline%
30) Sally has 40 pens. Rachel has an unknown number of pens. In total, they have 87 pens. How many pens does Rachel have?%
\newline%
\newline%
\newline%
\pagebreak%
\large%
\begin{center}%
\textbf{Finding the Missing Number- Worksheet 4}%
\newline%
\end{center} \normalsize%
1) Fill in the blank: 16 + \_\_\_\_\_\_\_\_ = 54%
\newline%
\newline%
\newline%
2) Fill in the blank: 28 + \_\_\_\_\_\_\_\_ = 88%
\newline%
\newline%
\newline%
3) Fill in the blank: 45 + \_\_\_\_\_\_\_\_ = 54%
\newline%
\newline%
\newline%
4) Fill in the blank: 30 + \_\_\_\_\_\_\_\_ = 83%
\newline%
\newline%
\newline%
5) Fill in the blank: 10 + \_\_\_\_\_\_\_\_ = 73%
\newline%
\newline%
\newline%
6) Fill in the blank: 24 + \_\_\_\_\_\_\_\_ = 74%
\newline%
\newline%
\newline%
7) Fill in the blank: 24 + \_\_\_\_\_\_\_\_ = 76%
\newline%
\newline%
\newline%
8) Fill in the blank: 11 + \_\_\_\_\_\_\_\_ = 64%
\newline%
\newline%
\newline%
9) Fill in the blank: 18 + \_\_\_\_\_\_\_\_ = 49%
\newline%
\newline%
\newline%
10) Fill in the blank: 44 + \_\_\_\_\_\_\_\_ = 76%
\newline%
\newline%
\newline%
11) Fill in the blank: 43 + \_\_\_\_\_\_\_\_ = 58%
\newline%
\newline%
\newline%
12) Fill in the blank: 36 + \_\_\_\_\_\_\_\_ = 65%
\newline%
\newline%
\newline%
13) Fill in the blank: 32 + \_\_\_\_\_\_\_\_ = 63%
\newline%
\newline%
\newline%
14) Fill in the blank: 3 + \_\_\_\_\_\_\_\_ = 85%
\newline%
\newline%
\newline%
15) Fill in the blank: 41 + \_\_\_\_\_\_\_\_ = 86%
\newline%
\newline%
\newline%
16) Fill in the blank: 45 + \_\_\_\_\_\_\_\_ = 59%
\newline%
\newline%
\newline%
17) Fill in the blank: 34 + \_\_\_\_\_\_\_\_ = 53%
\newline%
\newline%
\newline%
18) Fill in the blank: 16 + \_\_\_\_\_\_\_\_ = 84%
\newline%
\newline%
\newline%
19) Fill in the blank: 38 + \_\_\_\_\_\_\_\_ = 82%
\newline%
\newline%
\newline%
20) Fill in the blank: 39 + \_\_\_\_\_\_\_\_ = 83%
\newline%
\newline%
\newline%
21) Fill in the blank: 45 + \_\_\_\_\_\_\_\_ = 81%
\newline%
\newline%
\newline%
22) Fill in the blank: 10 + \_\_\_\_\_\_\_\_ = 69%
\newline%
\newline%
\newline%
23) Fill in the blank: 42 + \_\_\_\_\_\_\_\_ = 56%
\newline%
\newline%
\newline%
24) Fill in the blank: 7 + \_\_\_\_\_\_\_\_ = 69%
\newline%
\newline%
\newline%
25) Fill in the blank: 13 + \_\_\_\_\_\_\_\_ = 55%
\newline%
\newline%
\newline%
26) Sam has 7 bottles. Michael has an unknown number of bottles. In total, they have 72 bottles. How many bottles does Michael have?%
\newline%
\newline%
\newline%
27) Rachel has 23 pens. Michael has an unknown number of pens. In total, they have 89 pens. How many pens does Michael have?%
\newline%
\newline%
\newline%
28) Laura has 41 papers. James has an unknown number of papers. In total, they have 49 papers. How many papers does James have?%
\newline%
\newline%
\newline%
29) Bob has 18 toys. Michael has an unknown number of toys. In total, they have 65 toys. How many toys does Michael have?%
\newline%
\newline%
\newline%
30) Laura has 12 books. Sam has an unknown number of books. In total, they have 71 books. How many books does Sam have?%
\newline%
\newline%
\newline%
\pagebreak%
\large%
\begin{center}%
\textbf{Adding Three Numbers- Worksheet 1}%
\newline%
\end{center} \normalsize%
1) 7 + 12 + 11 = \_\_\_\_%
\newline%
\newline%
\newline%
2) 9 + 10 + 16 = \_\_\_\_%
\newline%
\newline%
\newline%
3) 9 + 19 + 19 = \_\_\_\_%
\newline%
\newline%
\newline%
4) 14 + 6 + 4 = \_\_\_\_%
\newline%
\newline%
\newline%
5) 3 + 5 + 17 = \_\_\_\_%
\newline%
\newline%
\newline%
6) 7 + 8 + 20 = \_\_\_\_%
\newline%
\newline%
\newline%
7) 6 + 8 + 16 = \_\_\_\_%
\newline%
\newline%
\newline%
8) 14 + 9 + 8 = \_\_\_\_%
\newline%
\newline%
\newline%
9) 17 + 8 + 20 = \_\_\_\_%
\newline%
\newline%
\newline%
10) 17 + 18 + 14 = \_\_\_\_%
\newline%
\newline%
\newline%
11) 15 + 16 + 7 = \_\_\_\_%
\newline%
\newline%
\newline%
12) 20 + 2 + 12 = \_\_\_\_%
\newline%
\newline%
\newline%
13) 20 + 20 + 3 = \_\_\_\_%
\newline%
\newline%
\newline%
14) 6 + 16 + 8 = \_\_\_\_%
\newline%
\newline%
\newline%
15) 5 + 14 + 9 = \_\_\_\_%
\newline%
\newline%
\newline%
16) 2 + 13 + 14 = \_\_\_\_%
\newline%
\newline%
\newline%
17) 3 + 14 + 9 = \_\_\_\_%
\newline%
\newline%
\newline%
18) 17 + 13 + 7 = \_\_\_\_%
\newline%
\newline%
\newline%
19) 14 + 12 + 12 = \_\_\_\_%
\newline%
\newline%
\newline%
20) 4 + 13 + 18 = \_\_\_\_%
\newline%
\newline%
\newline%
21) 6 + 9 + 4 = \_\_\_\_%
\newline%
\newline%
\newline%
22) 7 + 14 + 14 = \_\_\_\_%
\newline%
\newline%
\newline%
23) 17 + 19 + 15 = \_\_\_\_%
\newline%
\newline%
\newline%
24) 14 + 17 + 4 = \_\_\_\_%
\newline%
\newline%
\newline%
25) 15 + 16 + 11 = \_\_\_\_%
\newline%
\newline%
\newline%
26) Laura has 16 bottles. Sam has 15 bottles. Alex has 9 bottles. How many bottles do they have in all?%
\newline%
\newline%
\newline%
27) James has 7 books. Sam has 10 books. Bob has 11 books. How many books do they have in all?%
\newline%
\newline%
\newline%
28) Bob has 7 marbles. Michael has 16 marbles. James has 8 marbles. How many marbles do they have in all?%
\newline%
\newline%
\newline%
29) Alex has 10 dollars. Michael has 11 dollars. Bob has 2 dollars. How many dollars do they have in all?%
\newline%
\newline%
\newline%
30) Bob has 11 marbles. Rachel has 16 marbles. Alex has 13 marbles. How many marbles do they have in all?%
\newline%
\newline%
\newline%
\pagebreak%
\large%
\begin{center}%
\textbf{Adding Three Numbers- Worksheet 2}%
\newline%
\end{center} \normalsize%
1) 7 + 11 + 17 = \_\_\_\_%
\newline%
\newline%
\newline%
2) 13 + 20 + 16 = \_\_\_\_%
\newline%
\newline%
\newline%
3) 2 + 15 + 6 = \_\_\_\_%
\newline%
\newline%
\newline%
4) 2 + 12 + 14 = \_\_\_\_%
\newline%
\newline%
\newline%
5) 17 + 12 + 20 = \_\_\_\_%
\newline%
\newline%
\newline%
6) 7 + 19 + 7 = \_\_\_\_%
\newline%
\newline%
\newline%
7) 5 + 3 + 17 = \_\_\_\_%
\newline%
\newline%
\newline%
8) 7 + 15 + 5 = \_\_\_\_%
\newline%
\newline%
\newline%
9) 4 + 15 + 11 = \_\_\_\_%
\newline%
\newline%
\newline%
10) 17 + 9 + 5 = \_\_\_\_%
\newline%
\newline%
\newline%
11) 3 + 6 + 14 = \_\_\_\_%
\newline%
\newline%
\newline%
12) 16 + 6 + 9 = \_\_\_\_%
\newline%
\newline%
\newline%
13) 12 + 11 + 18 = \_\_\_\_%
\newline%
\newline%
\newline%
14) 8 + 12 + 6 = \_\_\_\_%
\newline%
\newline%
\newline%
15) 6 + 18 + 4 = \_\_\_\_%
\newline%
\newline%
\newline%
16) 14 + 13 + 10 = \_\_\_\_%
\newline%
\newline%
\newline%
17) 11 + 17 + 2 = \_\_\_\_%
\newline%
\newline%
\newline%
18) 8 + 8 + 9 = \_\_\_\_%
\newline%
\newline%
\newline%
19) 13 + 4 + 3 = \_\_\_\_%
\newline%
\newline%
\newline%
20) 13 + 8 + 3 = \_\_\_\_%
\newline%
\newline%
\newline%
21) 14 + 2 + 9 = \_\_\_\_%
\newline%
\newline%
\newline%
22) 14 + 9 + 18 = \_\_\_\_%
\newline%
\newline%
\newline%
23) 11 + 18 + 11 = \_\_\_\_%
\newline%
\newline%
\newline%
24) 15 + 11 + 15 = \_\_\_\_%
\newline%
\newline%
\newline%
25) 12 + 8 + 3 = \_\_\_\_%
\newline%
\newline%
\newline%
26) Sam has 18 books. Bob has 10 books. Rachel has 7 books. How many books do they have in all?%
\newline%
\newline%
\newline%
27) Rachel has 5 pens. Sally has 3 pens. James has 7 pens. How many pens do they have in all?%
\newline%
\newline%
\newline%
28) Sally has 19 books. Alex has 16 books. Rachel has 19 books. How many books do they have in all?%
\newline%
\newline%
\newline%
29) Sam has 12 papers. Laura has 20 papers. James has 16 papers. How many papers do they have in all?%
\newline%
\newline%
\newline%
30) Alex has 9 books. Sam has 17 books. Bob has 4 books. How many books do they have in all?%
\newline%
\newline%
\newline%
\pagebreak%
\large%
\begin{center}%
\textbf{Adding Three Numbers- Worksheet 3}%
\newline%
\end{center} \normalsize%
1) 13 + 18 + 4 = \_\_\_\_%
\newline%
\newline%
\newline%
2) 16 + 3 + 12 = \_\_\_\_%
\newline%
\newline%
\newline%
3) 20 + 6 + 18 = \_\_\_\_%
\newline%
\newline%
\newline%
4) 16 + 20 + 16 = \_\_\_\_%
\newline%
\newline%
\newline%
5) 12 + 14 + 20 = \_\_\_\_%
\newline%
\newline%
\newline%
6) 10 + 7 + 7 = \_\_\_\_%
\newline%
\newline%
\newline%
7) 2 + 8 + 14 = \_\_\_\_%
\newline%
\newline%
\newline%
8) 14 + 13 + 20 = \_\_\_\_%
\newline%
\newline%
\newline%
9) 5 + 3 + 17 = \_\_\_\_%
\newline%
\newline%
\newline%
10) 15 + 8 + 8 = \_\_\_\_%
\newline%
\newline%
\newline%
11) 15 + 5 + 13 = \_\_\_\_%
\newline%
\newline%
\newline%
12) 18 + 9 + 3 = \_\_\_\_%
\newline%
\newline%
\newline%
13) 4 + 10 + 5 = \_\_\_\_%
\newline%
\newline%
\newline%
14) 5 + 10 + 8 = \_\_\_\_%
\newline%
\newline%
\newline%
15) 12 + 18 + 12 = \_\_\_\_%
\newline%
\newline%
\newline%
16) 17 + 10 + 20 = \_\_\_\_%
\newline%
\newline%
\newline%
17) 20 + 18 + 2 = \_\_\_\_%
\newline%
\newline%
\newline%
18) 2 + 11 + 2 = \_\_\_\_%
\newline%
\newline%
\newline%
19) 13 + 5 + 6 = \_\_\_\_%
\newline%
\newline%
\newline%
20) 17 + 17 + 19 = \_\_\_\_%
\newline%
\newline%
\newline%
21) 14 + 19 + 4 = \_\_\_\_%
\newline%
\newline%
\newline%
22) 13 + 3 + 8 = \_\_\_\_%
\newline%
\newline%
\newline%
23) 11 + 12 + 19 = \_\_\_\_%
\newline%
\newline%
\newline%
24) 8 + 3 + 17 = \_\_\_\_%
\newline%
\newline%
\newline%
25) 2 + 14 + 15 = \_\_\_\_%
\newline%
\newline%
\newline%
26) Alex has 14 marbles. Sam has 16 marbles. Michael has 6 marbles. How many marbles do they have in all?%
\newline%
\newline%
\newline%
27) Sam has 3 books. Michael has 3 books. Alex has 15 books. How many books do they have in all?%
\newline%
\newline%
\newline%
28) Bob has 13 pens. Alex has 7 pens. James has 3 pens. How many pens do they have in all?%
\newline%
\newline%
\newline%
29) Michael has 18 pens. Sally has 5 pens. James has 7 pens. How many pens do they have in all?%
\newline%
\newline%
\newline%
30) Rachel has 18 books. Michael has 4 books. Alex has 2 books. How many books do they have in all?%
\newline%
\newline%
\newline%
\pagebreak%
\large%
\begin{center}%
\textbf{Adding Three Numbers- Worksheet 4}%
\newline%
\end{center} \normalsize%
1) 17 + 18 + 16 = \_\_\_\_%
\newline%
\newline%
\newline%
2) 10 + 9 + 14 = \_\_\_\_%
\newline%
\newline%
\newline%
3) 19 + 7 + 12 = \_\_\_\_%
\newline%
\newline%
\newline%
4) 10 + 15 + 18 = \_\_\_\_%
\newline%
\newline%
\newline%
5) 6 + 12 + 12 = \_\_\_\_%
\newline%
\newline%
\newline%
6) 6 + 4 + 12 = \_\_\_\_%
\newline%
\newline%
\newline%
7) 12 + 8 + 4 = \_\_\_\_%
\newline%
\newline%
\newline%
8) 2 + 17 + 15 = \_\_\_\_%
\newline%
\newline%
\newline%
9) 7 + 17 + 2 = \_\_\_\_%
\newline%
\newline%
\newline%
10) 20 + 4 + 15 = \_\_\_\_%
\newline%
\newline%
\newline%
11) 19 + 16 + 10 = \_\_\_\_%
\newline%
\newline%
\newline%
12) 6 + 18 + 5 = \_\_\_\_%
\newline%
\newline%
\newline%
13) 16 + 17 + 19 = \_\_\_\_%
\newline%
\newline%
\newline%
14) 14 + 20 + 5 = \_\_\_\_%
\newline%
\newline%
\newline%
15) 4 + 12 + 8 = \_\_\_\_%
\newline%
\newline%
\newline%
16) 10 + 10 + 18 = \_\_\_\_%
\newline%
\newline%
\newline%
17) 4 + 20 + 14 = \_\_\_\_%
\newline%
\newline%
\newline%
18) 2 + 11 + 11 = \_\_\_\_%
\newline%
\newline%
\newline%
19) 2 + 11 + 4 = \_\_\_\_%
\newline%
\newline%
\newline%
20) 13 + 14 + 4 = \_\_\_\_%
\newline%
\newline%
\newline%
21) 20 + 19 + 18 = \_\_\_\_%
\newline%
\newline%
\newline%
22) 2 + 13 + 12 = \_\_\_\_%
\newline%
\newline%
\newline%
23) 16 + 11 + 10 = \_\_\_\_%
\newline%
\newline%
\newline%
24) 20 + 15 + 3 = \_\_\_\_%
\newline%
\newline%
\newline%
25) 6 + 17 + 19 = \_\_\_\_%
\newline%
\newline%
\newline%
26) Rachel has 10 bottles. Laura has 8 bottles. Michael has 20 bottles. How many bottles do they have in all?%
\newline%
\newline%
\newline%
27) Sally has 16 marbles. Rachel has 19 marbles. Laura has 16 marbles. How many marbles do they have in all?%
\newline%
\newline%
\newline%
28) Michael has 9 pens. Laura has 7 pens. Sally has 19 pens. How many pens do they have in all?%
\newline%
\newline%
\newline%
29) James has 16 books. Rachel has 19 books. Sally has 3 books. How many books do they have in all?%
\newline%
\newline%
\newline%
30) James has 6 papers. Michael has 16 papers. Rachel has 17 papers. How many papers do they have in all?%
\newline%
\newline%
\newline%
\pagebreak%
\large%
\begin{center}%
\textbf{Before and After- Worksheet 1}%
\newline%
\end{center} \normalsize%
1) What number comes 6 before 28?%
\newline%
\newline%
\newline%
2) What number comes 8 after 128?%
\newline%
\newline%
\newline%
3) What number comes 6 before 97?%
\newline%
\newline%
\newline%
4) What number comes 8 after 32?%
\newline%
\newline%
\newline%
5) What number comes 3 after 62?%
\newline%
\newline%
\newline%
6) What number comes 9 after 79?%
\newline%
\newline%
\newline%
7) What number comes 5 before 23?%
\newline%
\newline%
\newline%
8) What number comes 7 before 41?%
\newline%
\newline%
\newline%
9) What number comes 6 after 107?%
\newline%
\newline%
\newline%
10) What number comes 10 before 62?%
\newline%
\newline%
\newline%
11) What number comes 9 before 140?%
\newline%
\newline%
\newline%
12) What number comes 5 after 74?%
\newline%
\newline%
\newline%
13) What number comes 7 after 101?%
\newline%
\newline%
\newline%
14) What number comes 3 before 136?%
\newline%
\newline%
\newline%
15) What number comes 10 after 86?%
\newline%
\newline%
\newline%
16) What number comes 3 after 58?%
\newline%
\newline%
\newline%
17) What number comes 1 after 110?%
\newline%
\newline%
\newline%
18) What number comes 3 before 77?%
\newline%
\newline%
\newline%
19) What number comes 10 after 140?%
\newline%
\newline%
\newline%
20) What number comes 1 after 88?%
\newline%
\newline%
\newline%
21) What number comes 9 after 57?%
\newline%
\newline%
\newline%
22) What number comes 1 before 43?%
\newline%
\newline%
\newline%
23) What number comes 7 after 48?%
\newline%
\newline%
\newline%
24) What number comes 5 after 118?%
\newline%
\newline%
\newline%
25) What number comes 8 before 100?%
\newline%
\newline%
\newline%
26) What number comes 1 after 132?%
\newline%
\newline%
\newline%
27) What number comes 8 after 53?%
\newline%
\newline%
\newline%
28) What number comes 4 after 85?%
\newline%
\newline%
\newline%
29) What number comes 10 after 125?%
\newline%
\newline%
\newline%
30) What number comes 1 before 35?%
\newline%
\newline%
\newline%
\pagebreak%
\large%
\begin{center}%
\textbf{Before and After- Worksheet 2}%
\newline%
\end{center} \normalsize%
1) What number comes 4 after 110?%
\newline%
\newline%
\newline%
2) What number comes 7 after 48?%
\newline%
\newline%
\newline%
3) What number comes 9 before 62?%
\newline%
\newline%
\newline%
4) What number comes 4 before 117?%
\newline%
\newline%
\newline%
5) What number comes 3 before 24?%
\newline%
\newline%
\newline%
6) What number comes 1 after 20?%
\newline%
\newline%
\newline%
7) What number comes 10 after 136?%
\newline%
\newline%
\newline%
8) What number comes 7 before 66?%
\newline%
\newline%
\newline%
9) What number comes 8 before 94?%
\newline%
\newline%
\newline%
10) What number comes 3 after 122?%
\newline%
\newline%
\newline%
11) What number comes 7 before 22?%
\newline%
\newline%
\newline%
12) What number comes 4 before 69?%
\newline%
\newline%
\newline%
13) What number comes 4 after 27?%
\newline%
\newline%
\newline%
14) What number comes 6 after 11?%
\newline%
\newline%
\newline%
15) What number comes 6 after 31?%
\newline%
\newline%
\newline%
16) What number comes 8 after 46?%
\newline%
\newline%
\newline%
17) What number comes 1 after 71?%
\newline%
\newline%
\newline%
18) What number comes 9 before 73?%
\newline%
\newline%
\newline%
19) What number comes 6 before 31?%
\newline%
\newline%
\newline%
20) What number comes 6 before 60?%
\newline%
\newline%
\newline%
21) What number comes 3 after 98?%
\newline%
\newline%
\newline%
22) What number comes 6 after 93?%
\newline%
\newline%
\newline%
23) What number comes 2 before 43?%
\newline%
\newline%
\newline%
24) What number comes 8 after 139?%
\newline%
\newline%
\newline%
25) What number comes 10 before 105?%
\newline%
\newline%
\newline%
26) What number comes 1 before 146?%
\newline%
\newline%
\newline%
27) What number comes 10 before 42?%
\newline%
\newline%
\newline%
28) What number comes 5 after 55?%
\newline%
\newline%
\newline%
29) What number comes 3 after 124?%
\newline%
\newline%
\newline%
30) What number comes 4 before 32?%
\newline%
\newline%
\newline%
\pagebreak%
\large%
\begin{center}%
\textbf{Before and After- Worksheet 3}%
\newline%
\end{center} \normalsize%
1) What number comes 10 after 33?%
\newline%
\newline%
\newline%
2) What number comes 7 after 73?%
\newline%
\newline%
\newline%
3) What number comes 2 after 39?%
\newline%
\newline%
\newline%
4) What number comes 1 after 64?%
\newline%
\newline%
\newline%
5) What number comes 8 before 30?%
\newline%
\newline%
\newline%
6) What number comes 10 before 82?%
\newline%
\newline%
\newline%
7) What number comes 9 before 47?%
\newline%
\newline%
\newline%
8) What number comes 5 before 35?%
\newline%
\newline%
\newline%
9) What number comes 5 before 17?%
\newline%
\newline%
\newline%
10) What number comes 1 after 135?%
\newline%
\newline%
\newline%
11) What number comes 3 after 136?%
\newline%
\newline%
\newline%
12) What number comes 2 before 68?%
\newline%
\newline%
\newline%
13) What number comes 4 before 63?%
\newline%
\newline%
\newline%
14) What number comes 5 after 68?%
\newline%
\newline%
\newline%
15) What number comes 4 after 27?%
\newline%
\newline%
\newline%
16) What number comes 2 after 135?%
\newline%
\newline%
\newline%
17) What number comes 9 after 49?%
\newline%
\newline%
\newline%
18) What number comes 5 after 14?%
\newline%
\newline%
\newline%
19) What number comes 7 before 43?%
\newline%
\newline%
\newline%
20) What number comes 3 after 19?%
\newline%
\newline%
\newline%
21) What number comes 4 after 66?%
\newline%
\newline%
\newline%
22) What number comes 7 before 127?%
\newline%
\newline%
\newline%
23) What number comes 9 before 48?%
\newline%
\newline%
\newline%
24) What number comes 9 before 108?%
\newline%
\newline%
\newline%
25) What number comes 2 before 116?%
\newline%
\newline%
\newline%
26) What number comes 4 after 32?%
\newline%
\newline%
\newline%
27) What number comes 9 after 117?%
\newline%
\newline%
\newline%
28) What number comes 6 before 13?%
\newline%
\newline%
\newline%
29) What number comes 2 before 13?%
\newline%
\newline%
\newline%
30) What number comes 2 after 62?%
\newline%
\newline%
\newline%
\pagebreak%
\large%
\begin{center}%
\textbf{Before and After- Worksheet 4}%
\newline%
\end{center} \normalsize%
1) What number comes 9 after 134?%
\newline%
\newline%
\newline%
2) What number comes 5 after 86?%
\newline%
\newline%
\newline%
3) What number comes 3 after 117?%
\newline%
\newline%
\newline%
4) What number comes 9 before 20?%
\newline%
\newline%
\newline%
5) What number comes 1 before 84?%
\newline%
\newline%
\newline%
6) What number comes 7 before 61?%
\newline%
\newline%
\newline%
7) What number comes 1 before 80?%
\newline%
\newline%
\newline%
8) What number comes 6 before 143?%
\newline%
\newline%
\newline%
9) What number comes 8 after 148?%
\newline%
\newline%
\newline%
10) What number comes 1 before 149?%
\newline%
\newline%
\newline%
11) What number comes 10 before 23?%
\newline%
\newline%
\newline%
12) What number comes 1 before 34?%
\newline%
\newline%
\newline%
13) What number comes 2 before 37?%
\newline%
\newline%
\newline%
14) What number comes 9 after 143?%
\newline%
\newline%
\newline%
15) What number comes 7 after 79?%
\newline%
\newline%
\newline%
16) What number comes 7 before 145?%
\newline%
\newline%
\newline%
17) What number comes 3 after 60?%
\newline%
\newline%
\newline%
18) What number comes 7 before 94?%
\newline%
\newline%
\newline%
19) What number comes 2 after 149?%
\newline%
\newline%
\newline%
20) What number comes 2 before 14?%
\newline%
\newline%
\newline%
21) What number comes 2 before 35?%
\newline%
\newline%
\newline%
22) What number comes 4 before 20?%
\newline%
\newline%
\newline%
23) What number comes 10 before 72?%
\newline%
\newline%
\newline%
24) What number comes 9 before 25?%
\newline%
\newline%
\newline%
25) What number comes 5 before 35?%
\newline%
\newline%
\newline%
26) What number comes 6 before 107?%
\newline%
\newline%
\newline%
27) What number comes 2 before 112?%
\newline%
\newline%
\newline%
28) What number comes 6 after 148?%
\newline%
\newline%
\newline%
29) What number comes 2 before 131?%
\newline%
\newline%
\newline%
30) What number comes 3 after 71?%
\newline%
\newline%
\newline%
\pagebreak%
\large%
\begin{center}%
\textbf{Basic Place Value- Worksheet 1}%
\newline%
\end{center} \normalsize%
1) What number is equivalent to 5 tens and 9 ones?%
\newline%
\newline%
\newline%
2) What number is equivalent to 4 tens and 7 ones?%
\newline%
\newline%
\newline%
3) What number is equivalent to 9 tens and 3 ones?%
\newline%
\newline%
\newline%
4) What number is equivalent to 8 tens and 4 ones?%
\newline%
\newline%
\newline%
5) What number is equivalent to 6 tens and 5 ones?%
\newline%
\newline%
\newline%
6) What number is equivalent to 4 tens and 9 ones?%
\newline%
\newline%
\newline%
7) What number is equivalent to 10 tens and 4 ones?%
\newline%
\newline%
\newline%
8) What number is equivalent to 9 tens and 2 ones?%
\newline%
\newline%
\newline%
9) What number is equivalent to 8 tens and 6 ones?%
\newline%
\newline%
\newline%
10) What number is equivalent to 8 tens and 8 ones?%
\newline%
\newline%
\newline%
11) What number is equivalent to 3 tens and 3 ones?%
\newline%
\newline%
\newline%
12) What number is equivalent to 9 tens and 5 ones?%
\newline%
\newline%
\newline%
13) What number is equivalent to 8 tens and 2 ones?%
\newline%
\newline%
\newline%
14) What number is equivalent to 2 tens and 5 ones?%
\newline%
\newline%
\newline%
15) What number is equivalent to 2 tens and 4 ones?%
\newline%
\newline%
\newline%
16) What number is equivalent to 10 tens and 5 ones?%
\newline%
\newline%
\newline%
17) What number is equivalent to 9 tens and 4 ones?%
\newline%
\newline%
\newline%
18) What number is equivalent to 3 tens and 4 ones?%
\newline%
\newline%
\newline%
19) What number is equivalent to 7 tens and 2 ones?%
\newline%
\newline%
\newline%
20) What number is equivalent to 9 tens and 8 ones?%
\newline%
\newline%
\newline%
21) What number is equivalent to 7 tens and 4 ones?%
\newline%
\newline%
\newline%
22) What number is equivalent to 4 tens and 6 ones?%
\newline%
\newline%
\newline%
23) What number is equivalent to 2 tens and 6 ones?%
\newline%
\newline%
\newline%
24) What number is equivalent to 10 tens and 9 ones?%
\newline%
\newline%
\newline%
25) What number is equivalent to 10 tens and 8 ones?%
\newline%
\newline%
\newline%
26) What number is equivalent to 8 tens and 9 ones?%
\newline%
\newline%
\newline%
27) What number is equivalent to 3 tens and 6 ones?%
\newline%
\newline%
\newline%
28) What number is equivalent to 4 tens and 2 ones?%
\newline%
\newline%
\newline%
29) What number is equivalent to 3 tens and 8 ones?%
\newline%
\newline%
\newline%
30) What number is equivalent to 3 tens and 5 ones?%
\newline%
\newline%
\newline%
\pagebreak%
\large%
\begin{center}%
\textbf{Basic Place Value- Worksheet 2}%
\newline%
\end{center} \normalsize%
1) What number is equivalent to 3 tens and 6 ones?%
\newline%
\newline%
\newline%
2) What number is equivalent to 5 tens and 9 ones?%
\newline%
\newline%
\newline%
3) What number is equivalent to 7 tens and 4 ones?%
\newline%
\newline%
\newline%
4) What number is equivalent to 7 tens and 9 ones?%
\newline%
\newline%
\newline%
5) What number is equivalent to 10 tens and 9 ones?%
\newline%
\newline%
\newline%
6) What number is equivalent to 10 tens and 5 ones?%
\newline%
\newline%
\newline%
7) What number is equivalent to 5 tens and 4 ones?%
\newline%
\newline%
\newline%
8) What number is equivalent to 9 tens and 5 ones?%
\newline%
\newline%
\newline%
9) What number is equivalent to 10 tens and 6 ones?%
\newline%
\newline%
\newline%
10) What number is equivalent to 8 tens and 9 ones?%
\newline%
\newline%
\newline%
11) What number is equivalent to 2 tens and 9 ones?%
\newline%
\newline%
\newline%
12) What number is equivalent to 3 tens and 3 ones?%
\newline%
\newline%
\newline%
13) What number is equivalent to 8 tens and 2 ones?%
\newline%
\newline%
\newline%
14) What number is equivalent to 4 tens and 7 ones?%
\newline%
\newline%
\newline%
15) What number is equivalent to 9 tens and 3 ones?%
\newline%
\newline%
\newline%
16) What number is equivalent to 7 tens and 3 ones?%
\newline%
\newline%
\newline%
17) What number is equivalent to 7 tens and 5 ones?%
\newline%
\newline%
\newline%
18) What number is equivalent to 6 tens and 9 ones?%
\newline%
\newline%
\newline%
19) What number is equivalent to 8 tens and 4 ones?%
\newline%
\newline%
\newline%
20) What number is equivalent to 7 tens and 6 ones?%
\newline%
\newline%
\newline%
21) What number is equivalent to 4 tens and 9 ones?%
\newline%
\newline%
\newline%
22) What number is equivalent to 4 tens and 3 ones?%
\newline%
\newline%
\newline%
23) What number is equivalent to 6 tens and 7 ones?%
\newline%
\newline%
\newline%
24) What number is equivalent to 3 tens and 5 ones?%
\newline%
\newline%
\newline%
25) What number is equivalent to 8 tens and 7 ones?%
\newline%
\newline%
\newline%
26) What number is equivalent to 5 tens and 7 ones?%
\newline%
\newline%
\newline%
27) What number is equivalent to 2 tens and 8 ones?%
\newline%
\newline%
\newline%
28) What number is equivalent to 5 tens and 3 ones?%
\newline%
\newline%
\newline%
29) What number is equivalent to 2 tens and 2 ones?%
\newline%
\newline%
\newline%
30) What number is equivalent to 2 tens and 4 ones?%
\newline%
\newline%
\newline%
\pagebreak%
\large%
\begin{center}%
\textbf{Basic Place Value- Worksheet 3}%
\newline%
\end{center} \normalsize%
1) What number is equivalent to 10 tens and 7 ones?%
\newline%
\newline%
\newline%
2) What number is equivalent to 10 tens and 3 ones?%
\newline%
\newline%
\newline%
3) What number is equivalent to 7 tens and 7 ones?%
\newline%
\newline%
\newline%
4) What number is equivalent to 2 tens and 2 ones?%
\newline%
\newline%
\newline%
5) What number is equivalent to 4 tens and 9 ones?%
\newline%
\newline%
\newline%
6) What number is equivalent to 9 tens and 3 ones?%
\newline%
\newline%
\newline%
7) What number is equivalent to 4 tens and 8 ones?%
\newline%
\newline%
\newline%
8) What number is equivalent to 8 tens and 3 ones?%
\newline%
\newline%
\newline%
9) What number is equivalent to 7 tens and 8 ones?%
\newline%
\newline%
\newline%
10) What number is equivalent to 2 tens and 6 ones?%
\newline%
\newline%
\newline%
11) What number is equivalent to 8 tens and 9 ones?%
\newline%
\newline%
\newline%
12) What number is equivalent to 9 tens and 4 ones?%
\newline%
\newline%
\newline%
13) What number is equivalent to 2 tens and 7 ones?%
\newline%
\newline%
\newline%
14) What number is equivalent to 5 tens and 7 ones?%
\newline%
\newline%
\newline%
15) What number is equivalent to 6 tens and 7 ones?%
\newline%
\newline%
\newline%
16) What number is equivalent to 5 tens and 2 ones?%
\newline%
\newline%
\newline%
17) What number is equivalent to 7 tens and 9 ones?%
\newline%
\newline%
\newline%
18) What number is equivalent to 2 tens and 4 ones?%
\newline%
\newline%
\newline%
19) What number is equivalent to 9 tens and 8 ones?%
\newline%
\newline%
\newline%
20) What number is equivalent to 2 tens and 3 ones?%
\newline%
\newline%
\newline%
21) What number is equivalent to 3 tens and 2 ones?%
\newline%
\newline%
\newline%
22) What number is equivalent to 6 tens and 8 ones?%
\newline%
\newline%
\newline%
23) What number is equivalent to 6 tens and 2 ones?%
\newline%
\newline%
\newline%
24) What number is equivalent to 3 tens and 5 ones?%
\newline%
\newline%
\newline%
25) What number is equivalent to 10 tens and 9 ones?%
\newline%
\newline%
\newline%
26) What number is equivalent to 8 tens and 6 ones?%
\newline%
\newline%
\newline%
27) What number is equivalent to 4 tens and 7 ones?%
\newline%
\newline%
\newline%
28) What number is equivalent to 8 tens and 2 ones?%
\newline%
\newline%
\newline%
29) What number is equivalent to 5 tens and 5 ones?%
\newline%
\newline%
\newline%
30) What number is equivalent to 6 tens and 4 ones?%
\newline%
\newline%
\newline%
\pagebreak%
\large%
\begin{center}%
\textbf{Basic Place Value- Worksheet 4}%
\newline%
\end{center} \normalsize%
1) What number is equivalent to 10 tens and 6 ones?%
\newline%
\newline%
\newline%
2) What number is equivalent to 6 tens and 4 ones?%
\newline%
\newline%
\newline%
3) What number is equivalent to 3 tens and 3 ones?%
\newline%
\newline%
\newline%
4) What number is equivalent to 3 tens and 8 ones?%
\newline%
\newline%
\newline%
5) What number is equivalent to 4 tens and 9 ones?%
\newline%
\newline%
\newline%
6) What number is equivalent to 6 tens and 5 ones?%
\newline%
\newline%
\newline%
7) What number is equivalent to 7 tens and 6 ones?%
\newline%
\newline%
\newline%
8) What number is equivalent to 7 tens and 8 ones?%
\newline%
\newline%
\newline%
9) What number is equivalent to 10 tens and 9 ones?%
\newline%
\newline%
\newline%
10) What number is equivalent to 2 tens and 9 ones?%
\newline%
\newline%
\newline%
11) What number is equivalent to 6 tens and 8 ones?%
\newline%
\newline%
\newline%
12) What number is equivalent to 3 tens and 6 ones?%
\newline%
\newline%
\newline%
13) What number is equivalent to 7 tens and 3 ones?%
\newline%
\newline%
\newline%
14) What number is equivalent to 2 tens and 4 ones?%
\newline%
\newline%
\newline%
15) What number is equivalent to 9 tens and 8 ones?%
\newline%
\newline%
\newline%
16) What number is equivalent to 10 tens and 8 ones?%
\newline%
\newline%
\newline%
17) What number is equivalent to 8 tens and 4 ones?%
\newline%
\newline%
\newline%
18) What number is equivalent to 8 tens and 9 ones?%
\newline%
\newline%
\newline%
19) What number is equivalent to 7 tens and 7 ones?%
\newline%
\newline%
\newline%
20) What number is equivalent to 6 tens and 6 ones?%
\newline%
\newline%
\newline%
21) What number is equivalent to 5 tens and 4 ones?%
\newline%
\newline%
\newline%
22) What number is equivalent to 8 tens and 8 ones?%
\newline%
\newline%
\newline%
23) What number is equivalent to 3 tens and 2 ones?%
\newline%
\newline%
\newline%
24) What number is equivalent to 5 tens and 5 ones?%
\newline%
\newline%
\newline%
25) What number is equivalent to 2 tens and 6 ones?%
\newline%
\newline%
\newline%
26) What number is equivalent to 2 tens and 2 ones?%
\newline%
\newline%
\newline%
27) What number is equivalent to 9 tens and 4 ones?%
\newline%
\newline%
\newline%
28) What number is equivalent to 10 tens and 5 ones?%
\newline%
\newline%
\newline%
29) What number is equivalent to 4 tens and 6 ones?%
\newline%
\newline%
\newline%
30) What number is equivalent to 5 tens and 2 ones?%
\newline%
\newline%
\newline%
\pagebreak%
\large%
\begin{center}%
\textbf{Breaking Apart Numbers- Worksheet 1}%
\newline%
\end{center} \normalsize%
1) Break apart the solution of 24 + 3 into tens and ones%
\newline%
\newline%
\newline%
2) Break apart the solution of 35 + 3 into tens and ones%
\newline%
\newline%
\newline%
3) Break apart the solution of 42 + 30 into tens and ones%
\newline%
\newline%
\newline%
4) Break apart the solution of 20 + 23 into tens and ones%
\newline%
\newline%
\newline%
5) Break apart the solution of 21 + 23 into tens and ones%
\newline%
\newline%
\newline%
6) Break apart the solution of 7 + 3 into tens and ones%
\newline%
\newline%
\newline%
7) Break apart the solution of 6 + 38 into tens and ones%
\newline%
\newline%
\newline%
8) Break apart the solution of 26 + 30 into tens and ones%
\newline%
\newline%
\newline%
9) Break apart the solution of 12 + 16 into tens and ones%
\newline%
\newline%
\newline%
10) Break apart the solution of 39 + 41 into tens and ones%
\newline%
\newline%
\newline%
11) Break apart the solution of 4 + 16 into tens and ones%
\newline%
\newline%
\newline%
12) Break apart the solution of 44 + 38 into tens and ones%
\newline%
\newline%
\newline%
13) Break apart the solution of 7 + 33 into tens and ones%
\newline%
\newline%
\newline%
14) Break apart the solution of 43 + 10 into tens and ones%
\newline%
\newline%
\newline%
15) Break apart the solution of 37 + 35 into tens and ones%
\newline%
\newline%
\newline%
16) Break apart the solution of 45 + 20 into tens and ones%
\newline%
\newline%
\newline%
17) Break apart the solution of 23 + 4 into tens and ones%
\newline%
\newline%
\newline%
18) Break apart the solution of 2 + 38 into tens and ones%
\newline%
\newline%
\newline%
19) Break apart the solution of 42 + 37 into tens and ones%
\newline%
\newline%
\newline%
20) Break apart the solution of 33 + 10 into tens and ones%
\newline%
\newline%
\newline%
21) Break apart the solution of 17 + 44 into tens and ones%
\newline%
\newline%
\newline%
22) Break apart the solution of 40 + 36 into tens and ones%
\newline%
\newline%
\newline%
23) Break apart the solution of 12 + 38 into tens and ones%
\newline%
\newline%
\newline%
24) Break apart the solution of 42 + 1 into tens and ones%
\newline%
\newline%
\newline%
25) Break apart the solution of 8 + 1 into tens and ones%
\newline%
\newline%
\newline%
26) Break apart the solution of 41 + 14 into tens and ones%
\newline%
\newline%
\newline%
27) Break apart the solution of 31 + 11 into tens and ones%
\newline%
\newline%
\newline%
28) Break apart the solution of 45 + 21 into tens and ones%
\newline%
\newline%
\newline%
29) Break apart the solution of 10 + 17 into tens and ones%
\newline%
\newline%
\newline%
30) Break apart the solution of 27 + 25 into tens and ones%
\newline%
\newline%
\newline%
\pagebreak%
\large%
\begin{center}%
\textbf{Breaking Apart Numbers- Worksheet 2}%
\newline%
\end{center} \normalsize%
1) Break apart the solution of 30 + 6 into tens and ones%
\newline%
\newline%
\newline%
2) Break apart the solution of 11 + 41 into tens and ones%
\newline%
\newline%
\newline%
3) Break apart the solution of 38 + 27 into tens and ones%
\newline%
\newline%
\newline%
4) Break apart the solution of 21 + 22 into tens and ones%
\newline%
\newline%
\newline%
5) Break apart the solution of 14 + 29 into tens and ones%
\newline%
\newline%
\newline%
6) Break apart the solution of 26 + 26 into tens and ones%
\newline%
\newline%
\newline%
7) Break apart the solution of 26 + 22 into tens and ones%
\newline%
\newline%
\newline%
8) Break apart the solution of 30 + 35 into tens and ones%
\newline%
\newline%
\newline%
9) Break apart the solution of 4 + 19 into tens and ones%
\newline%
\newline%
\newline%
10) Break apart the solution of 21 + 11 into tens and ones%
\newline%
\newline%
\newline%
11) Break apart the solution of 39 + 7 into tens and ones%
\newline%
\newline%
\newline%
12) Break apart the solution of 17 + 3 into tens and ones%
\newline%
\newline%
\newline%
13) Break apart the solution of 10 + 9 into tens and ones%
\newline%
\newline%
\newline%
14) Break apart the solution of 30 + 14 into tens and ones%
\newline%
\newline%
\newline%
15) Break apart the solution of 33 + 35 into tens and ones%
\newline%
\newline%
\newline%
16) Break apart the solution of 14 + 4 into tens and ones%
\newline%
\newline%
\newline%
17) Break apart the solution of 19 + 31 into tens and ones%
\newline%
\newline%
\newline%
18) Break apart the solution of 28 + 34 into tens and ones%
\newline%
\newline%
\newline%
19) Break apart the solution of 21 + 26 into tens and ones%
\newline%
\newline%
\newline%
20) Break apart the solution of 7 + 30 into tens and ones%
\newline%
\newline%
\newline%
21) Break apart the solution of 23 + 19 into tens and ones%
\newline%
\newline%
\newline%
22) Break apart the solution of 8 + 6 into tens and ones%
\newline%
\newline%
\newline%
23) Break apart the solution of 28 + 19 into tens and ones%
\newline%
\newline%
\newline%
24) Break apart the solution of 19 + 39 into tens and ones%
\newline%
\newline%
\newline%
25) Break apart the solution of 15 + 31 into tens and ones%
\newline%
\newline%
\newline%
26) Break apart the solution of 9 + 26 into tens and ones%
\newline%
\newline%
\newline%
27) Break apart the solution of 30 + 27 into tens and ones%
\newline%
\newline%
\newline%
28) Break apart the solution of 5 + 18 into tens and ones%
\newline%
\newline%
\newline%
29) Break apart the solution of 36 + 8 into tens and ones%
\newline%
\newline%
\newline%
30) Break apart the solution of 7 + 7 into tens and ones%
\newline%
\newline%
\newline%
\pagebreak%
\large%
\begin{center}%
\textbf{Breaking Apart Numbers- Worksheet 3}%
\newline%
\end{center} \normalsize%
1) Break apart the solution of 12 + 19 into tens and ones%
\newline%
\newline%
\newline%
2) Break apart the solution of 22 + 11 into tens and ones%
\newline%
\newline%
\newline%
3) Break apart the solution of 7 + 1 into tens and ones%
\newline%
\newline%
\newline%
4) Break apart the solution of 41 + 8 into tens and ones%
\newline%
\newline%
\newline%
5) Break apart the solution of 15 + 44 into tens and ones%
\newline%
\newline%
\newline%
6) Break apart the solution of 18 + 25 into tens and ones%
\newline%
\newline%
\newline%
7) Break apart the solution of 16 + 5 into tens and ones%
\newline%
\newline%
\newline%
8) Break apart the solution of 6 + 24 into tens and ones%
\newline%
\newline%
\newline%
9) Break apart the solution of 19 + 21 into tens and ones%
\newline%
\newline%
\newline%
10) Break apart the solution of 20 + 16 into tens and ones%
\newline%
\newline%
\newline%
11) Break apart the solution of 10 + 26 into tens and ones%
\newline%
\newline%
\newline%
12) Break apart the solution of 8 + 14 into tens and ones%
\newline%
\newline%
\newline%
13) Break apart the solution of 45 + 26 into tens and ones%
\newline%
\newline%
\newline%
14) Break apart the solution of 28 + 44 into tens and ones%
\newline%
\newline%
\newline%
15) Break apart the solution of 44 + 9 into tens and ones%
\newline%
\newline%
\newline%
16) Break apart the solution of 40 + 34 into tens and ones%
\newline%
\newline%
\newline%
17) Break apart the solution of 32 + 24 into tens and ones%
\newline%
\newline%
\newline%
18) Break apart the solution of 38 + 19 into tens and ones%
\newline%
\newline%
\newline%
19) Break apart the solution of 26 + 24 into tens and ones%
\newline%
\newline%
\newline%
20) Break apart the solution of 15 + 13 into tens and ones%
\newline%
\newline%
\newline%
21) Break apart the solution of 2 + 35 into tens and ones%
\newline%
\newline%
\newline%
22) Break apart the solution of 3 + 18 into tens and ones%
\newline%
\newline%
\newline%
23) Break apart the solution of 15 + 43 into tens and ones%
\newline%
\newline%
\newline%
24) Break apart the solution of 41 + 4 into tens and ones%
\newline%
\newline%
\newline%
25) Break apart the solution of 35 + 12 into tens and ones%
\newline%
\newline%
\newline%
26) Break apart the solution of 33 + 7 into tens and ones%
\newline%
\newline%
\newline%
27) Break apart the solution of 45 + 21 into tens and ones%
\newline%
\newline%
\newline%
28) Break apart the solution of 8 + 9 into tens and ones%
\newline%
\newline%
\newline%
29) Break apart the solution of 12 + 4 into tens and ones%
\newline%
\newline%
\newline%
30) Break apart the solution of 26 + 43 into tens and ones%
\newline%
\newline%
\newline%
\pagebreak%
\large%
\begin{center}%
\textbf{Breaking Apart Numbers- Worksheet 4}%
\newline%
\end{center} \normalsize%
1) Break apart the solution of 9 + 8 into tens and ones%
\newline%
\newline%
\newline%
2) Break apart the solution of 45 + 32 into tens and ones%
\newline%
\newline%
\newline%
3) Break apart the solution of 28 + 24 into tens and ones%
\newline%
\newline%
\newline%
4) Break apart the solution of 33 + 30 into tens and ones%
\newline%
\newline%
\newline%
5) Break apart the solution of 12 + 31 into tens and ones%
\newline%
\newline%
\newline%
6) Break apart the solution of 23 + 35 into tens and ones%
\newline%
\newline%
\newline%
7) Break apart the solution of 25 + 8 into tens and ones%
\newline%
\newline%
\newline%
8) Break apart the solution of 31 + 40 into tens and ones%
\newline%
\newline%
\newline%
9) Break apart the solution of 9 + 2 into tens and ones%
\newline%
\newline%
\newline%
10) Break apart the solution of 10 + 38 into tens and ones%
\newline%
\newline%
\newline%
11) Break apart the solution of 29 + 44 into tens and ones%
\newline%
\newline%
\newline%
12) Break apart the solution of 33 + 14 into tens and ones%
\newline%
\newline%
\newline%
13) Break apart the solution of 10 + 6 into tens and ones%
\newline%
\newline%
\newline%
14) Break apart the solution of 30 + 38 into tens and ones%
\newline%
\newline%
\newline%
15) Break apart the solution of 5 + 20 into tens and ones%
\newline%
\newline%
\newline%
16) Break apart the solution of 26 + 13 into tens and ones%
\newline%
\newline%
\newline%
17) Break apart the solution of 12 + 10 into tens and ones%
\newline%
\newline%
\newline%
18) Break apart the solution of 45 + 10 into tens and ones%
\newline%
\newline%
\newline%
19) Break apart the solution of 4 + 9 into tens and ones%
\newline%
\newline%
\newline%
20) Break apart the solution of 26 + 8 into tens and ones%
\newline%
\newline%
\newline%
21) Break apart the solution of 20 + 41 into tens and ones%
\newline%
\newline%
\newline%
22) Break apart the solution of 18 + 8 into tens and ones%
\newline%
\newline%
\newline%
23) Break apart the solution of 35 + 9 into tens and ones%
\newline%
\newline%
\newline%
24) Break apart the solution of 28 + 31 into tens and ones%
\newline%
\newline%
\newline%
25) Break apart the solution of 21 + 9 into tens and ones%
\newline%
\newline%
\newline%
26) Break apart the solution of 23 + 17 into tens and ones%
\newline%
\newline%
\newline%
27) Break apart the solution of 3 + 20 into tens and ones%
\newline%
\newline%
\newline%
28) Break apart the solution of 18 + 38 into tens and ones%
\newline%
\newline%
\newline%
29) Break apart the solution of 10 + 20 into tens and ones%
\newline%
\newline%
\newline%
30) Break apart the solution of 16 + 41 into tens and ones%
\newline%
\newline%
\newline%
\pagebreak%
\huge%
\vspace*{\fill}%
\begin{center}%
Solutions%
\end{center}%
\vspace*{\fill}%
\normalsize%
\pagebreak%
\large%
\begin{center}%
\textbf{Comparing Two Digit Numbers- Solution 1}%
\newline%
\end{center} \normalsize%
1) 76 > 25%
\newline%
2) 28 < 71%
\newline%
3) 71 < 92%
\newline%
4) 70 > 64%
\newline%
5) 69 > 15%
\newline%
6) 1 < 6%
\newline%
7) 58 > 12%
\newline%
8) 70 > 20%
\newline%
9) 32 > 22%
\newline%
10) 39 > 16%
\newline%
11) 59 < 82%
\newline%
12) 62 > 60%
\newline%
13) 11 < 39%
\newline%
14) 16 < 39%
\newline%
15) 72 < 88%
\newline%
16) 75 > 31%
\newline%
17) 60 < 82%
\newline%
18) 89 > 11%
\newline%
19) 50 > 24%
\newline%
20) 84 > 61%
\newline%
21) 19 < 31%
\newline%
22) 23 < 87%
\newline%
23) 68 > 51%
\newline%
24) 31 < 56%
\newline%
25) 15 < 77%
\newline%
26) Michael has more pencils.%
\newline%
27) Rachel has more toys.%
\newline%
28) Michael has more pens.%
\newline%
29) Sam has more pencils.%
\newline%
30) Laura has more dollars.%
\newline%
\newpage%
\large%
\begin{center}%
\textbf{Comparing Two Digit Numbers- Solution 2}%
\newline%
\end{center} \normalsize%
1) 17 < 97%
\newline%
2) 7 < 44%
\newline%
3) 92 > 84%
\newline%
4) 99 > 3%
\newline%
5) 88 > 44%
\newline%
6) 94 > 69%
\newline%
7) 7 < 87%
\newline%
8) 53 < 85%
\newline%
9) 26 < 45%
\newline%
10) 20 < 32%
\newline%
11) 76 > 66%
\newline%
12) 85 > 5%
\newline%
13) 67 < 70%
\newline%
14) 5 < 28%
\newline%
15) 36 < 90%
\newline%
16) 63 > 19%
\newline%
17) 98 > 10%
\newline%
18) 20 < 42%
\newline%
19) 47 < 97%
\newline%
20) 72 < 82%
\newline%
21) 57 < 63%
\newline%
22) 12 < 38%
\newline%
23) 45 < 53%
\newline%
24) 78 > 27%
\newline%
25) 91 > 43%
\newline%
26) Sam has more papers.%
\newline%
27) Michael has more papers.%
\newline%
28) Rachel has more pencils.%
\newline%
29) Sally has more papers.%
\newline%
30) Laura has more bottles.%
\newline%
\newpage%
\large%
\begin{center}%
\textbf{Comparing Two Digit Numbers- Solution 3}%
\newline%
\end{center} \normalsize%
1) 52 < 53%
\newline%
2) 65 > 12%
\newline%
3) 23 < 33%
\newline%
4) 57 > 44%
\newline%
5) 43 < 62%
\newline%
6) 42 < 46%
\newline%
7) 34 < 63%
\newline%
8) 95 > 56%
\newline%
9) 58 < 74%
\newline%
10) 18 < 19%
\newline%
11) 70 > 57%
\newline%
12) 31 > 28%
\newline%
13) 78 > 21%
\newline%
14) 33 < 59%
\newline%
15) 66 > 25%
\newline%
16) 88 < 93%
\newline%
17) 87 > 33%
\newline%
18) 38 < 80%
\newline%
19) 94 > 16%
\newline%
20) 99 > 63%
\newline%
21) 74 > 34%
\newline%
22) 25 < 87%
\newline%
23) 41 > 1%
\newline%
24) 82 > 16%
\newline%
25) 16 < 86%
\newline%
26) They have the same amount of marbles.%
\newline%
27) James has more toys.%
\newline%
28) Sam has more bottles.%
\newline%
29) Bob has more pencils.%
\newline%
30) Alex has more toys.%
\newline%
\newpage%
\large%
\begin{center}%
\textbf{Comparing Two Digit Numbers- Solution 4}%
\newline%
\end{center} \normalsize%
1) 34 > 12%
\newline%
2) 35 > 22%
\newline%
3) 43 > 25%
\newline%
4) 49 < 91%
\newline%
5) 16 < 91%
\newline%
6) 28 < 97%
\newline%
7) 49 > 36%
\newline%
8) 92 < 94%
\newline%
9) 60 < 79%
\newline%
10) 48 < 83%
\newline%
11) 86 < 93%
\newline%
12) 69 < 80%
\newline%
13) 39 > 9%
\newline%
14) 9 < 39%
\newline%
15) 53 < 58%
\newline%
16) 56 > 24%
\newline%
17) 97 > 50%
\newline%
18) 86 > 64%
\newline%
19) 1 < 87%
\newline%
20) 28 < 44%
\newline%
21) 44 < 83%
\newline%
22) 57 > 42%
\newline%
23) 51 < 76%
\newline%
24) 26 < 62%
\newline%
25) 14 < 33%
\newline%
26) Sam has more dollars.%
\newline%
27) Bob has more toys.%
\newline%
28) Alex has more marbles.%
\newline%
29) Sally has more books.%
\newline%
30) Sam has more books.%
\newline%
\newpage%
\large%
\begin{center}%
\textbf{Adding Numbers Within 20- Solution 1}%
\newline%
\end{center} \normalsize%
1) 11 + 11 = 22%
\newline%
2) 19 + 19 = 38%
\newline%
3) 59 + 64 = 123%
\newline%
4) 20 + 20 = 40%
\newline%
5) 4 + 4 = 8%
\newline%
6) 12 + 29 = 41%
\newline%
7) 18 + 18 = 36%
\newline%
8) 57 + 64 = 121%
\newline%
9) 8 + 8 = 16%
\newline%
10) 6 + 6 = 12%
\newline%
11) 16 + 16 = 32%
\newline%
12) 28 + 48 = 76%
\newline%
13) 33 + 44 = 77%
\newline%
14) 31 + 34 = 65%
\newline%
15) 7 + 7 = 14%
\newline%
16) 3 + 6 = 9%
\newline%
17) 6 + 10 = 16%
\newline%
18) 14 + 20 = 34%
\newline%
19) 37 + 56 = 93%
\newline%
20) 26 + 43 = 69%
\newline%
21) 17 + 18 = 35%
\newline%
22) 13 + 13 = 26%
\newline%
23) 51 + 52 = 103%
\newline%
24) 9 + 19 = 28%
\newline%
25) 22 + 23 = 45%
\newline%
26) Laura has 34 dollars.%
\newline%
27) Sally has 106 pens.%
\newline%
28) Sally has 122 papers.%
\newline%
29) Alex has 132 dollars.%
\newline%
30) James has 23 pencils.%
\newline%
\newpage%
\large%
\begin{center}%
\textbf{Adding Numbers Within 20- Solution 2}%
\newline%
\end{center} \normalsize%
1) 32 + 35 = 67%
\newline%
2) 11 + 17 = 28%
\newline%
3) 19 + 19 = 38%
\newline%
4) 38 + 57 = 95%
\newline%
5) 17 + 17 = 34%
\newline%
6) 11 + 11 = 22%
\newline%
7) 7 + 26 = 33%
\newline%
8) 28 + 41 = 69%
\newline%
9) 4 + 8 = 12%
\newline%
10) 48 + 64 = 112%
\newline%
11) 5 + 5 = 10%
\newline%
12) 32 + 42 = 74%
\newline%
13) 14 + 28 = 42%
\newline%
14) 15 + 19 = 34%
\newline%
15) 69 + 80 = 149%
\newline%
16) 18 + 34 = 52%
\newline%
17) 4 + 21 = 25%
\newline%
18) 56 + 75 = 131%
\newline%
19) 6 + 6 = 12%
\newline%
20) 2 + 2 = 4%
\newline%
21) 56 + 66 = 122%
\newline%
22) 53 + 65 = 118%
\newline%
23) 45 + 65 = 110%
\newline%
24) 20 + 20 = 40%
\newline%
25) 51 + 68 = 119%
\newline%
26) Michael has 20 books.%
\newline%
27) Sally has 36 papers.%
\newline%
28) Bob has 122 toys.%
\newline%
29) Sally has 2 pencils.%
\newline%
30) Michael has 75 toys.%
\newline%
\newpage%
\large%
\begin{center}%
\textbf{Adding Numbers Within 20- Solution 3}%
\newline%
\end{center} \normalsize%
1) 39 + 52 = 91%
\newline%
2) 14 + 14 = 28%
\newline%
3) 12 + 30 = 42%
\newline%
4) 2 + 17 = 19%
\newline%
5) 46 + 54 = 100%
\newline%
6) 44 + 61 = 105%
\newline%
7) 20 + 20 = 40%
\newline%
8) 36 + 48 = 84%
\newline%
9) 16 + 16 = 32%
\newline%
10) 52 + 71 = 123%
\newline%
11) 2 + 2 = 4%
\newline%
12) 44 + 56 = 100%
\newline%
13) 1 + 1 = 2%
\newline%
14) 19 + 19 = 38%
\newline%
15) 2 + 8 = 10%
\newline%
16) 43 + 53 = 96%
\newline%
17) 7 + 7 = 14%
\newline%
18) 58 + 60 = 118%
\newline%
19) 9 + 9 = 18%
\newline%
20) 69 + 73 = 142%
\newline%
21) 12 + 22 = 34%
\newline%
22) 6 + 21 = 27%
\newline%
23) 40 + 47 = 87%
\newline%
24) 8 + 8 = 16%
\newline%
25) 4 + 4 = 8%
\newline%
26) Alex has 32 pencils.%
\newline%
27) Michael has 111 dollars.%
\newline%
28) Alex has 26 marbles.%
\newline%
29) Rachel has 36 pens.%
\newline%
30) Laura has 12 dollars.%
\newline%
\newpage%
\large%
\begin{center}%
\textbf{Adding Numbers Within 20- Solution 4}%
\newline%
\end{center} \normalsize%
1) 44 + 45 = 89%
\newline%
2) 22 + 39 = 61%
\newline%
3) 70 + 88 = 158%
\newline%
4) 12 + 12 = 24%
\newline%
5) 28 + 44 = 72%
\newline%
6) 1 + 1 = 2%
\newline%
7) 61 + 71 = 132%
\newline%
8) 9 + 9 = 18%
\newline%
9) 10 + 13 = 23%
\newline%
10) 20 + 20 = 40%
\newline%
11) 45 + 55 = 100%
\newline%
12) 5 + 5 = 10%
\newline%
13) 67 + 74 = 141%
\newline%
14) 15 + 15 = 30%
\newline%
15) 47 + 51 = 98%
\newline%
16) 8 + 8 = 16%
\newline%
17) 2 + 2 = 4%
\newline%
18) 19 + 35 = 54%
\newline%
19) 68 + 71 = 139%
\newline%
20) 5 + 8 = 13%
\newline%
21) 10 + 14 = 24%
\newline%
22) 12 + 13 = 25%
\newline%
23) 10 + 10 = 20%
\newline%
24) 45 + 51 = 96%
\newline%
25) 19 + 25 = 44%
\newline%
26) Bob has 28 papers.%
\newline%
27) Laura has 73 pencils.%
\newline%
28) Michael has 12 pens.%
\newline%
29) Bob has 14 pens.%
\newline%
30) Michael has 38 books.%
\newline%
\newpage%
\large%
\begin{center}%
\textbf{Subtracting Numbers Within 20- Solution 1}%
\newline%
\end{center} \normalsize%
1) 22 {-} 5 = 17%
\newline%
2) 22 {-} 17 = 5%
\newline%
3) 67 {-} 4 = 63%
\newline%
4) 28 {-} 17 = 11%
\newline%
5) 23 {-} 8 = 15%
\newline%
6) 32 {-} 3 = 29%
\newline%
7) 81 {-} 10 = 71%
\newline%
8) 62 {-} 3 = 59%
\newline%
9) 21 {-} 7 = 14%
\newline%
10) 28 {-} 14 = 14%
\newline%
11) 31 {-} 5 = 26%
\newline%
12) 80 {-} 2 = 78%
\newline%
13) 54 {-} 17 = 37%
\newline%
14) 52 {-} 18 = 34%
\newline%
15) 63 {-} 18 = 45%
\newline%
16) 23 {-} 2 = 21%
\newline%
17) 74 {-} 19 = 55%
\newline%
18) 86 {-} 11 = 75%
\newline%
19) 70 {-} 14 = 56%
\newline%
20) 41 {-} 18 = 23%
\newline%
21) 93 {-} 1 = 92%
\newline%
22) 66 {-} 5 = 61%
\newline%
23) 55 {-} 3 = 52%
\newline%
24) 90 {-} 18 = 72%
\newline%
25) 27 {-} 13 = 14%
\newline%
26) Michael has 65 pens.%
\newline%
27) Alex has 54 papers.%
\newline%
28) Bob has 65 dollars.%
\newline%
29) Sally has 57 pens.%
\newline%
30) Sam has 32 marbles.%
\newline%
\newpage%
\large%
\begin{center}%
\textbf{Subtracting Numbers Within 20- Solution 2}%
\newline%
\end{center} \normalsize%
1) 61 {-} 19 = 42%
\newline%
2) 84 {-} 13 = 71%
\newline%
3) 62 {-} 5 = 57%
\newline%
4) 96 {-} 15 = 81%
\newline%
5) 25 {-} 10 = 15%
\newline%
6) 82 {-} 18 = 64%
\newline%
7) 32 {-} 17 = 15%
\newline%
8) 64 {-} 7 = 57%
\newline%
9) 24 {-} 13 = 11%
\newline%
10) 40 {-} 18 = 22%
\newline%
11) 97 {-} 1 = 96%
\newline%
12) 28 {-} 2 = 26%
\newline%
13) 81 {-} 10 = 71%
\newline%
14) 88 {-} 8 = 80%
\newline%
15) 71 {-} 6 = 65%
\newline%
16) 64 {-} 20 = 44%
\newline%
17) 97 {-} 3 = 94%
\newline%
18) 21 {-} 9 = 12%
\newline%
19) 21 {-} 13 = 8%
\newline%
20) 26 {-} 4 = 22%
\newline%
21) 49 {-} 9 = 40%
\newline%
22) 67 {-} 9 = 58%
\newline%
23) 46 {-} 11 = 35%
\newline%
24) 39 {-} 20 = 19%
\newline%
25) 47 {-} 8 = 39%
\newline%
26) Rachel has 16 toys.%
\newline%
27) Alex has 90 dollars.%
\newline%
28) James has 44 pens.%
\newline%
29) Sally has 43 toys.%
\newline%
30) Sam has 29 marbles.%
\newline%
\newpage%
\large%
\begin{center}%
\textbf{Subtracting Numbers Within 20- Solution 3}%
\newline%
\end{center} \normalsize%
1) 37 {-} 1 = 36%
\newline%
2) 27 {-} 7 = 20%
\newline%
3) 83 {-} 1 = 82%
\newline%
4) 46 {-} 5 = 41%
\newline%
5) 50 {-} 9 = 41%
\newline%
6) 90 {-} 2 = 88%
\newline%
7) 93 {-} 17 = 76%
\newline%
8) 42 {-} 9 = 33%
\newline%
9) 56 {-} 3 = 53%
\newline%
10) 31 {-} 15 = 16%
\newline%
11) 39 {-} 6 = 33%
\newline%
12) 50 {-} 11 = 39%
\newline%
13) 93 {-} 18 = 75%
\newline%
14) 28 {-} 6 = 22%
\newline%
15) 51 {-} 4 = 47%
\newline%
16) 89 {-} 10 = 79%
\newline%
17) 36 {-} 10 = 26%
\newline%
18) 87 {-} 3 = 84%
\newline%
19) 71 {-} 18 = 53%
\newline%
20) 87 {-} 15 = 72%
\newline%
21) 50 {-} 20 = 30%
\newline%
22) 69 {-} 1 = 68%
\newline%
23) 32 {-} 16 = 16%
\newline%
24) 66 {-} 11 = 55%
\newline%
25) 88 {-} 4 = 84%
\newline%
26) Laura has 63 pencils.%
\newline%
27) Sam has 32 papers.%
\newline%
28) James has 38 pens.%
\newline%
29) Rachel has 14 papers.%
\newline%
30) James has 17 pens.%
\newline%
\newpage%
\large%
\begin{center}%
\textbf{Subtracting Numbers Within 20- Solution 4}%
\newline%
\end{center} \normalsize%
1) 79 {-} 15 = 64%
\newline%
2) 30 {-} 10 = 20%
\newline%
3) 57 {-} 16 = 41%
\newline%
4) 40 {-} 16 = 24%
\newline%
5) 44 {-} 18 = 26%
\newline%
6) 60 {-} 6 = 54%
\newline%
7) 41 {-} 12 = 29%
\newline%
8) 33 {-} 19 = 14%
\newline%
9) 84 {-} 5 = 79%
\newline%
10) 39 {-} 17 = 22%
\newline%
11) 59 {-} 5 = 54%
\newline%
12) 27 {-} 17 = 10%
\newline%
13) 74 {-} 6 = 68%
\newline%
14) 61 {-} 11 = 50%
\newline%
15) 97 {-} 6 = 91%
\newline%
16) 78 {-} 1 = 77%
\newline%
17) 25 {-} 12 = 13%
\newline%
18) 66 {-} 4 = 62%
\newline%
19) 42 {-} 12 = 30%
\newline%
20) 25 {-} 6 = 19%
\newline%
21) 60 {-} 15 = 45%
\newline%
22) 79 {-} 16 = 63%
\newline%
23) 55 {-} 15 = 40%
\newline%
24) 93 {-} 1 = 92%
\newline%
25) 69 {-} 5 = 64%
\newline%
26) Sally has 28 books.%
\newline%
27) Rachel has 58 pencils.%
\newline%
28) Alex has 88 bottles.%
\newline%
29) Laura has 49 books.%
\newline%
30) Laura has 60 bottles.%
\newline%
\newpage%
\large%
\begin{center}%
\textbf{Finding the Missing Number- Solution 1}%
\newline%
\end{center} \normalsize%
1) 36 + 26 = 62%
\newline%
2) 25 + 35 = 60%
\newline%
3) 13 + 75 = 88%
\newline%
4) 40 + 27 = 67%
\newline%
5) 12 + 35 = 47%
\newline%
6) 16 + 65 = 81%
\newline%
7) 36 + 15 = 51%
\newline%
8) 25 + 41 = 66%
\newline%
9) 42 + 46 = 88%
\newline%
10) 33 + 31 = 64%
\newline%
11) 26 + 59 = 85%
\newline%
12) 39 + 25 = 64%
\newline%
13) 28 + 18 = 46%
\newline%
14) 43 + 15 = 58%
\newline%
15) 12 + 74 = 86%
\newline%
16) 7 + 60 = 67%
\newline%
17) 15 + 73 = 88%
\newline%
18) 4 + 50 = 54%
\newline%
19) 12 + 60 = 72%
\newline%
20) 7 + 71 = 78%
\newline%
21) 33 + 33 = 66%
\newline%
22) 26 + 44 = 70%
\newline%
23) 34 + 22 = 56%
\newline%
24) 28 + 22 = 50%
\newline%
25) 40 + 36 = 76%
\newline%
26) Bob has 54 toys.%
\newline%
27) Sam has 25 toys.%
\newline%
28) Laura has 56 papers.%
\newline%
29) Sally has 54 marbles.%
\newline%
30) Laura has 13 toys.%
\newline%
\newpage%
\large%
\begin{center}%
\textbf{Finding the Missing Number- Solution 2}%
\newline%
\end{center} \normalsize%
1) 25 + 33 = 58%
\newline%
2) 2 + 60 = 62%
\newline%
3) 30 + 29 = 59%
\newline%
4) 16 + 61 = 77%
\newline%
5) 10 + 37 = 47%
\newline%
6) 41 + 8 = 49%
\newline%
7) 29 + 21 = 50%
\newline%
8) 9 + 51 = 60%
\newline%
9) 26 + 50 = 76%
\newline%
10) 5 + 79 = 84%
\newline%
11) 38 + 34 = 72%
\newline%
12) 45 + 2 = 47%
\newline%
13) 14 + 74 = 88%
\newline%
14) 26 + 34 = 60%
\newline%
15) 38 + 27 = 65%
\newline%
16) 42 + 48 = 90%
\newline%
17) 31 + 47 = 78%
\newline%
18) 32 + 22 = 54%
\newline%
19) 37 + 12 = 49%
\newline%
20) 17 + 49 = 66%
\newline%
21) 37 + 20 = 57%
\newline%
22) 23 + 36 = 59%
\newline%
23) 10 + 61 = 71%
\newline%
24) 6 + 52 = 58%
\newline%
25) 7 + 82 = 89%
\newline%
26) Sam has 58 marbles.%
\newline%
27) Bob has 24 pencils.%
\newline%
28) Bob has 71 dollars.%
\newline%
29) Laura has 45 bottles.%
\newline%
30) Rachel has 48 pencils.%
\newline%
\newpage%
\large%
\begin{center}%
\textbf{Finding the Missing Number- Solution 3}%
\newline%
\end{center} \normalsize%
1) 31 + 40 = 71%
\newline%
2) 26 + 40 = 66%
\newline%
3) 10 + 59 = 69%
\newline%
4) 37 + 16 = 53%
\newline%
5) 20 + 29 = 49%
\newline%
6) 20 + 46 = 66%
\newline%
7) 14 + 56 = 70%
\newline%
8) 23 + 28 = 51%
\newline%
9) 3 + 69 = 72%
\newline%
10) 44 + 7 = 51%
\newline%
11) 3 + 51 = 54%
\newline%
12) 40 + 50 = 90%
\newline%
13) 28 + 36 = 64%
\newline%
14) 17 + 29 = 46%
\newline%
15) 8 + 82 = 90%
\newline%
16) 18 + 40 = 58%
\newline%
17) 31 + 37 = 68%
\newline%
18) 8 + 68 = 76%
\newline%
19) 27 + 44 = 71%
\newline%
20) 18 + 36 = 54%
\newline%
21) 28 + 57 = 85%
\newline%
22) 45 + 2 = 47%
\newline%
23) 34 + 26 = 60%
\newline%
24) 33 + 41 = 74%
\newline%
25) 31 + 55 = 86%
\newline%
26) James has 83 dollars.%
\newline%
27) Laura has 29 bottles.%
\newline%
28) Rachel has 48 bottles.%
\newline%
29) Rachel has 70 marbles.%
\newline%
30) Rachel has 47 pens.%
\newline%
\newpage%
\large%
\begin{center}%
\textbf{Finding the Missing Number- Solution 4}%
\newline%
\end{center} \normalsize%
1) 16 + 38 = 54%
\newline%
2) 28 + 60 = 88%
\newline%
3) 45 + 9 = 54%
\newline%
4) 30 + 53 = 83%
\newline%
5) 10 + 63 = 73%
\newline%
6) 24 + 50 = 74%
\newline%
7) 24 + 52 = 76%
\newline%
8) 11 + 53 = 64%
\newline%
9) 18 + 31 = 49%
\newline%
10) 44 + 32 = 76%
\newline%
11) 43 + 15 = 58%
\newline%
12) 36 + 29 = 65%
\newline%
13) 32 + 31 = 63%
\newline%
14) 3 + 82 = 85%
\newline%
15) 41 + 45 = 86%
\newline%
16) 45 + 14 = 59%
\newline%
17) 34 + 19 = 53%
\newline%
18) 16 + 68 = 84%
\newline%
19) 38 + 44 = 82%
\newline%
20) 39 + 44 = 83%
\newline%
21) 45 + 36 = 81%
\newline%
22) 10 + 59 = 69%
\newline%
23) 42 + 14 = 56%
\newline%
24) 7 + 62 = 69%
\newline%
25) 13 + 42 = 55%
\newline%
26) Michael has 65 bottles.%
\newline%
27) Michael has 66 pens.%
\newline%
28) James has 8 papers.%
\newline%
29) Michael has 47 toys.%
\newline%
30) Sam has 59 books.%
\newline%
\newpage%
\large%
\begin{center}%
\textbf{Adding Three Numbers- Solution 1}%
\newline%
\end{center} \normalsize%
1) 7 + 12 + 11 = 30%
\newline%
2) 9 + 10 + 16 = 35%
\newline%
3) 9 + 19 + 19 = 47%
\newline%
4) 14 + 6 + 4 = 24%
\newline%
5) 3 + 5 + 17 = 25%
\newline%
6) 7 + 8 + 20 = 35%
\newline%
7) 6 + 8 + 16 = 30%
\newline%
8) 14 + 9 + 8 = 31%
\newline%
9) 17 + 8 + 20 = 45%
\newline%
10) 17 + 18 + 14 = 49%
\newline%
11) 15 + 16 + 7 = 38%
\newline%
12) 20 + 2 + 12 = 34%
\newline%
13) 20 + 20 + 3 = 43%
\newline%
14) 6 + 16 + 8 = 30%
\newline%
15) 5 + 14 + 9 = 28%
\newline%
16) 2 + 13 + 14 = 29%
\newline%
17) 3 + 14 + 9 = 26%
\newline%
18) 17 + 13 + 7 = 37%
\newline%
19) 14 + 12 + 12 = 38%
\newline%
20) 4 + 13 + 18 = 35%
\newline%
21) 6 + 9 + 4 = 19%
\newline%
22) 7 + 14 + 14 = 35%
\newline%
23) 17 + 19 + 15 = 51%
\newline%
24) 14 + 17 + 4 = 35%
\newline%
25) 15 + 16 + 11 = 42%
\newline%
26) Laura, Sam, and Alex have a total of 40 bottles.%
\newline%
27) James, Sam, and Bob have a total of 28 books.%
\newline%
28) Bob, Michael, and James have a total of 31 marbles.%
\newline%
29) Alex, Michael, and Bob have a total of 23 dollars.%
\newline%
30) Bob, Rachel, and Alex have a total of 40 marbles.%
\newline%
\newpage%
\large%
\begin{center}%
\textbf{Adding Three Numbers- Solution 2}%
\newline%
\end{center} \normalsize%
1) 7 + 11 + 17 = 35%
\newline%
2) 13 + 20 + 16 = 49%
\newline%
3) 2 + 15 + 6 = 23%
\newline%
4) 2 + 12 + 14 = 28%
\newline%
5) 17 + 12 + 20 = 49%
\newline%
6) 7 + 19 + 7 = 33%
\newline%
7) 5 + 3 + 17 = 25%
\newline%
8) 7 + 15 + 5 = 27%
\newline%
9) 4 + 15 + 11 = 30%
\newline%
10) 17 + 9 + 5 = 31%
\newline%
11) 3 + 6 + 14 = 23%
\newline%
12) 16 + 6 + 9 = 31%
\newline%
13) 12 + 11 + 18 = 41%
\newline%
14) 8 + 12 + 6 = 26%
\newline%
15) 6 + 18 + 4 = 28%
\newline%
16) 14 + 13 + 10 = 37%
\newline%
17) 11 + 17 + 2 = 30%
\newline%
18) 8 + 8 + 9 = 25%
\newline%
19) 13 + 4 + 3 = 20%
\newline%
20) 13 + 8 + 3 = 24%
\newline%
21) 14 + 2 + 9 = 25%
\newline%
22) 14 + 9 + 18 = 41%
\newline%
23) 11 + 18 + 11 = 40%
\newline%
24) 15 + 11 + 15 = 41%
\newline%
25) 12 + 8 + 3 = 23%
\newline%
26) Sam, Bob, and Rachel have a total of 35 books.%
\newline%
27) Rachel, Sally, and James have a total of 15 pens.%
\newline%
28) Sally, Alex, and Rachel have a total of 54 books.%
\newline%
29) Sam, Laura, and James have a total of 48 papers.%
\newline%
30) Alex, Sam, and Bob have a total of 30 books.%
\newline%
\newpage%
\large%
\begin{center}%
\textbf{Adding Three Numbers- Solution 3}%
\newline%
\end{center} \normalsize%
1) 13 + 18 + 4 = 35%
\newline%
2) 16 + 3 + 12 = 31%
\newline%
3) 20 + 6 + 18 = 44%
\newline%
4) 16 + 20 + 16 = 52%
\newline%
5) 12 + 14 + 20 = 46%
\newline%
6) 10 + 7 + 7 = 24%
\newline%
7) 2 + 8 + 14 = 24%
\newline%
8) 14 + 13 + 20 = 47%
\newline%
9) 5 + 3 + 17 = 25%
\newline%
10) 15 + 8 + 8 = 31%
\newline%
11) 15 + 5 + 13 = 33%
\newline%
12) 18 + 9 + 3 = 30%
\newline%
13) 4 + 10 + 5 = 19%
\newline%
14) 5 + 10 + 8 = 23%
\newline%
15) 12 + 18 + 12 = 42%
\newline%
16) 17 + 10 + 20 = 47%
\newline%
17) 20 + 18 + 2 = 40%
\newline%
18) 2 + 11 + 2 = 15%
\newline%
19) 13 + 5 + 6 = 24%
\newline%
20) 17 + 17 + 19 = 53%
\newline%
21) 14 + 19 + 4 = 37%
\newline%
22) 13 + 3 + 8 = 24%
\newline%
23) 11 + 12 + 19 = 42%
\newline%
24) 8 + 3 + 17 = 28%
\newline%
25) 2 + 14 + 15 = 31%
\newline%
26) Alex, Sam, and Michael have a total of 36 marbles.%
\newline%
27) Sam, Michael, and Alex have a total of 21 books.%
\newline%
28) Bob, Alex, and James have a total of 23 pens.%
\newline%
29) Michael, Sally, and James have a total of 30 pens.%
\newline%
30) Rachel, Michael, and Alex have a total of 24 books.%
\newline%
\newpage%
\large%
\begin{center}%
\textbf{Adding Three Numbers- Solution 4}%
\newline%
\end{center} \normalsize%
1) 17 + 18 + 16 = 51%
\newline%
2) 10 + 9 + 14 = 33%
\newline%
3) 19 + 7 + 12 = 38%
\newline%
4) 10 + 15 + 18 = 43%
\newline%
5) 6 + 12 + 12 = 30%
\newline%
6) 6 + 4 + 12 = 22%
\newline%
7) 12 + 8 + 4 = 24%
\newline%
8) 2 + 17 + 15 = 34%
\newline%
9) 7 + 17 + 2 = 26%
\newline%
10) 20 + 4 + 15 = 39%
\newline%
11) 19 + 16 + 10 = 45%
\newline%
12) 6 + 18 + 5 = 29%
\newline%
13) 16 + 17 + 19 = 52%
\newline%
14) 14 + 20 + 5 = 39%
\newline%
15) 4 + 12 + 8 = 24%
\newline%
16) 10 + 10 + 18 = 38%
\newline%
17) 4 + 20 + 14 = 38%
\newline%
18) 2 + 11 + 11 = 24%
\newline%
19) 2 + 11 + 4 = 17%
\newline%
20) 13 + 14 + 4 = 31%
\newline%
21) 20 + 19 + 18 = 57%
\newline%
22) 2 + 13 + 12 = 27%
\newline%
23) 16 + 11 + 10 = 37%
\newline%
24) 20 + 15 + 3 = 38%
\newline%
25) 6 + 17 + 19 = 42%
\newline%
26) Rachel, Laura, and Michael have a total of 38 bottles.%
\newline%
27) Sally, Rachel, and Laura have a total of 51 marbles.%
\newline%
28) Michael, Laura, and Sally have a total of 35 pens.%
\newline%
29) James, Rachel, and Sally have a total of 38 books.%
\newline%
30) James, Michael, and Rachel have a total of 39 papers.%
\newline%
\newpage%
\large%
\begin{center}%
\textbf{Before and After- Solution 1}%
\newline%
\end{center} \normalsize%
1) 22 comes 6 before 28.%
\newline%
2) 136 comes 8 after 128.%
\newline%
3) 91 comes 6 before 97.%
\newline%
4) 40 comes 8 after 32.%
\newline%
5) 65 comes 3 after 62.%
\newline%
6) 88 comes 9 after 79.%
\newline%
7) 18 comes 5 before 23.%
\newline%
8) 34 comes 7 before 41.%
\newline%
9) 113 comes 6 after 107.%
\newline%
10) 52 comes 10 before 62.%
\newline%
11) 131 comes 9 before 140.%
\newline%
12) 79 comes 5 after 74.%
\newline%
13) 108 comes 7 after 101.%
\newline%
14) 133 comes 3 before 136.%
\newline%
15) 96 comes 10 after 86.%
\newline%
16) 61 comes 3 after 58.%
\newline%
17) 111 comes 1 after 110.%
\newline%
18) 74 comes 3 before 77.%
\newline%
19) 150 comes 10 after 140.%
\newline%
20) 89 comes 1 after 88.%
\newline%
21) 66 comes 9 after 57.%
\newline%
22) 42 comes 1 before 43.%
\newline%
23) 55 comes 7 after 48.%
\newline%
24) 123 comes 5 after 118.%
\newline%
25) 92 comes 8 before 100.%
\newline%
26) 133 comes 1 after 132.%
\newline%
27) 61 comes 8 after 53.%
\newline%
28) 89 comes 4 after 85.%
\newline%
29) 135 comes 10 after 125.%
\newline%
30) 34 comes 1 before 35.%
\newline%
\newpage%
\large%
\begin{center}%
\textbf{Before and After- Solution 2}%
\newline%
\end{center} \normalsize%
1) 114 comes 4 after 110.%
\newline%
2) 55 comes 7 after 48.%
\newline%
3) 53 comes 9 before 62.%
\newline%
4) 113 comes 4 before 117.%
\newline%
5) 21 comes 3 before 24.%
\newline%
6) 21 comes 1 after 20.%
\newline%
7) 146 comes 10 after 136.%
\newline%
8) 59 comes 7 before 66.%
\newline%
9) 86 comes 8 before 94.%
\newline%
10) 125 comes 3 after 122.%
\newline%
11) 15 comes 7 before 22.%
\newline%
12) 65 comes 4 before 69.%
\newline%
13) 31 comes 4 after 27.%
\newline%
14) 17 comes 6 after 11.%
\newline%
15) 37 comes 6 after 31.%
\newline%
16) 54 comes 8 after 46.%
\newline%
17) 72 comes 1 after 71.%
\newline%
18) 64 comes 9 before 73.%
\newline%
19) 25 comes 6 before 31.%
\newline%
20) 54 comes 6 before 60.%
\newline%
21) 101 comes 3 after 98.%
\newline%
22) 99 comes 6 after 93.%
\newline%
23) 41 comes 2 before 43.%
\newline%
24) 147 comes 8 after 139.%
\newline%
25) 95 comes 10 before 105.%
\newline%
26) 145 comes 1 before 146.%
\newline%
27) 32 comes 10 before 42.%
\newline%
28) 60 comes 5 after 55.%
\newline%
29) 127 comes 3 after 124.%
\newline%
30) 28 comes 4 before 32.%
\newline%
\newpage%
\large%
\begin{center}%
\textbf{Before and After- Solution 3}%
\newline%
\end{center} \normalsize%
1) 43 comes 10 after 33.%
\newline%
2) 80 comes 7 after 73.%
\newline%
3) 41 comes 2 after 39.%
\newline%
4) 65 comes 1 after 64.%
\newline%
5) 22 comes 8 before 30.%
\newline%
6) 72 comes 10 before 82.%
\newline%
7) 38 comes 9 before 47.%
\newline%
8) 30 comes 5 before 35.%
\newline%
9) 12 comes 5 before 17.%
\newline%
10) 136 comes 1 after 135.%
\newline%
11) 139 comes 3 after 136.%
\newline%
12) 66 comes 2 before 68.%
\newline%
13) 59 comes 4 before 63.%
\newline%
14) 73 comes 5 after 68.%
\newline%
15) 31 comes 4 after 27.%
\newline%
16) 137 comes 2 after 135.%
\newline%
17) 58 comes 9 after 49.%
\newline%
18) 19 comes 5 after 14.%
\newline%
19) 36 comes 7 before 43.%
\newline%
20) 22 comes 3 after 19.%
\newline%
21) 70 comes 4 after 66.%
\newline%
22) 120 comes 7 before 127.%
\newline%
23) 39 comes 9 before 48.%
\newline%
24) 99 comes 9 before 108.%
\newline%
25) 114 comes 2 before 116.%
\newline%
26) 36 comes 4 after 32.%
\newline%
27) 126 comes 9 after 117.%
\newline%
28) 7 comes 6 before 13.%
\newline%
29) 11 comes 2 before 13.%
\newline%
30) 64 comes 2 after 62.%
\newline%
\newpage%
\large%
\begin{center}%
\textbf{Before and After- Solution 4}%
\newline%
\end{center} \normalsize%
1) 143 comes 9 after 134.%
\newline%
2) 91 comes 5 after 86.%
\newline%
3) 120 comes 3 after 117.%
\newline%
4) 11 comes 9 before 20.%
\newline%
5) 83 comes 1 before 84.%
\newline%
6) 54 comes 7 before 61.%
\newline%
7) 79 comes 1 before 80.%
\newline%
8) 137 comes 6 before 143.%
\newline%
9) 156 comes 8 after 148.%
\newline%
10) 148 comes 1 before 149.%
\newline%
11) 13 comes 10 before 23.%
\newline%
12) 33 comes 1 before 34.%
\newline%
13) 35 comes 2 before 37.%
\newline%
14) 152 comes 9 after 143.%
\newline%
15) 86 comes 7 after 79.%
\newline%
16) 138 comes 7 before 145.%
\newline%
17) 63 comes 3 after 60.%
\newline%
18) 87 comes 7 before 94.%
\newline%
19) 151 comes 2 after 149.%
\newline%
20) 12 comes 2 before 14.%
\newline%
21) 33 comes 2 before 35.%
\newline%
22) 16 comes 4 before 20.%
\newline%
23) 62 comes 10 before 72.%
\newline%
24) 16 comes 9 before 25.%
\newline%
25) 30 comes 5 before 35.%
\newline%
26) 101 comes 6 before 107.%
\newline%
27) 110 comes 2 before 112.%
\newline%
28) 154 comes 6 after 148.%
\newline%
29) 129 comes 2 before 131.%
\newline%
30) 74 comes 3 after 71.%
\newline%
\newpage%
\large%
\begin{center}%
\textbf{Basic Place Value- Solution 1}%
\newline%
\end{center} \normalsize%
1) 59 is equivalent to  5 tens and 9 ones.%
\newline%
2) 47 is equivalent to  4 tens and 7 ones.%
\newline%
3) 93 is equivalent to  9 tens and 3 ones.%
\newline%
4) 84 is equivalent to  8 tens and 4 ones.%
\newline%
5) 65 is equivalent to  6 tens and 5 ones.%
\newline%
6) 49 is equivalent to  4 tens and 9 ones.%
\newline%
7) 104 is equivalent to  10 tens and 4 ones.%
\newline%
8) 92 is equivalent to  9 tens and 2 ones.%
\newline%
9) 86 is equivalent to  8 tens and 6 ones.%
\newline%
10) 88 is equivalent to  8 tens and 8 ones.%
\newline%
11) 33 is equivalent to  3 tens and 3 ones.%
\newline%
12) 95 is equivalent to  9 tens and 5 ones.%
\newline%
13) 82 is equivalent to  8 tens and 2 ones.%
\newline%
14) 25 is equivalent to  2 tens and 5 ones.%
\newline%
15) 24 is equivalent to  2 tens and 4 ones.%
\newline%
16) 105 is equivalent to  10 tens and 5 ones.%
\newline%
17) 94 is equivalent to  9 tens and 4 ones.%
\newline%
18) 34 is equivalent to  3 tens and 4 ones.%
\newline%
19) 72 is equivalent to  7 tens and 2 ones.%
\newline%
20) 98 is equivalent to  9 tens and 8 ones.%
\newline%
21) 74 is equivalent to  7 tens and 4 ones.%
\newline%
22) 46 is equivalent to  4 tens and 6 ones.%
\newline%
23) 26 is equivalent to  2 tens and 6 ones.%
\newline%
24) 109 is equivalent to  10 tens and 9 ones.%
\newline%
25) 108 is equivalent to  10 tens and 8 ones.%
\newline%
26) 89 is equivalent to  8 tens and 9 ones.%
\newline%
27) 36 is equivalent to  3 tens and 6 ones.%
\newline%
28) 42 is equivalent to  4 tens and 2 ones.%
\newline%
29) 38 is equivalent to  3 tens and 8 ones.%
\newline%
30) 35 is equivalent to  3 tens and 5 ones.%
\newline%
\newpage%
\large%
\begin{center}%
\textbf{Basic Place Value- Solution 2}%
\newline%
\end{center} \normalsize%
1) 36 is equivalent to  3 tens and 6 ones.%
\newline%
2) 59 is equivalent to  5 tens and 9 ones.%
\newline%
3) 74 is equivalent to  7 tens and 4 ones.%
\newline%
4) 79 is equivalent to  7 tens and 9 ones.%
\newline%
5) 109 is equivalent to  10 tens and 9 ones.%
\newline%
6) 105 is equivalent to  10 tens and 5 ones.%
\newline%
7) 54 is equivalent to  5 tens and 4 ones.%
\newline%
8) 95 is equivalent to  9 tens and 5 ones.%
\newline%
9) 106 is equivalent to  10 tens and 6 ones.%
\newline%
10) 89 is equivalent to  8 tens and 9 ones.%
\newline%
11) 29 is equivalent to  2 tens and 9 ones.%
\newline%
12) 33 is equivalent to  3 tens and 3 ones.%
\newline%
13) 82 is equivalent to  8 tens and 2 ones.%
\newline%
14) 47 is equivalent to  4 tens and 7 ones.%
\newline%
15) 93 is equivalent to  9 tens and 3 ones.%
\newline%
16) 73 is equivalent to  7 tens and 3 ones.%
\newline%
17) 75 is equivalent to  7 tens and 5 ones.%
\newline%
18) 69 is equivalent to  6 tens and 9 ones.%
\newline%
19) 84 is equivalent to  8 tens and 4 ones.%
\newline%
20) 76 is equivalent to  7 tens and 6 ones.%
\newline%
21) 49 is equivalent to  4 tens and 9 ones.%
\newline%
22) 43 is equivalent to  4 tens and 3 ones.%
\newline%
23) 67 is equivalent to  6 tens and 7 ones.%
\newline%
24) 35 is equivalent to  3 tens and 5 ones.%
\newline%
25) 87 is equivalent to  8 tens and 7 ones.%
\newline%
26) 57 is equivalent to  5 tens and 7 ones.%
\newline%
27) 28 is equivalent to  2 tens and 8 ones.%
\newline%
28) 53 is equivalent to  5 tens and 3 ones.%
\newline%
29) 22 is equivalent to  2 tens and 2 ones.%
\newline%
30) 24 is equivalent to  2 tens and 4 ones.%
\newline%
\newpage%
\large%
\begin{center}%
\textbf{Basic Place Value- Solution 3}%
\newline%
\end{center} \normalsize%
1) 107 is equivalent to  10 tens and 7 ones.%
\newline%
2) 103 is equivalent to  10 tens and 3 ones.%
\newline%
3) 77 is equivalent to  7 tens and 7 ones.%
\newline%
4) 22 is equivalent to  2 tens and 2 ones.%
\newline%
5) 49 is equivalent to  4 tens and 9 ones.%
\newline%
6) 93 is equivalent to  9 tens and 3 ones.%
\newline%
7) 48 is equivalent to  4 tens and 8 ones.%
\newline%
8) 83 is equivalent to  8 tens and 3 ones.%
\newline%
9) 78 is equivalent to  7 tens and 8 ones.%
\newline%
10) 26 is equivalent to  2 tens and 6 ones.%
\newline%
11) 89 is equivalent to  8 tens and 9 ones.%
\newline%
12) 94 is equivalent to  9 tens and 4 ones.%
\newline%
13) 27 is equivalent to  2 tens and 7 ones.%
\newline%
14) 57 is equivalent to  5 tens and 7 ones.%
\newline%
15) 67 is equivalent to  6 tens and 7 ones.%
\newline%
16) 52 is equivalent to  5 tens and 2 ones.%
\newline%
17) 79 is equivalent to  7 tens and 9 ones.%
\newline%
18) 24 is equivalent to  2 tens and 4 ones.%
\newline%
19) 98 is equivalent to  9 tens and 8 ones.%
\newline%
20) 23 is equivalent to  2 tens and 3 ones.%
\newline%
21) 32 is equivalent to  3 tens and 2 ones.%
\newline%
22) 68 is equivalent to  6 tens and 8 ones.%
\newline%
23) 62 is equivalent to  6 tens and 2 ones.%
\newline%
24) 35 is equivalent to  3 tens and 5 ones.%
\newline%
25) 109 is equivalent to  10 tens and 9 ones.%
\newline%
26) 86 is equivalent to  8 tens and 6 ones.%
\newline%
27) 47 is equivalent to  4 tens and 7 ones.%
\newline%
28) 82 is equivalent to  8 tens and 2 ones.%
\newline%
29) 55 is equivalent to  5 tens and 5 ones.%
\newline%
30) 64 is equivalent to  6 tens and 4 ones.%
\newline%
\newpage%
\large%
\begin{center}%
\textbf{Basic Place Value- Solution 4}%
\newline%
\end{center} \normalsize%
1) 106 is equivalent to  10 tens and 6 ones.%
\newline%
2) 64 is equivalent to  6 tens and 4 ones.%
\newline%
3) 33 is equivalent to  3 tens and 3 ones.%
\newline%
4) 38 is equivalent to  3 tens and 8 ones.%
\newline%
5) 49 is equivalent to  4 tens and 9 ones.%
\newline%
6) 65 is equivalent to  6 tens and 5 ones.%
\newline%
7) 76 is equivalent to  7 tens and 6 ones.%
\newline%
8) 78 is equivalent to  7 tens and 8 ones.%
\newline%
9) 109 is equivalent to  10 tens and 9 ones.%
\newline%
10) 29 is equivalent to  2 tens and 9 ones.%
\newline%
11) 68 is equivalent to  6 tens and 8 ones.%
\newline%
12) 36 is equivalent to  3 tens and 6 ones.%
\newline%
13) 73 is equivalent to  7 tens and 3 ones.%
\newline%
14) 24 is equivalent to  2 tens and 4 ones.%
\newline%
15) 98 is equivalent to  9 tens and 8 ones.%
\newline%
16) 108 is equivalent to  10 tens and 8 ones.%
\newline%
17) 84 is equivalent to  8 tens and 4 ones.%
\newline%
18) 89 is equivalent to  8 tens and 9 ones.%
\newline%
19) 77 is equivalent to  7 tens and 7 ones.%
\newline%
20) 66 is equivalent to  6 tens and 6 ones.%
\newline%
21) 54 is equivalent to  5 tens and 4 ones.%
\newline%
22) 88 is equivalent to  8 tens and 8 ones.%
\newline%
23) 32 is equivalent to  3 tens and 2 ones.%
\newline%
24) 55 is equivalent to  5 tens and 5 ones.%
\newline%
25) 26 is equivalent to  2 tens and 6 ones.%
\newline%
26) 22 is equivalent to  2 tens and 2 ones.%
\newline%
27) 94 is equivalent to  9 tens and 4 ones.%
\newline%
28) 105 is equivalent to  10 tens and 5 ones.%
\newline%
29) 46 is equivalent to  4 tens and 6 ones.%
\newline%
30) 52 is equivalent to  5 tens and 2 ones.%
\newline%
\newpage%
\large%
\begin{center}%
\textbf{Breaking Apart Numbers- Solution 1}%
\newline%
\end{center} \normalsize%
1) 24 + 3 is equivalent to 2 tens 7 ones.%
\newline%
2) 35 + 3 is equivalent to 3 tens 8 ones.%
\newline%
3) 42 + 30 is equivalent to 7 tens 2 ones.%
\newline%
4) 20 + 23 is equivalent to 4 tens 3 ones.%
\newline%
5) 21 + 23 is equivalent to 4 tens 4 ones.%
\newline%
6) 7 + 3 is equivalent to 1 tens 0 ones.%
\newline%
7) 6 + 38 is equivalent to 4 tens 4 ones.%
\newline%
8) 26 + 30 is equivalent to 5 tens 6 ones.%
\newline%
9) 12 + 16 is equivalent to 2 tens 8 ones.%
\newline%
10) 39 + 41 is equivalent to 8 tens 0 ones.%
\newline%
11) 4 + 16 is equivalent to 2 tens 0 ones.%
\newline%
12) 44 + 38 is equivalent to 8 tens 2 ones.%
\newline%
13) 7 + 33 is equivalent to 4 tens 0 ones.%
\newline%
14) 43 + 10 is equivalent to 5 tens 3 ones.%
\newline%
15) 37 + 35 is equivalent to 7 tens 2 ones.%
\newline%
16) 45 + 20 is equivalent to 6 tens 5 ones.%
\newline%
17) 23 + 4 is equivalent to 2 tens 7 ones.%
\newline%
18) 2 + 38 is equivalent to 4 tens 0 ones.%
\newline%
19) 42 + 37 is equivalent to 7 tens 9 ones.%
\newline%
20) 33 + 10 is equivalent to 4 tens 3 ones.%
\newline%
21) 17 + 44 is equivalent to 6 tens 1 ones.%
\newline%
22) 40 + 36 is equivalent to 7 tens 6 ones.%
\newline%
23) 12 + 38 is equivalent to 5 tens 0 ones.%
\newline%
24) 42 + 1 is equivalent to 4 tens 3 ones.%
\newline%
25) 8 + 1 is equivalent to 0 tens 9 ones.%
\newline%
26) 41 + 14 is equivalent to 5 tens 5 ones.%
\newline%
27) 31 + 11 is equivalent to 4 tens 2 ones.%
\newline%
28) 45 + 21 is equivalent to 6 tens 6 ones.%
\newline%
29) 10 + 17 is equivalent to 2 tens 7 ones.%
\newline%
30) 27 + 25 is equivalent to 5 tens 2 ones.%
\newline%
\newpage%
\large%
\begin{center}%
\textbf{Breaking Apart Numbers- Solution 2}%
\newline%
\end{center} \normalsize%
1) 30 + 6 is equivalent to 3 tens 6 ones.%
\newline%
2) 11 + 41 is equivalent to 5 tens 2 ones.%
\newline%
3) 38 + 27 is equivalent to 6 tens 5 ones.%
\newline%
4) 21 + 22 is equivalent to 4 tens 3 ones.%
\newline%
5) 14 + 29 is equivalent to 4 tens 3 ones.%
\newline%
6) 26 + 26 is equivalent to 5 tens 2 ones.%
\newline%
7) 26 + 22 is equivalent to 4 tens 8 ones.%
\newline%
8) 30 + 35 is equivalent to 6 tens 5 ones.%
\newline%
9) 4 + 19 is equivalent to 2 tens 3 ones.%
\newline%
10) 21 + 11 is equivalent to 3 tens 2 ones.%
\newline%
11) 39 + 7 is equivalent to 4 tens 6 ones.%
\newline%
12) 17 + 3 is equivalent to 2 tens 0 ones.%
\newline%
13) 10 + 9 is equivalent to 1 tens 9 ones.%
\newline%
14) 30 + 14 is equivalent to 4 tens 4 ones.%
\newline%
15) 33 + 35 is equivalent to 6 tens 8 ones.%
\newline%
16) 14 + 4 is equivalent to 1 tens 8 ones.%
\newline%
17) 19 + 31 is equivalent to 5 tens 0 ones.%
\newline%
18) 28 + 34 is equivalent to 6 tens 2 ones.%
\newline%
19) 21 + 26 is equivalent to 4 tens 7 ones.%
\newline%
20) 7 + 30 is equivalent to 3 tens 7 ones.%
\newline%
21) 23 + 19 is equivalent to 4 tens 2 ones.%
\newline%
22) 8 + 6 is equivalent to 1 tens 4 ones.%
\newline%
23) 28 + 19 is equivalent to 4 tens 7 ones.%
\newline%
24) 19 + 39 is equivalent to 5 tens 8 ones.%
\newline%
25) 15 + 31 is equivalent to 4 tens 6 ones.%
\newline%
26) 9 + 26 is equivalent to 3 tens 5 ones.%
\newline%
27) 30 + 27 is equivalent to 5 tens 7 ones.%
\newline%
28) 5 + 18 is equivalent to 2 tens 3 ones.%
\newline%
29) 36 + 8 is equivalent to 4 tens 4 ones.%
\newline%
30) 7 + 7 is equivalent to 1 tens 4 ones.%
\newline%
\newpage%
\large%
\begin{center}%
\textbf{Breaking Apart Numbers- Solution 3}%
\newline%
\end{center} \normalsize%
1) 12 + 19 is equivalent to 3 tens 1 ones.%
\newline%
2) 22 + 11 is equivalent to 3 tens 3 ones.%
\newline%
3) 7 + 1 is equivalent to 0 tens 8 ones.%
\newline%
4) 41 + 8 is equivalent to 4 tens 9 ones.%
\newline%
5) 15 + 44 is equivalent to 5 tens 9 ones.%
\newline%
6) 18 + 25 is equivalent to 4 tens 3 ones.%
\newline%
7) 16 + 5 is equivalent to 2 tens 1 ones.%
\newline%
8) 6 + 24 is equivalent to 3 tens 0 ones.%
\newline%
9) 19 + 21 is equivalent to 4 tens 0 ones.%
\newline%
10) 20 + 16 is equivalent to 3 tens 6 ones.%
\newline%
11) 10 + 26 is equivalent to 3 tens 6 ones.%
\newline%
12) 8 + 14 is equivalent to 2 tens 2 ones.%
\newline%
13) 45 + 26 is equivalent to 7 tens 1 ones.%
\newline%
14) 28 + 44 is equivalent to 7 tens 2 ones.%
\newline%
15) 44 + 9 is equivalent to 5 tens 3 ones.%
\newline%
16) 40 + 34 is equivalent to 7 tens 4 ones.%
\newline%
17) 32 + 24 is equivalent to 5 tens 6 ones.%
\newline%
18) 38 + 19 is equivalent to 5 tens 7 ones.%
\newline%
19) 26 + 24 is equivalent to 5 tens 0 ones.%
\newline%
20) 15 + 13 is equivalent to 2 tens 8 ones.%
\newline%
21) 2 + 35 is equivalent to 3 tens 7 ones.%
\newline%
22) 3 + 18 is equivalent to 2 tens 1 ones.%
\newline%
23) 15 + 43 is equivalent to 5 tens 8 ones.%
\newline%
24) 41 + 4 is equivalent to 4 tens 5 ones.%
\newline%
25) 35 + 12 is equivalent to 4 tens 7 ones.%
\newline%
26) 33 + 7 is equivalent to 4 tens 0 ones.%
\newline%
27) 45 + 21 is equivalent to 6 tens 6 ones.%
\newline%
28) 8 + 9 is equivalent to 1 tens 7 ones.%
\newline%
29) 12 + 4 is equivalent to 1 tens 6 ones.%
\newline%
30) 26 + 43 is equivalent to 6 tens 9 ones.%
\newline%
\newpage%
\large%
\begin{center}%
\textbf{Breaking Apart Numbers- Solution 4}%
\newline%
\end{center} \normalsize%
1) 9 + 8 is equivalent to 1 tens 7 ones.%
\newline%
2) 45 + 32 is equivalent to 7 tens 7 ones.%
\newline%
3) 28 + 24 is equivalent to 5 tens 2 ones.%
\newline%
4) 33 + 30 is equivalent to 6 tens 3 ones.%
\newline%
5) 12 + 31 is equivalent to 4 tens 3 ones.%
\newline%
6) 23 + 35 is equivalent to 5 tens 8 ones.%
\newline%
7) 25 + 8 is equivalent to 3 tens 3 ones.%
\newline%
8) 31 + 40 is equivalent to 7 tens 1 ones.%
\newline%
9) 9 + 2 is equivalent to 1 tens 1 ones.%
\newline%
10) 10 + 38 is equivalent to 4 tens 8 ones.%
\newline%
11) 29 + 44 is equivalent to 7 tens 3 ones.%
\newline%
12) 33 + 14 is equivalent to 4 tens 7 ones.%
\newline%
13) 10 + 6 is equivalent to 1 tens 6 ones.%
\newline%
14) 30 + 38 is equivalent to 6 tens 8 ones.%
\newline%
15) 5 + 20 is equivalent to 2 tens 5 ones.%
\newline%
16) 26 + 13 is equivalent to 3 tens 9 ones.%
\newline%
17) 12 + 10 is equivalent to 2 tens 2 ones.%
\newline%
18) 45 + 10 is equivalent to 5 tens 5 ones.%
\newline%
19) 4 + 9 is equivalent to 1 tens 3 ones.%
\newline%
20) 26 + 8 is equivalent to 3 tens 4 ones.%
\newline%
21) 20 + 41 is equivalent to 6 tens 1 ones.%
\newline%
22) 18 + 8 is equivalent to 2 tens 6 ones.%
\newline%
23) 35 + 9 is equivalent to 4 tens 4 ones.%
\newline%
24) 28 + 31 is equivalent to 5 tens 9 ones.%
\newline%
25) 21 + 9 is equivalent to 3 tens 0 ones.%
\newline%
26) 23 + 17 is equivalent to 4 tens 0 ones.%
\newline%
27) 3 + 20 is equivalent to 2 tens 3 ones.%
\newline%
28) 18 + 38 is equivalent to 5 tens 6 ones.%
\newline%
29) 10 + 20 is equivalent to 3 tens 0 ones.%
\newline%
30) 16 + 41 is equivalent to 5 tens 7 ones.%
\newline%
\newpage%
\end{document}