\documentclass{article}%
\usepackage[T1]{fontenc}%
\usepackage[utf8]{inputenc}%
\usepackage{lmodern}%
\usepackage{textcomp}%
\usepackage{lastpage}%
\usepackage{tikz}%
\usepackage{pgfplots}%
%
\title{Algebra 1 Math Workbook}%
\author{Akshai Srinivasan, Teja Koripella, Skye Tyrrell, Angellou Sutharsan}%
\date{}%
%
\begin{document}%
\normalsize%
\maketitle%
\vfill%
\begin{center}%
ISBN: 9798848825312%
\linebreak%
\copyright%
MathMaestro.org 2022%
\end{center}%
\newpage%
\section{Table of Contents}%
\label{sec:TableofContents}%
Problems...........................................................................................Page 3%
\newline%
Solving Basic Algebraic Equations%
.%
.%
.%
.%
.%
.%
.%
.%
.%
.%
.%
.%
.%
.%
.%
.%
.%
.%
.%
.%
.%
.%
.%
.%
.%
.%
.%
.%
.%
.%
.%
.%
.%
.%
.%
.%
.%
.%
.%
.%
.%
.%
.%
.%
.%
Page 4{-}15%
\newline%
Solving Inequalities%
.%
.%
.%
.%
.%
.%
.%
.%
.%
.%
.%
.%
.%
.%
.%
.%
.%
.%
.%
.%
.%
.%
.%
.%
.%
.%
.%
.%
.%
.%
.%
.%
.%
.%
.%
.%
.%
.%
.%
.%
.%
.%
.%
.%
.%
.%
.%
.%
.%
.%
.%
.%
.%
.%
.%
.%
.%
.%
.%
.%
.%
.%
.%
.%
.%
.%
.%
.%
.%
Page 16{-}27%
\newline%
Factoring Basic Quadratic Equations%
.%
.%
.%
.%
.%
.%
.%
.%
.%
.%
.%
.%
.%
.%
.%
.%
.%
.%
.%
.%
.%
.%
.%
.%
.%
.%
.%
.%
.%
.%
.%
.%
.%
.%
.%
.%
.%
.%
.%
.%
.%
.%
Page 28{-}39%
\newline%
Factoring Advanced Quadratic Equations%
.%
.%
.%
.%
.%
.%
.%
.%
.%
.%
.%
.%
.%
.%
.%
.%
.%
.%
.%
.%
.%
.%
.%
.%
.%
.%
.%
.%
.%
.%
.%
.%
.%
.%
.%
.%
Page 40{-}51%
\newline%
Finding the Equation of a Line%
.%
.%
.%
.%
.%
.%
.%
.%
.%
.%
.%
.%
.%
.%
.%
.%
.%
.%
.%
.%
.%
.%
.%
.%
.%
.%
.%
.%
.%
.%
.%
.%
.%
.%
.%
.%
.%
.%
.%
.%
.%
.%
.%
.%
.%
.%
.%
.%
.%
.%
.%
Page 52{-}71%
\newline%
Solutions......................................................................................Page 72%
\newline%
Solving Basic Algebraic Equations Solutions%
.%
.%
.%
.%
.%
.%
.%
.%
.%
.%
.%
.%
.%
.%
.%
.%
.%
.%
.%
.%
.%
.%
.%
.%
.%
.%
.%
Page 73{-}76%
\newline%
Solving Inequalities Solutions%
.%
.%
.%
.%
.%
.%
.%
.%
.%
.%
.%
.%
.%
.%
.%
.%
.%
.%
.%
.%
.%
.%
.%
.%
.%
.%
.%
.%
.%
.%
.%
.%
.%
.%
.%
.%
.%
.%
.%
.%
.%
.%
.%
.%
.%
.%
.%
.%
.%
.%
.%
Page 77{-}80%
\newline%
Factoring Basic Quadratic Equations Solutions%
.%
.%
.%
.%
.%
.%
.%
.%
.%
.%
.%
.%
.%
.%
.%
.%
.%
.%
.%
.%
.%
.%
.%
.%
Page 81{-}84%
\newline%
Factoring Advanced Quadratic Equations Solutions%
.%
.%
.%
.%
.%
.%
.%
.%
.%
.%
.%
.%
.%
.%
.%
.%
.%
.%
Page 85{-}88%
\newline%
Finding the Equation of a Line Solutions%
.%
.%
.%
.%
.%
.%
.%
.%
.%
.%
.%
.%
.%
.%
.%
.%
.%
.%
.%
.%
.%
.%
.%
.%
.%
.%
.%
.%
.%
.%
.%
.%
.%
Page 89{-}92%
\newline%
\newpage

%
\huge%
\vspace*{\fill}%
\begin{center}%
Problems%
\end{center}%
\vspace*{\fill}%
\pagebreak%
\normalsize%
\large%
\begin{center}%
\textbf{Solving Basic Algebraic Equations- Worksheet 1}%
\newline%
\newline%
\newline%
\end{center} \normalsize%
1) 9x - 9 = 36%
\newline%
\newline%
\newline%
2) x + 7 = 12%
\newline%
\newline%
\newline%
3) 2x + 1 = 13%
\newline%
\newline%
\newline%
4) 2x - 7 = -1%
\newline%
\newline%
\newline%
5) 7x + 9 = 51%
\newline%
\newline%
\newline%
6) 2x + 6 = 10%
\newline%
\newline%
\newline%
7) 5x - 7 = 33%
\newline%
\newline%
\newline%
8) 6x + 6 = 36%
\newline%
\newline%
\newline%
9) 2x + 4 = 10%
\newline%
\newline%
\newline%
10) x - 3 = 5%
\newline%
\newline%
\newline%
11) 2x - 3 = 5%
\newline%
\newline%
\newline%
12) 2x + 5 = 15%
\newline%
\newline%
\newline%
13) 4x - 8 = 28%
\newline%
\newline%
\newline%
14) 7x - 3 = 60%
\newline%
\newline%
\newline%
15) 2x - 4 = 4%
\newline%
\newline%
\newline%
16) 6x + 9 = 15%
\newline%
\newline%
\newline%
17) 9x - 3 = 51%
\newline%
\newline%
\newline%
18) 7x - 9 = 40%
\newline%
\newline%
\newline%
19) 2x + 4 = 12%
\newline%
\newline%
\newline%
20) x + 5 = 6%
\newline%
\newline%
\newline%
21) 2x - 1 = 15%
\newline%
\newline%
\newline%
22) 8x - 1 = 7%
\newline%
\newline%
\newline%
23) 2x - 9 = -3%
\newline%
\newline%
\newline%
24) 3x + 8 = 14%
\newline%
\newline%
\newline%
25) 5x + 4 = 24%
\newline%
\newline%
\newline%
26) Laura had an unknown number of books. They got 4 times as many books before getting 4 more. In total, they now have 8 books. How many objects did Laura start with?%
\newline%
\newline%
\newline%
27) Laura had an unknown number of pencils. They got 3 times as many pencils before losing 5 more. In total, they now have -2 pencils. How many objects did Laura start with?%
\newline%
\newline%
\newline%
28) Alex had an unknown number of marbles. They got 2 times as many marbles before getting 3 more. In total, they now have 7 marbles. How many objects did Alex start with?%
\newline%
\newline%
\newline%
29) Laura had an unknown number of bottles. They got 9 times as many bottles before getting 4 more. In total, they now have 67 bottles. How many objects did Laura start with?%
\newline%
\newline%
\newline%
30) Sally had an unknown number of pens. They got 2 times as many pens before getting 7 more. In total, they now have 17 pens. How many objects did Sally start with?%
\newline%
\newline%
\newline%
\pagebreak%
\large%
\begin{center}%
\textbf{Solving Basic Algebraic Equations- Worksheet 2}%
\newline%
\newline%
\newline%
\end{center} \normalsize%
1) 9x - 9 = 36%
\newline%
\newline%
\newline%
2) x + 7 = 12%
\newline%
\newline%
\newline%
3) 2x + 1 = 13%
\newline%
\newline%
\newline%
4) 2x - 7 = -1%
\newline%
\newline%
\newline%
5) 7x + 9 = 51%
\newline%
\newline%
\newline%
6) 2x + 6 = 10%
\newline%
\newline%
\newline%
7) 5x - 7 = 33%
\newline%
\newline%
\newline%
8) 6x + 6 = 36%
\newline%
\newline%
\newline%
9) 2x + 4 = 10%
\newline%
\newline%
\newline%
10) x - 3 = 5%
\newline%
\newline%
\newline%
11) 2x - 3 = 5%
\newline%
\newline%
\newline%
12) 2x + 5 = 15%
\newline%
\newline%
\newline%
13) 4x - 8 = 28%
\newline%
\newline%
\newline%
14) 7x - 3 = 60%
\newline%
\newline%
\newline%
15) 2x - 4 = 4%
\newline%
\newline%
\newline%
16) 6x + 9 = 15%
\newline%
\newline%
\newline%
17) 9x - 3 = 51%
\newline%
\newline%
\newline%
18) 7x - 9 = 40%
\newline%
\newline%
\newline%
19) 2x + 4 = 12%
\newline%
\newline%
\newline%
20) x + 5 = 6%
\newline%
\newline%
\newline%
21) 2x - 1 = 15%
\newline%
\newline%
\newline%
22) 8x - 1 = 7%
\newline%
\newline%
\newline%
23) 2x - 9 = -3%
\newline%
\newline%
\newline%
24) 3x + 8 = 14%
\newline%
\newline%
\newline%
25) 5x + 4 = 24%
\newline%
\newline%
\newline%
26) Laura had an unknown number of books. They got 4 times as many books before getting 4 more. In total, they now have 8 books. How many objects did Laura start with?%
\newline%
\newline%
\newline%
27) Laura had an unknown number of pencils. They got 3 times as many pencils before losing 5 more. In total, they now have -2 pencils. How many objects did Laura start with?%
\newline%
\newline%
\newline%
28) Alex had an unknown number of marbles. They got 2 times as many marbles before getting 3 more. In total, they now have 7 marbles. How many objects did Alex start with?%
\newline%
\newline%
\newline%
29) Laura had an unknown number of bottles. They got 9 times as many bottles before getting 4 more. In total, they now have 67 bottles. How many objects did Laura start with?%
\newline%
\newline%
\newline%
30) Sally had an unknown number of pens. They got 2 times as many pens before getting 7 more. In total, they now have 17 pens. How many objects did Sally start with?%
\newline%
\newline%
\newline%
\pagebreak%
\large%
\begin{center}%
\textbf{Solving Basic Algebraic Equations- Worksheet 3}%
\newline%
\newline%
\newline%
\end{center} \normalsize%
1) 9x - 9 = 36%
\newline%
\newline%
\newline%
2) x + 7 = 12%
\newline%
\newline%
\newline%
3) 2x + 1 = 13%
\newline%
\newline%
\newline%
4) 2x - 7 = -1%
\newline%
\newline%
\newline%
5) 7x + 9 = 51%
\newline%
\newline%
\newline%
6) 2x + 6 = 10%
\newline%
\newline%
\newline%
7) 5x - 7 = 33%
\newline%
\newline%
\newline%
8) 6x + 6 = 36%
\newline%
\newline%
\newline%
9) 2x + 4 = 10%
\newline%
\newline%
\newline%
10) x - 3 = 5%
\newline%
\newline%
\newline%
11) 2x - 3 = 5%
\newline%
\newline%
\newline%
12) 2x + 5 = 15%
\newline%
\newline%
\newline%
13) 4x - 8 = 28%
\newline%
\newline%
\newline%
14) 7x - 3 = 60%
\newline%
\newline%
\newline%
15) 2x - 4 = 4%
\newline%
\newline%
\newline%
16) 6x + 9 = 15%
\newline%
\newline%
\newline%
17) 9x - 3 = 51%
\newline%
\newline%
\newline%
18) 7x - 9 = 40%
\newline%
\newline%
\newline%
19) 2x + 4 = 12%
\newline%
\newline%
\newline%
20) x + 5 = 6%
\newline%
\newline%
\newline%
21) 2x - 1 = 15%
\newline%
\newline%
\newline%
22) 8x - 1 = 7%
\newline%
\newline%
\newline%
23) 2x - 9 = -3%
\newline%
\newline%
\newline%
24) 3x + 8 = 14%
\newline%
\newline%
\newline%
25) 5x + 4 = 24%
\newline%
\newline%
\newline%
26) Laura had an unknown number of books. They got 4 times as many books before getting 4 more. In total, they now have 8 books. How many objects did Laura start with?%
\newline%
\newline%
\newline%
27) Laura had an unknown number of pencils. They got 3 times as many pencils before losing 5 more. In total, they now have -2 pencils. How many objects did Laura start with?%
\newline%
\newline%
\newline%
28) Alex had an unknown number of marbles. They got 2 times as many marbles before getting 3 more. In total, they now have 7 marbles. How many objects did Alex start with?%
\newline%
\newline%
\newline%
29) Laura had an unknown number of bottles. They got 9 times as many bottles before getting 4 more. In total, they now have 67 bottles. How many objects did Laura start with?%
\newline%
\newline%
\newline%
30) Sally had an unknown number of pens. They got 2 times as many pens before getting 7 more. In total, they now have 17 pens. How many objects did Sally start with?%
\newline%
\newline%
\newline%
\pagebreak%
\large%
\begin{center}%
\textbf{Solving Basic Algebraic Equations- Worksheet 4}%
\newline%
\newline%
\newline%
\end{center} \normalsize%
1) 9x - 9 = 36%
\newline%
\newline%
\newline%
2) x + 7 = 12%
\newline%
\newline%
\newline%
3) 2x + 1 = 13%
\newline%
\newline%
\newline%
4) 2x - 7 = -1%
\newline%
\newline%
\newline%
5) 7x + 9 = 51%
\newline%
\newline%
\newline%
6) 2x + 6 = 10%
\newline%
\newline%
\newline%
7) 5x - 7 = 33%
\newline%
\newline%
\newline%
8) 6x + 6 = 36%
\newline%
\newline%
\newline%
9) 2x + 4 = 10%
\newline%
\newline%
\newline%
10) x - 3 = 5%
\newline%
\newline%
\newline%
11) 2x - 3 = 5%
\newline%
\newline%
\newline%
12) 2x + 5 = 15%
\newline%
\newline%
\newline%
13) 4x - 8 = 28%
\newline%
\newline%
\newline%
14) 7x - 3 = 60%
\newline%
\newline%
\newline%
15) 2x - 4 = 4%
\newline%
\newline%
\newline%
16) 6x + 9 = 15%
\newline%
\newline%
\newline%
17) 9x - 3 = 51%
\newline%
\newline%
\newline%
18) 7x - 9 = 40%
\newline%
\newline%
\newline%
19) 2x + 4 = 12%
\newline%
\newline%
\newline%
20) x + 5 = 6%
\newline%
\newline%
\newline%
21) 2x - 1 = 15%
\newline%
\newline%
\newline%
22) 8x - 1 = 7%
\newline%
\newline%
\newline%
23) 2x - 9 = -3%
\newline%
\newline%
\newline%
24) 3x + 8 = 14%
\newline%
\newline%
\newline%
25) 5x + 4 = 24%
\newline%
\newline%
\newline%
26) Laura had an unknown number of books. They got 4 times as many books before getting 4 more. In total, they now have 8 books. How many objects did Laura start with?%
\newline%
\newline%
\newline%
27) Laura had an unknown number of pencils. They got 3 times as many pencils before losing 5 more. In total, they now have -2 pencils. How many objects did Laura start with?%
\newline%
\newline%
\newline%
28) Alex had an unknown number of marbles. They got 2 times as many marbles before getting 3 more. In total, they now have 7 marbles. How many objects did Alex start with?%
\newline%
\newline%
\newline%
29) Laura had an unknown number of bottles. They got 9 times as many bottles before getting 4 more. In total, they now have 67 bottles. How many objects did Laura start with?%
\newline%
\newline%
\newline%
30) Sally had an unknown number of pens. They got 2 times as many pens before getting 7 more. In total, they now have 17 pens. How many objects did Sally start with?%
\newline%
\newline%
\newline%
\pagebreak%
\large%
\begin{center}%
\textbf{Solving Inequalities- Worksheet 1}%
\newline%
\newline%
\newline%
\end{center} \normalsize%
1) Solve: -8x - 7 < -31%
\newline%
\newline%
\newline%
2) Solve: -7x + 4 < -31%
\newline%
\newline%
\newline%
3) Solve: 3x - 7 < 14%
\newline%
\newline%
\newline%
4) Solve: -1x + 6 $\leq$ 3%
\newline%
\newline%
\newline%
5) Solve: -2x + 7 > -1%
\newline%
\newline%
\newline%
6) Solve: 5x + 5 $\leq$ 35%
\newline%
\newline%
\newline%
7) Solve: 9x + 6 > 15%
\newline%
\newline%
\newline%
8) Solve: -1x - 8 < -16%
\newline%
\newline%
\newline%
9) Solve: 8x - 6 > 18%
\newline%
\newline%
\newline%
10) Solve: -3x - 3 $\leq$ -27%
\newline%
\newline%
\newline%
11) Solve: 4x - 2 > 22%
\newline%
\newline%
\newline%
12) Solve: -8x + 4 $\geq$ -20%
\newline%
\newline%
\newline%
13) Solve: -9x + 5 > -49%
\newline%
\newline%
\newline%
14) Solve: 6x - 3 $\geq$ 27%
\newline%
\newline%
\newline%
15) Solve: -4x - 1 $\geq$ -37%
\newline%
\newline%
\newline%
16) Solve: -2x + 3 < -1%
\newline%
\newline%
\newline%
17) Solve: -5x - 8 $\leq$ -28%
\newline%
\newline%
\newline%
18) Solve: 3x + 6 $\geq$ 18%
\newline%
\newline%
\newline%
19) Solve: -5x + 8 $\geq$ -12%
\newline%
\newline%
\newline%
20) Solve: -2x + 9 $\geq$ 7%
\newline%
\newline%
\newline%
21) Solve: -1x + 9 < 5%
\newline%
\newline%
\newline%
22) Solve: -2x - 4 > -14%
\newline%
\newline%
\newline%
23) Solve: -4x - 1 $\leq$ -21%
\newline%
\newline%
\newline%
24) Solve: 9x - 9 $\geq$ 63%
\newline%
\newline%
\newline%
25) Solve: -9x + 9 $\leq$ 0%
\newline%
\newline%
\newline%
26) Solve: 8x - 1 $\leq$ 71%
\newline%
\newline%
\newline%
27) Solve: 5x + 7 $\leq$ 42%
\newline%
\newline%
\newline%
28) Solve: 5x + 4 < 19%
\newline%
\newline%
\newline%
29) Solve: -1x - 4 > -8%
\newline%
\newline%
\newline%
30) Solve: 3x - 6 $\geq$ 18%
\newline%
\newline%
\newline%
\pagebreak%
\large%
\begin{center}%
\textbf{Solving Inequalities- Worksheet 2}%
\newline%
\newline%
\newline%
\end{center} \normalsize%
1) Solve: -8x - 7 < -31%
\newline%
\newline%
\newline%
2) Solve: -7x + 4 < -31%
\newline%
\newline%
\newline%
3) Solve: 3x - 7 < 14%
\newline%
\newline%
\newline%
4) Solve: -1x + 6 $\leq$ 3%
\newline%
\newline%
\newline%
5) Solve: -2x + 7 > -1%
\newline%
\newline%
\newline%
6) Solve: 5x + 5 $\leq$ 35%
\newline%
\newline%
\newline%
7) Solve: 9x + 6 > 15%
\newline%
\newline%
\newline%
8) Solve: -1x - 8 < -16%
\newline%
\newline%
\newline%
9) Solve: 8x - 6 > 18%
\newline%
\newline%
\newline%
10) Solve: -3x - 3 $\leq$ -27%
\newline%
\newline%
\newline%
11) Solve: 4x - 2 > 22%
\newline%
\newline%
\newline%
12) Solve: -8x + 4 $\geq$ -20%
\newline%
\newline%
\newline%
13) Solve: -9x + 5 > -49%
\newline%
\newline%
\newline%
14) Solve: 6x - 3 $\geq$ 27%
\newline%
\newline%
\newline%
15) Solve: -4x - 1 $\geq$ -37%
\newline%
\newline%
\newline%
16) Solve: -2x + 3 < -1%
\newline%
\newline%
\newline%
17) Solve: -5x - 8 $\leq$ -28%
\newline%
\newline%
\newline%
18) Solve: 3x + 6 $\geq$ 18%
\newline%
\newline%
\newline%
19) Solve: -5x + 8 $\geq$ -12%
\newline%
\newline%
\newline%
20) Solve: -2x + 9 $\geq$ 7%
\newline%
\newline%
\newline%
21) Solve: -1x + 9 < 5%
\newline%
\newline%
\newline%
22) Solve: -2x - 4 > -14%
\newline%
\newline%
\newline%
23) Solve: -4x - 1 $\leq$ -21%
\newline%
\newline%
\newline%
24) Solve: 9x - 9 $\geq$ 63%
\newline%
\newline%
\newline%
25) Solve: -9x + 9 $\leq$ 0%
\newline%
\newline%
\newline%
26) Solve: 8x - 1 $\leq$ 71%
\newline%
\newline%
\newline%
27) Solve: 5x + 7 $\leq$ 42%
\newline%
\newline%
\newline%
28) Solve: 5x + 4 < 19%
\newline%
\newline%
\newline%
29) Solve: -1x - 4 > -8%
\newline%
\newline%
\newline%
30) Solve: 3x - 6 $\geq$ 18%
\newline%
\newline%
\newline%
\pagebreak%
\large%
\begin{center}%
\textbf{Solving Inequalities- Worksheet 3}%
\newline%
\newline%
\newline%
\end{center} \normalsize%
1) Solve: -8x - 7 < -31%
\newline%
\newline%
\newline%
2) Solve: -7x + 4 < -31%
\newline%
\newline%
\newline%
3) Solve: 3x - 7 < 14%
\newline%
\newline%
\newline%
4) Solve: -1x + 6 $\leq$ 3%
\newline%
\newline%
\newline%
5) Solve: -2x + 7 > -1%
\newline%
\newline%
\newline%
6) Solve: 5x + 5 $\leq$ 35%
\newline%
\newline%
\newline%
7) Solve: 9x + 6 > 15%
\newline%
\newline%
\newline%
8) Solve: -1x - 8 < -16%
\newline%
\newline%
\newline%
9) Solve: 8x - 6 > 18%
\newline%
\newline%
\newline%
10) Solve: -3x - 3 $\leq$ -27%
\newline%
\newline%
\newline%
11) Solve: 4x - 2 > 22%
\newline%
\newline%
\newline%
12) Solve: -8x + 4 $\geq$ -20%
\newline%
\newline%
\newline%
13) Solve: -9x + 5 > -49%
\newline%
\newline%
\newline%
14) Solve: 6x - 3 $\geq$ 27%
\newline%
\newline%
\newline%
15) Solve: -4x - 1 $\geq$ -37%
\newline%
\newline%
\newline%
16) Solve: -2x + 3 < -1%
\newline%
\newline%
\newline%
17) Solve: -5x - 8 $\leq$ -28%
\newline%
\newline%
\newline%
18) Solve: 3x + 6 $\geq$ 18%
\newline%
\newline%
\newline%
19) Solve: -5x + 8 $\geq$ -12%
\newline%
\newline%
\newline%
20) Solve: -2x + 9 $\geq$ 7%
\newline%
\newline%
\newline%
21) Solve: -1x + 9 < 5%
\newline%
\newline%
\newline%
22) Solve: -2x - 4 > -14%
\newline%
\newline%
\newline%
23) Solve: -4x - 1 $\leq$ -21%
\newline%
\newline%
\newline%
24) Solve: 9x - 9 $\geq$ 63%
\newline%
\newline%
\newline%
25) Solve: -9x + 9 $\leq$ 0%
\newline%
\newline%
\newline%
26) Solve: 8x - 1 $\leq$ 71%
\newline%
\newline%
\newline%
27) Solve: 5x + 7 $\leq$ 42%
\newline%
\newline%
\newline%
28) Solve: 5x + 4 < 19%
\newline%
\newline%
\newline%
29) Solve: -1x - 4 > -8%
\newline%
\newline%
\newline%
30) Solve: 3x - 6 $\geq$ 18%
\newline%
\newline%
\newline%
\pagebreak%
\large%
\begin{center}%
\textbf{Solving Inequalities- Worksheet 4}%
\newline%
\newline%
\newline%
\end{center} \normalsize%
1) Solve: -8x - 7 < -31%
\newline%
\newline%
\newline%
2) Solve: -7x + 4 < -31%
\newline%
\newline%
\newline%
3) Solve: 3x - 7 < 14%
\newline%
\newline%
\newline%
4) Solve: -1x + 6 $\leq$ 3%
\newline%
\newline%
\newline%
5) Solve: -2x + 7 > -1%
\newline%
\newline%
\newline%
6) Solve: 5x + 5 $\leq$ 35%
\newline%
\newline%
\newline%
7) Solve: 9x + 6 > 15%
\newline%
\newline%
\newline%
8) Solve: -1x - 8 < -16%
\newline%
\newline%
\newline%
9) Solve: 8x - 6 > 18%
\newline%
\newline%
\newline%
10) Solve: -3x - 3 $\leq$ -27%
\newline%
\newline%
\newline%
11) Solve: 4x - 2 > 22%
\newline%
\newline%
\newline%
12) Solve: -8x + 4 $\geq$ -20%
\newline%
\newline%
\newline%
13) Solve: -9x + 5 > -49%
\newline%
\newline%
\newline%
14) Solve: 6x - 3 $\geq$ 27%
\newline%
\newline%
\newline%
15) Solve: -4x - 1 $\geq$ -37%
\newline%
\newline%
\newline%
16) Solve: -2x + 3 < -1%
\newline%
\newline%
\newline%
17) Solve: -5x - 8 $\leq$ -28%
\newline%
\newline%
\newline%
18) Solve: 3x + 6 $\geq$ 18%
\newline%
\newline%
\newline%
19) Solve: -5x + 8 $\geq$ -12%
\newline%
\newline%
\newline%
20) Solve: -2x + 9 $\geq$ 7%
\newline%
\newline%
\newline%
21) Solve: -1x + 9 < 5%
\newline%
\newline%
\newline%
22) Solve: -2x - 4 > -14%
\newline%
\newline%
\newline%
23) Solve: -4x - 1 $\leq$ -21%
\newline%
\newline%
\newline%
24) Solve: 9x - 9 $\geq$ 63%
\newline%
\newline%
\newline%
25) Solve: -9x + 9 $\leq$ 0%
\newline%
\newline%
\newline%
26) Solve: 8x - 1 $\leq$ 71%
\newline%
\newline%
\newline%
27) Solve: 5x + 7 $\leq$ 42%
\newline%
\newline%
\newline%
28) Solve: 5x + 4 < 19%
\newline%
\newline%
\newline%
29) Solve: -1x - 4 > -8%
\newline%
\newline%
\newline%
30) Solve: 3x - 6 $\geq$ 18%
\newline%
\newline%
\newline%
\pagebreak%
\large%
\begin{center}%
\textbf{Factoring Basic Quadratic Equations- Worksheet 1}%
\newline%
\newline%
\newline%
\end{center} \normalsize%
1) Factorize: $x^2$ - 11x + 28%
\newline%
\newline%
\newline%
2) Factorize: $x^2$ + 4x - 12%
\newline%
\newline%
\newline%
3) Factorize: $x^2$ - 13x + 30%
\newline%
\newline%
\newline%
4) Factorize: $x^2$ - 8x - 9%
\newline%
\newline%
\newline%
5) Factorize: $x^2$ - 13x + 40%
\newline%
\newline%
\newline%
6) Factorize: $x^2$ - 4x - 60%
\newline%
\newline%
\newline%
7) Factorize: $x^2$ + 3x - 70%
\newline%
\newline%
\newline%
8) Factorize: $x^2$ - 49%
\newline%
\newline%
\newline%
9) Factorize: $x^2$ + 3x - 70%
\newline%
\newline%
\newline%
10) Factorize: $x^2$ + 4x - 21%
\newline%
\newline%
\newline%
11) Factorize: $x^2$ + 13x + 36%
\newline%
\newline%
\newline%
12) Factorize: $x^2$ - 3x - 40%
\newline%
\newline%
\newline%
13) Factorize: $x^2$ + 5x - 50%
\newline%
\newline%
\newline%
14) Factorize: $x^2$ - 19x + 90%
\newline%
\newline%
\newline%
15) Factorize: $x^2$ - 10x + 24%
\newline%
\newline%
\newline%
16) Factorize: $x^2$ + 9x + 20%
\newline%
\newline%
\newline%
17) Factorize: $x^2$ - 1x - 12%
\newline%
\newline%
\newline%
18) Factorize: $x^2$ + 2x - 8%
\newline%
\newline%
\newline%
19) Factorize: $x^2$ - 6x + 5%
\newline%
\newline%
\newline%
20) Factorize: $x^2$ - 4x - 32%
\newline%
\newline%
\newline%
21) Factorize: $x^2$ + 2x - 8%
\newline%
\newline%
\newline%
22) Factorize: $x^2$ - 4x + 4%
\newline%
\newline%
\newline%
23) Factorize: $x^2$ - 8x + 15%
\newline%
\newline%
\newline%
24) Factorize: $x^2$ - 8x + 16%
\newline%
\newline%
\newline%
25) Factorize: $x^2$ - 16%
\newline%
\newline%
\newline%
26) Factorize: $x^2$ + 1x - 56%
\newline%
\newline%
\newline%
27) Factorize: $x^2$ + 1x - 42%
\newline%
\newline%
\newline%
28) Factorize: $x^2$ - 5x - 6%
\newline%
\newline%
\newline%
29) Factorize: $x^2$ + 5x - 24%
\newline%
\newline%
\newline%
30) Factorize: $x^2$ - 25%
\newline%
\newline%
\newline%
\pagebreak%
\large%
\begin{center}%
\textbf{Factoring Basic Quadratic Equations- Worksheet 2}%
\newline%
\newline%
\newline%
\end{center} \normalsize%
1) Factorize: $x^2$ - 11x + 28%
\newline%
\newline%
\newline%
2) Factorize: $x^2$ + 4x - 12%
\newline%
\newline%
\newline%
3) Factorize: $x^2$ - 13x + 30%
\newline%
\newline%
\newline%
4) Factorize: $x^2$ - 8x - 9%
\newline%
\newline%
\newline%
5) Factorize: $x^2$ - 13x + 40%
\newline%
\newline%
\newline%
6) Factorize: $x^2$ - 4x - 60%
\newline%
\newline%
\newline%
7) Factorize: $x^2$ + 3x - 70%
\newline%
\newline%
\newline%
8) Factorize: $x^2$ - 49%
\newline%
\newline%
\newline%
9) Factorize: $x^2$ + 3x - 70%
\newline%
\newline%
\newline%
10) Factorize: $x^2$ + 4x - 21%
\newline%
\newline%
\newline%
11) Factorize: $x^2$ + 13x + 36%
\newline%
\newline%
\newline%
12) Factorize: $x^2$ - 3x - 40%
\newline%
\newline%
\newline%
13) Factorize: $x^2$ + 5x - 50%
\newline%
\newline%
\newline%
14) Factorize: $x^2$ - 19x + 90%
\newline%
\newline%
\newline%
15) Factorize: $x^2$ - 10x + 24%
\newline%
\newline%
\newline%
16) Factorize: $x^2$ + 9x + 20%
\newline%
\newline%
\newline%
17) Factorize: $x^2$ - 1x - 12%
\newline%
\newline%
\newline%
18) Factorize: $x^2$ + 2x - 8%
\newline%
\newline%
\newline%
19) Factorize: $x^2$ - 6x + 5%
\newline%
\newline%
\newline%
20) Factorize: $x^2$ - 4x - 32%
\newline%
\newline%
\newline%
21) Factorize: $x^2$ + 2x - 8%
\newline%
\newline%
\newline%
22) Factorize: $x^2$ - 4x + 4%
\newline%
\newline%
\newline%
23) Factorize: $x^2$ - 8x + 15%
\newline%
\newline%
\newline%
24) Factorize: $x^2$ - 8x + 16%
\newline%
\newline%
\newline%
25) Factorize: $x^2$ - 16%
\newline%
\newline%
\newline%
26) Factorize: $x^2$ + 1x - 56%
\newline%
\newline%
\newline%
27) Factorize: $x^2$ + 1x - 42%
\newline%
\newline%
\newline%
28) Factorize: $x^2$ - 5x - 6%
\newline%
\newline%
\newline%
29) Factorize: $x^2$ + 5x - 24%
\newline%
\newline%
\newline%
30) Factorize: $x^2$ - 25%
\newline%
\newline%
\newline%
\pagebreak%
\large%
\begin{center}%
\textbf{Factoring Basic Quadratic Equations- Worksheet 3}%
\newline%
\newline%
\newline%
\end{center} \normalsize%
1) Factorize: $x^2$ - 11x + 28%
\newline%
\newline%
\newline%
2) Factorize: $x^2$ + 4x - 12%
\newline%
\newline%
\newline%
3) Factorize: $x^2$ - 13x + 30%
\newline%
\newline%
\newline%
4) Factorize: $x^2$ - 8x - 9%
\newline%
\newline%
\newline%
5) Factorize: $x^2$ - 13x + 40%
\newline%
\newline%
\newline%
6) Factorize: $x^2$ - 4x - 60%
\newline%
\newline%
\newline%
7) Factorize: $x^2$ + 3x - 70%
\newline%
\newline%
\newline%
8) Factorize: $x^2$ - 49%
\newline%
\newline%
\newline%
9) Factorize: $x^2$ + 3x - 70%
\newline%
\newline%
\newline%
10) Factorize: $x^2$ + 4x - 21%
\newline%
\newline%
\newline%
11) Factorize: $x^2$ + 13x + 36%
\newline%
\newline%
\newline%
12) Factorize: $x^2$ - 3x - 40%
\newline%
\newline%
\newline%
13) Factorize: $x^2$ + 5x - 50%
\newline%
\newline%
\newline%
14) Factorize: $x^2$ - 19x + 90%
\newline%
\newline%
\newline%
15) Factorize: $x^2$ - 10x + 24%
\newline%
\newline%
\newline%
16) Factorize: $x^2$ + 9x + 20%
\newline%
\newline%
\newline%
17) Factorize: $x^2$ - 1x - 12%
\newline%
\newline%
\newline%
18) Factorize: $x^2$ + 2x - 8%
\newline%
\newline%
\newline%
19) Factorize: $x^2$ - 6x + 5%
\newline%
\newline%
\newline%
20) Factorize: $x^2$ - 4x - 32%
\newline%
\newline%
\newline%
21) Factorize: $x^2$ + 2x - 8%
\newline%
\newline%
\newline%
22) Factorize: $x^2$ - 4x + 4%
\newline%
\newline%
\newline%
23) Factorize: $x^2$ - 8x + 15%
\newline%
\newline%
\newline%
24) Factorize: $x^2$ - 8x + 16%
\newline%
\newline%
\newline%
25) Factorize: $x^2$ - 16%
\newline%
\newline%
\newline%
26) Factorize: $x^2$ + 1x - 56%
\newline%
\newline%
\newline%
27) Factorize: $x^2$ + 1x - 42%
\newline%
\newline%
\newline%
28) Factorize: $x^2$ - 5x - 6%
\newline%
\newline%
\newline%
29) Factorize: $x^2$ + 5x - 24%
\newline%
\newline%
\newline%
30) Factorize: $x^2$ - 25%
\newline%
\newline%
\newline%
\pagebreak%
\large%
\begin{center}%
\textbf{Factoring Basic Quadratic Equations- Worksheet 4}%
\newline%
\newline%
\newline%
\end{center} \normalsize%
1) Factorize: $x^2$ - 11x + 28%
\newline%
\newline%
\newline%
2) Factorize: $x^2$ + 4x - 12%
\newline%
\newline%
\newline%
3) Factorize: $x^2$ - 13x + 30%
\newline%
\newline%
\newline%
4) Factorize: $x^2$ - 8x - 9%
\newline%
\newline%
\newline%
5) Factorize: $x^2$ - 13x + 40%
\newline%
\newline%
\newline%
6) Factorize: $x^2$ - 4x - 60%
\newline%
\newline%
\newline%
7) Factorize: $x^2$ + 3x - 70%
\newline%
\newline%
\newline%
8) Factorize: $x^2$ - 49%
\newline%
\newline%
\newline%
9) Factorize: $x^2$ + 3x - 70%
\newline%
\newline%
\newline%
10) Factorize: $x^2$ + 4x - 21%
\newline%
\newline%
\newline%
11) Factorize: $x^2$ + 13x + 36%
\newline%
\newline%
\newline%
12) Factorize: $x^2$ - 3x - 40%
\newline%
\newline%
\newline%
13) Factorize: $x^2$ + 5x - 50%
\newline%
\newline%
\newline%
14) Factorize: $x^2$ - 19x + 90%
\newline%
\newline%
\newline%
15) Factorize: $x^2$ - 10x + 24%
\newline%
\newline%
\newline%
16) Factorize: $x^2$ + 9x + 20%
\newline%
\newline%
\newline%
17) Factorize: $x^2$ - 1x - 12%
\newline%
\newline%
\newline%
18) Factorize: $x^2$ + 2x - 8%
\newline%
\newline%
\newline%
19) Factorize: $x^2$ - 6x + 5%
\newline%
\newline%
\newline%
20) Factorize: $x^2$ - 4x - 32%
\newline%
\newline%
\newline%
21) Factorize: $x^2$ + 2x - 8%
\newline%
\newline%
\newline%
22) Factorize: $x^2$ - 4x + 4%
\newline%
\newline%
\newline%
23) Factorize: $x^2$ - 8x + 15%
\newline%
\newline%
\newline%
24) Factorize: $x^2$ - 8x + 16%
\newline%
\newline%
\newline%
25) Factorize: $x^2$ - 16%
\newline%
\newline%
\newline%
26) Factorize: $x^2$ + 1x - 56%
\newline%
\newline%
\newline%
27) Factorize: $x^2$ + 1x - 42%
\newline%
\newline%
\newline%
28) Factorize: $x^2$ - 5x - 6%
\newline%
\newline%
\newline%
29) Factorize: $x^2$ + 5x - 24%
\newline%
\newline%
\newline%
30) Factorize: $x^2$ - 25%
\newline%
\newline%
\newline%
\pagebreak%
\large%
\begin{center}%
\textbf{Factoring Advanced Quadratic Equations- Worksheet 1}%
\newline%
\newline%
\newline%
\end{center} \normalsize%
1) Factorize: $24x^2$ - 56x + 16%
\newline%
\newline%
\newline%
2) Factorize: $-4x^2$ + 10x + 6%
\newline%
\newline%
\newline%
3) Factorize: $-5x^2$ - 45x - 40%
\newline%
\newline%
\newline%
4) Factorize: $-10x^2$ + 20x + 80%
\newline%
\newline%
\newline%
5) Factorize: $-30x^2$ - 70x - 40%
\newline%
\newline%
\newline%
6) Factorize: $15x^2$ - 12x - 3%
\newline%
\newline%
\newline%
7) Factorize: $-6x^2$ - 10x + 56%
\newline%
\newline%
\newline%
8) Factorize: $6x^2$ - 1x - 1%
\newline%
\newline%
\newline%
9) Factorize: $-6x^2$ + 20x - 16%
\newline%
\newline%
\newline%
10) Factorize: $-30x^2$ - 51x - 9%
\newline%
\newline%
\newline%
11) Factorize: $6x^2$ + 13x - 5%
\newline%
\newline%
\newline%
12) Factorize: $-3x^2$ - 16x + 64%
\newline%
\newline%
\newline%
13) Factorize: $-18x^2$ + 9x + 14%
\newline%
\newline%
\newline%
14) Factorize: $6x^2$ - 19x + 14%
\newline%
\newline%
\newline%
15) Factorize: $15x^2$ - 49x + 40%
\newline%
\newline%
\newline%
16) Factorize: $-3x^2$ + 15x - 18%
\newline%
\newline%
\newline%
17) Factorize: $15x^2$ + 6x - 21%
\newline%
\newline%
\newline%
18) Factorize: $15x^2$ - 11x + 2%
\newline%
\newline%
\newline%
19) Factorize: $10x^2$ + 51x + 27%
\newline%
\newline%
\newline%
20) Factorize: $-24x^2$ + 74x - 45%
\newline%
\newline%
\newline%
21) Factorize: $12x^2$ + 48x + 21%
\newline%
\newline%
\newline%
22) Factorize: $-10x^2$ + 28x - 18%
\newline%
\newline%
\newline%
23) Factorize: $3x^2$ + 16x + 16%
\newline%
\newline%
\newline%
24) Factorize: $-2x^2$ - 26x - 72%
\newline%
\newline%
\newline%
25) Factorize: $12x^2$ + 34x - 6%
\newline%
\newline%
\newline%
26) Factorize: $-3x^2$ + 2x + 21%
\newline%
\newline%
\newline%
27) Factorize: $3x^2$ - 18x + 15%
\newline%
\newline%
\newline%
28) Factorize: $12x^2$ - 80x + 100%
\newline%
\newline%
\newline%
29) Factorize: $12x^2$ + 2x - 24%
\newline%
\newline%
\newline%
30) Factorize: $-4x^2$ + 17x + 42%
\newline%
\newline%
\newline%
\pagebreak%
\large%
\begin{center}%
\textbf{Factoring Advanced Quadratic Equations- Worksheet 2}%
\newline%
\newline%
\newline%
\end{center} \normalsize%
1) Factorize: $24x^2$ - 56x + 16%
\newline%
\newline%
\newline%
2) Factorize: $-4x^2$ + 10x + 6%
\newline%
\newline%
\newline%
3) Factorize: $-5x^2$ - 45x - 40%
\newline%
\newline%
\newline%
4) Factorize: $-10x^2$ + 20x + 80%
\newline%
\newline%
\newline%
5) Factorize: $-30x^2$ - 70x - 40%
\newline%
\newline%
\newline%
6) Factorize: $15x^2$ - 12x - 3%
\newline%
\newline%
\newline%
7) Factorize: $-6x^2$ - 10x + 56%
\newline%
\newline%
\newline%
8) Factorize: $6x^2$ - 1x - 1%
\newline%
\newline%
\newline%
9) Factorize: $-6x^2$ + 20x - 16%
\newline%
\newline%
\newline%
10) Factorize: $-30x^2$ - 51x - 9%
\newline%
\newline%
\newline%
11) Factorize: $6x^2$ + 13x - 5%
\newline%
\newline%
\newline%
12) Factorize: $-3x^2$ - 16x + 64%
\newline%
\newline%
\newline%
13) Factorize: $-18x^2$ + 9x + 14%
\newline%
\newline%
\newline%
14) Factorize: $6x^2$ - 19x + 14%
\newline%
\newline%
\newline%
15) Factorize: $15x^2$ - 49x + 40%
\newline%
\newline%
\newline%
16) Factorize: $-3x^2$ + 15x - 18%
\newline%
\newline%
\newline%
17) Factorize: $15x^2$ + 6x - 21%
\newline%
\newline%
\newline%
18) Factorize: $15x^2$ - 11x + 2%
\newline%
\newline%
\newline%
19) Factorize: $10x^2$ + 51x + 27%
\newline%
\newline%
\newline%
20) Factorize: $-24x^2$ + 74x - 45%
\newline%
\newline%
\newline%
21) Factorize: $12x^2$ + 48x + 21%
\newline%
\newline%
\newline%
22) Factorize: $-10x^2$ + 28x - 18%
\newline%
\newline%
\newline%
23) Factorize: $3x^2$ + 16x + 16%
\newline%
\newline%
\newline%
24) Factorize: $-2x^2$ - 26x - 72%
\newline%
\newline%
\newline%
25) Factorize: $12x^2$ + 34x - 6%
\newline%
\newline%
\newline%
26) Factorize: $-3x^2$ + 2x + 21%
\newline%
\newline%
\newline%
27) Factorize: $3x^2$ - 18x + 15%
\newline%
\newline%
\newline%
28) Factorize: $12x^2$ - 80x + 100%
\newline%
\newline%
\newline%
29) Factorize: $12x^2$ + 2x - 24%
\newline%
\newline%
\newline%
30) Factorize: $-4x^2$ + 17x + 42%
\newline%
\newline%
\newline%
\pagebreak%
\large%
\begin{center}%
\textbf{Factoring Advanced Quadratic Equations- Worksheet 3}%
\newline%
\newline%
\newline%
\end{center} \normalsize%
1) Factorize: $24x^2$ - 56x + 16%
\newline%
\newline%
\newline%
2) Factorize: $-4x^2$ + 10x + 6%
\newline%
\newline%
\newline%
3) Factorize: $-5x^2$ - 45x - 40%
\newline%
\newline%
\newline%
4) Factorize: $-10x^2$ + 20x + 80%
\newline%
\newline%
\newline%
5) Factorize: $-30x^2$ - 70x - 40%
\newline%
\newline%
\newline%
6) Factorize: $15x^2$ - 12x - 3%
\newline%
\newline%
\newline%
7) Factorize: $-6x^2$ - 10x + 56%
\newline%
\newline%
\newline%
8) Factorize: $6x^2$ - 1x - 1%
\newline%
\newline%
\newline%
9) Factorize: $-6x^2$ + 20x - 16%
\newline%
\newline%
\newline%
10) Factorize: $-30x^2$ - 51x - 9%
\newline%
\newline%
\newline%
11) Factorize: $6x^2$ + 13x - 5%
\newline%
\newline%
\newline%
12) Factorize: $-3x^2$ - 16x + 64%
\newline%
\newline%
\newline%
13) Factorize: $-18x^2$ + 9x + 14%
\newline%
\newline%
\newline%
14) Factorize: $6x^2$ - 19x + 14%
\newline%
\newline%
\newline%
15) Factorize: $15x^2$ - 49x + 40%
\newline%
\newline%
\newline%
16) Factorize: $-3x^2$ + 15x - 18%
\newline%
\newline%
\newline%
17) Factorize: $15x^2$ + 6x - 21%
\newline%
\newline%
\newline%
18) Factorize: $15x^2$ - 11x + 2%
\newline%
\newline%
\newline%
19) Factorize: $10x^2$ + 51x + 27%
\newline%
\newline%
\newline%
20) Factorize: $-24x^2$ + 74x - 45%
\newline%
\newline%
\newline%
21) Factorize: $12x^2$ + 48x + 21%
\newline%
\newline%
\newline%
22) Factorize: $-10x^2$ + 28x - 18%
\newline%
\newline%
\newline%
23) Factorize: $3x^2$ + 16x + 16%
\newline%
\newline%
\newline%
24) Factorize: $-2x^2$ - 26x - 72%
\newline%
\newline%
\newline%
25) Factorize: $12x^2$ + 34x - 6%
\newline%
\newline%
\newline%
26) Factorize: $-3x^2$ + 2x + 21%
\newline%
\newline%
\newline%
27) Factorize: $3x^2$ - 18x + 15%
\newline%
\newline%
\newline%
28) Factorize: $12x^2$ - 80x + 100%
\newline%
\newline%
\newline%
29) Factorize: $12x^2$ + 2x - 24%
\newline%
\newline%
\newline%
30) Factorize: $-4x^2$ + 17x + 42%
\newline%
\newline%
\newline%
\pagebreak%
\large%
\begin{center}%
\textbf{Factoring Advanced Quadratic Equations- Worksheet 4}%
\newline%
\newline%
\newline%
\end{center} \normalsize%
1) Factorize: $24x^2$ - 56x + 16%
\newline%
\newline%
\newline%
2) Factorize: $-4x^2$ + 10x + 6%
\newline%
\newline%
\newline%
3) Factorize: $-5x^2$ - 45x - 40%
\newline%
\newline%
\newline%
4) Factorize: $-10x^2$ + 20x + 80%
\newline%
\newline%
\newline%
5) Factorize: $-30x^2$ - 70x - 40%
\newline%
\newline%
\newline%
6) Factorize: $15x^2$ - 12x - 3%
\newline%
\newline%
\newline%
7) Factorize: $-6x^2$ - 10x + 56%
\newline%
\newline%
\newline%
8) Factorize: $6x^2$ - 1x - 1%
\newline%
\newline%
\newline%
9) Factorize: $-6x^2$ + 20x - 16%
\newline%
\newline%
\newline%
10) Factorize: $-30x^2$ - 51x - 9%
\newline%
\newline%
\newline%
11) Factorize: $6x^2$ + 13x - 5%
\newline%
\newline%
\newline%
12) Factorize: $-3x^2$ - 16x + 64%
\newline%
\newline%
\newline%
13) Factorize: $-18x^2$ + 9x + 14%
\newline%
\newline%
\newline%
14) Factorize: $6x^2$ - 19x + 14%
\newline%
\newline%
\newline%
15) Factorize: $15x^2$ - 49x + 40%
\newline%
\newline%
\newline%
16) Factorize: $-3x^2$ + 15x - 18%
\newline%
\newline%
\newline%
17) Factorize: $15x^2$ + 6x - 21%
\newline%
\newline%
\newline%
18) Factorize: $15x^2$ - 11x + 2%
\newline%
\newline%
\newline%
19) Factorize: $10x^2$ + 51x + 27%
\newline%
\newline%
\newline%
20) Factorize: $-24x^2$ + 74x - 45%
\newline%
\newline%
\newline%
21) Factorize: $12x^2$ + 48x + 21%
\newline%
\newline%
\newline%
22) Factorize: $-10x^2$ + 28x - 18%
\newline%
\newline%
\newline%
23) Factorize: $3x^2$ + 16x + 16%
\newline%
\newline%
\newline%
24) Factorize: $-2x^2$ - 26x - 72%
\newline%
\newline%
\newline%
25) Factorize: $12x^2$ + 34x - 6%
\newline%
\newline%
\newline%
26) Factorize: $-3x^2$ + 2x + 21%
\newline%
\newline%
\newline%
27) Factorize: $3x^2$ - 18x + 15%
\newline%
\newline%
\newline%
28) Factorize: $12x^2$ - 80x + 100%
\newline%
\newline%
\newline%
29) Factorize: $12x^2$ + 2x - 24%
\newline%
\newline%
\newline%
30) Factorize: $-4x^2$ + 17x + 42%
\newline%
\newline%
\newline%
\pagebreak%
\large%
\begin{center}%
\textbf{Finding the Equation of a Line- Worksheet 1}%
\newline%
\newline%
\newline%
\end{center} \normalsize%
1) Find the equation of the line with the points (0, 0), (0.0, 0). Round your slope to the nearest whole number:

\begin{tikzpicture} 
\begin{axis}[xmin=-10, xmax=10, ymin=-10, ymax=10, axis x line=middle, axis y line=middle]\addplot[domain=-10:10]{-5*x+0};
\addplot[mark=*] coordinates {(0,0)};
\addplot[mark=*] coordinates {(0.0,0)};
\end{axis}
\end{tikzpicture}%
\newline%
\newline%
\newline%
2) Find the equation of the line with the points (0, -5), (-0.62, 0). Round your slope to the nearest whole number:

\begin{tikzpicture} 
\begin{axis}[xmin=-10, xmax=10, ymin=-10, ymax=10, axis x line=middle, axis y line=middle]\addplot[domain=-10:10]{-8*x+-5};
\addplot[mark=*] coordinates {(0,-5)};
\addplot[mark=*] coordinates {(-0.62,0)};
\end{axis}
\end{tikzpicture}%
\newline%
\newline%
\newline%
3) Find the equation of the line with the points (0, 10), (1.11, 0). Round your slope to the nearest whole number:

\begin{tikzpicture} 
\begin{axis}[xmin=-10, xmax=10, ymin=-10, ymax=10, axis x line=middle, axis y line=middle]\addplot[domain=-10:10]{-9*x+10};
\addplot[mark=*] coordinates {(0,10)};
\addplot[mark=*] coordinates {(1.11,0)};
\end{axis}
\end{tikzpicture}%
\newline%
\newline%
\newline%
4) Find the equation of the line with the points (0, 4), (0.4, 0). Round your slope to the nearest whole number:

\begin{tikzpicture} 
\begin{axis}[xmin=-10, xmax=10, ymin=-10, ymax=10, axis x line=middle, axis y line=middle]\addplot[domain=-10:10]{-10*x+4};
\addplot[mark=*] coordinates {(0,4)};
\addplot[mark=*] coordinates {(0.4,0)};
\end{axis}
\end{tikzpicture}%
\newline%
\newline%
\newline%
5) Find the equation of the line with the points (0, 4), (-4.0, 0). Round your slope to the nearest whole number:

\begin{tikzpicture} 
\begin{axis}[xmin=-10, xmax=10, ymin=-10, ymax=10, axis x line=middle, axis y line=middle]\addplot[domain=-10:10]{1*x+4};
\addplot[mark=*] coordinates {(0,4)};
\addplot[mark=*] coordinates {(-4.0,0)};
\end{axis}
\end{tikzpicture}%
\newline%
\newline%
\newline%
6) Find the equation of the line with the points (0, 5), (1.67, 0). Round your slope to the nearest whole number:

\begin{tikzpicture} 
\begin{axis}[xmin=-10, xmax=10, ymin=-10, ymax=10, axis x line=middle, axis y line=middle]\addplot[domain=-10:10]{-3*x+5};
\addplot[mark=*] coordinates {(0,5)};
\addplot[mark=*] coordinates {(1.67,0)};
\end{axis}
\end{tikzpicture}%
\newline%
\newline%
\newline%
7) Find the equation of the line with the points (0, 10), (-2.0, 0). Round your slope to the nearest whole number:

\begin{tikzpicture} 
\begin{axis}[xmin=-10, xmax=10, ymin=-10, ymax=10, axis x line=middle, axis y line=middle]\addplot[domain=-10:10]{5*x+10};
\addplot[mark=*] coordinates {(0,10)};
\addplot[mark=*] coordinates {(-2.0,0)};
\end{axis}
\end{tikzpicture}%
\newline%
\newline%
\newline%
8) Find the equation of the line with the points (0, 9), (2.25, 0). Round your slope to the nearest whole number:

\begin{tikzpicture} 
\begin{axis}[xmin=-10, xmax=10, ymin=-10, ymax=10, axis x line=middle, axis y line=middle]\addplot[domain=-10:10]{-4*x+9};
\addplot[mark=*] coordinates {(0,9)};
\addplot[mark=*] coordinates {(2.25,0)};
\end{axis}
\end{tikzpicture}%
\newline%
\newline%
\newline%
9) Find the equation of the line with the points (0, 1), (0.33, 0). Round your slope to the nearest whole number:

\begin{tikzpicture} 
\begin{axis}[xmin=-10, xmax=10, ymin=-10, ymax=10, axis x line=middle, axis y line=middle]\addplot[domain=-10:10]{-3*x+1};
\addplot[mark=*] coordinates {(0,1)};
\addplot[mark=*] coordinates {(0.33,0)};
\end{axis}
\end{tikzpicture}%
\newline%
\newline%
\newline%
10) Find the equation of the line with the points (0, -1), (-0.12, 0). Round your slope to the nearest whole number:

\begin{tikzpicture} 
\begin{axis}[xmin=-10, xmax=10, ymin=-10, ymax=10, axis x line=middle, axis y line=middle]\addplot[domain=-10:10]{-8*x+-1};
\addplot[mark=*] coordinates {(0,-1)};
\addplot[mark=*] coordinates {(-0.12,0)};
\end{axis}
\end{tikzpicture}%
\newline%
\newline%
\newline%
\pagebreak%
\large%
\begin{center}%
\textbf{Finding the Equation of a Line- Worksheet 2}%
\newline%
\newline%
\newline%
\end{center} \normalsize%
1) Find the equation of the line with the points (0, 0), (0.0, 0). Round your slope to the nearest whole number:

\begin{tikzpicture} 
\begin{axis}[xmin=-10, xmax=10, ymin=-10, ymax=10, axis x line=middle, axis y line=middle]\addplot[domain=-10:10]{-5*x+0};
\addplot[mark=*] coordinates {(0,0)};
\addplot[mark=*] coordinates {(0.0,0)};
\end{axis}
\end{tikzpicture}%
\newline%
\newline%
\newline%
2) Find the equation of the line with the points (0, -5), (-0.62, 0). Round your slope to the nearest whole number:

\begin{tikzpicture} 
\begin{axis}[xmin=-10, xmax=10, ymin=-10, ymax=10, axis x line=middle, axis y line=middle]\addplot[domain=-10:10]{-8*x+-5};
\addplot[mark=*] coordinates {(0,-5)};
\addplot[mark=*] coordinates {(-0.62,0)};
\end{axis}
\end{tikzpicture}%
\newline%
\newline%
\newline%
3) Find the equation of the line with the points (0, 10), (1.11, 0). Round your slope to the nearest whole number:

\begin{tikzpicture} 
\begin{axis}[xmin=-10, xmax=10, ymin=-10, ymax=10, axis x line=middle, axis y line=middle]\addplot[domain=-10:10]{-9*x+10};
\addplot[mark=*] coordinates {(0,10)};
\addplot[mark=*] coordinates {(1.11,0)};
\end{axis}
\end{tikzpicture}%
\newline%
\newline%
\newline%
4) Find the equation of the line with the points (0, 4), (0.4, 0). Round your slope to the nearest whole number:

\begin{tikzpicture} 
\begin{axis}[xmin=-10, xmax=10, ymin=-10, ymax=10, axis x line=middle, axis y line=middle]\addplot[domain=-10:10]{-10*x+4};
\addplot[mark=*] coordinates {(0,4)};
\addplot[mark=*] coordinates {(0.4,0)};
\end{axis}
\end{tikzpicture}%
\newline%
\newline%
\newline%
5) Find the equation of the line with the points (0, 4), (-4.0, 0). Round your slope to the nearest whole number:

\begin{tikzpicture} 
\begin{axis}[xmin=-10, xmax=10, ymin=-10, ymax=10, axis x line=middle, axis y line=middle]\addplot[domain=-10:10]{1*x+4};
\addplot[mark=*] coordinates {(0,4)};
\addplot[mark=*] coordinates {(-4.0,0)};
\end{axis}
\end{tikzpicture}%
\newline%
\newline%
\newline%
6) Find the equation of the line with the points (0, 5), (1.67, 0). Round your slope to the nearest whole number:

\begin{tikzpicture} 
\begin{axis}[xmin=-10, xmax=10, ymin=-10, ymax=10, axis x line=middle, axis y line=middle]\addplot[domain=-10:10]{-3*x+5};
\addplot[mark=*] coordinates {(0,5)};
\addplot[mark=*] coordinates {(1.67,0)};
\end{axis}
\end{tikzpicture}%
\newline%
\newline%
\newline%
7) Find the equation of the line with the points (0, 10), (-2.0, 0). Round your slope to the nearest whole number:

\begin{tikzpicture} 
\begin{axis}[xmin=-10, xmax=10, ymin=-10, ymax=10, axis x line=middle, axis y line=middle]\addplot[domain=-10:10]{5*x+10};
\addplot[mark=*] coordinates {(0,10)};
\addplot[mark=*] coordinates {(-2.0,0)};
\end{axis}
\end{tikzpicture}%
\newline%
\newline%
\newline%
8) Find the equation of the line with the points (0, 9), (2.25, 0). Round your slope to the nearest whole number:

\begin{tikzpicture} 
\begin{axis}[xmin=-10, xmax=10, ymin=-10, ymax=10, axis x line=middle, axis y line=middle]\addplot[domain=-10:10]{-4*x+9};
\addplot[mark=*] coordinates {(0,9)};
\addplot[mark=*] coordinates {(2.25,0)};
\end{axis}
\end{tikzpicture}%
\newline%
\newline%
\newline%
9) Find the equation of the line with the points (0, 1), (0.33, 0). Round your slope to the nearest whole number:

\begin{tikzpicture} 
\begin{axis}[xmin=-10, xmax=10, ymin=-10, ymax=10, axis x line=middle, axis y line=middle]\addplot[domain=-10:10]{-3*x+1};
\addplot[mark=*] coordinates {(0,1)};
\addplot[mark=*] coordinates {(0.33,0)};
\end{axis}
\end{tikzpicture}%
\newline%
\newline%
\newline%
10) Find the equation of the line with the points (0, -1), (-0.12, 0). Round your slope to the nearest whole number:

\begin{tikzpicture} 
\begin{axis}[xmin=-10, xmax=10, ymin=-10, ymax=10, axis x line=middle, axis y line=middle]\addplot[domain=-10:10]{-8*x+-1};
\addplot[mark=*] coordinates {(0,-1)};
\addplot[mark=*] coordinates {(-0.12,0)};
\end{axis}
\end{tikzpicture}%
\newline%
\newline%
\newline%
\pagebreak%
\large%
\begin{center}%
\textbf{Finding the Equation of a Line- Worksheet 3}%
\newline%
\newline%
\newline%
\end{center} \normalsize%
1) Find the equation of the line with the points (0, 0), (0.0, 0). Round your slope to the nearest whole number:

\begin{tikzpicture} 
\begin{axis}[xmin=-10, xmax=10, ymin=-10, ymax=10, axis x line=middle, axis y line=middle]\addplot[domain=-10:10]{-5*x+0};
\addplot[mark=*] coordinates {(0,0)};
\addplot[mark=*] coordinates {(0.0,0)};
\end{axis}
\end{tikzpicture}%
\newline%
\newline%
\newline%
2) Find the equation of the line with the points (0, -5), (-0.62, 0). Round your slope to the nearest whole number:

\begin{tikzpicture} 
\begin{axis}[xmin=-10, xmax=10, ymin=-10, ymax=10, axis x line=middle, axis y line=middle]\addplot[domain=-10:10]{-8*x+-5};
\addplot[mark=*] coordinates {(0,-5)};
\addplot[mark=*] coordinates {(-0.62,0)};
\end{axis}
\end{tikzpicture}%
\newline%
\newline%
\newline%
3) Find the equation of the line with the points (0, 10), (1.11, 0). Round your slope to the nearest whole number:

\begin{tikzpicture} 
\begin{axis}[xmin=-10, xmax=10, ymin=-10, ymax=10, axis x line=middle, axis y line=middle]\addplot[domain=-10:10]{-9*x+10};
\addplot[mark=*] coordinates {(0,10)};
\addplot[mark=*] coordinates {(1.11,0)};
\end{axis}
\end{tikzpicture}%
\newline%
\newline%
\newline%
4) Find the equation of the line with the points (0, 4), (0.4, 0). Round your slope to the nearest whole number:

\begin{tikzpicture} 
\begin{axis}[xmin=-10, xmax=10, ymin=-10, ymax=10, axis x line=middle, axis y line=middle]\addplot[domain=-10:10]{-10*x+4};
\addplot[mark=*] coordinates {(0,4)};
\addplot[mark=*] coordinates {(0.4,0)};
\end{axis}
\end{tikzpicture}%
\newline%
\newline%
\newline%
5) Find the equation of the line with the points (0, 4), (-4.0, 0). Round your slope to the nearest whole number:

\begin{tikzpicture} 
\begin{axis}[xmin=-10, xmax=10, ymin=-10, ymax=10, axis x line=middle, axis y line=middle]\addplot[domain=-10:10]{1*x+4};
\addplot[mark=*] coordinates {(0,4)};
\addplot[mark=*] coordinates {(-4.0,0)};
\end{axis}
\end{tikzpicture}%
\newline%
\newline%
\newline%
6) Find the equation of the line with the points (0, 5), (1.67, 0). Round your slope to the nearest whole number:

\begin{tikzpicture} 
\begin{axis}[xmin=-10, xmax=10, ymin=-10, ymax=10, axis x line=middle, axis y line=middle]\addplot[domain=-10:10]{-3*x+5};
\addplot[mark=*] coordinates {(0,5)};
\addplot[mark=*] coordinates {(1.67,0)};
\end{axis}
\end{tikzpicture}%
\newline%
\newline%
\newline%
7) Find the equation of the line with the points (0, 10), (-2.0, 0). Round your slope to the nearest whole number:

\begin{tikzpicture} 
\begin{axis}[xmin=-10, xmax=10, ymin=-10, ymax=10, axis x line=middle, axis y line=middle]\addplot[domain=-10:10]{5*x+10};
\addplot[mark=*] coordinates {(0,10)};
\addplot[mark=*] coordinates {(-2.0,0)};
\end{axis}
\end{tikzpicture}%
\newline%
\newline%
\newline%
8) Find the equation of the line with the points (0, 9), (2.25, 0). Round your slope to the nearest whole number:

\begin{tikzpicture} 
\begin{axis}[xmin=-10, xmax=10, ymin=-10, ymax=10, axis x line=middle, axis y line=middle]\addplot[domain=-10:10]{-4*x+9};
\addplot[mark=*] coordinates {(0,9)};
\addplot[mark=*] coordinates {(2.25,0)};
\end{axis}
\end{tikzpicture}%
\newline%
\newline%
\newline%
9) Find the equation of the line with the points (0, 1), (0.33, 0). Round your slope to the nearest whole number:

\begin{tikzpicture} 
\begin{axis}[xmin=-10, xmax=10, ymin=-10, ymax=10, axis x line=middle, axis y line=middle]\addplot[domain=-10:10]{-3*x+1};
\addplot[mark=*] coordinates {(0,1)};
\addplot[mark=*] coordinates {(0.33,0)};
\end{axis}
\end{tikzpicture}%
\newline%
\newline%
\newline%
10) Find the equation of the line with the points (0, -1), (-0.12, 0). Round your slope to the nearest whole number:

\begin{tikzpicture} 
\begin{axis}[xmin=-10, xmax=10, ymin=-10, ymax=10, axis x line=middle, axis y line=middle]\addplot[domain=-10:10]{-8*x+-1};
\addplot[mark=*] coordinates {(0,-1)};
\addplot[mark=*] coordinates {(-0.12,0)};
\end{axis}
\end{tikzpicture}%
\newline%
\newline%
\newline%
\pagebreak%
\large%
\begin{center}%
\textbf{Finding the Equation of a Line- Worksheet 4}%
\newline%
\newline%
\newline%
\end{center} \normalsize%
1) Find the equation of the line with the points (0, 0), (0.0, 0). Round your slope to the nearest whole number:

\begin{tikzpicture} 
\begin{axis}[xmin=-10, xmax=10, ymin=-10, ymax=10, axis x line=middle, axis y line=middle]\addplot[domain=-10:10]{-5*x+0};
\addplot[mark=*] coordinates {(0,0)};
\addplot[mark=*] coordinates {(0.0,0)};
\end{axis}
\end{tikzpicture}%
\newline%
\newline%
\newline%
2) Find the equation of the line with the points (0, -5), (-0.62, 0). Round your slope to the nearest whole number:

\begin{tikzpicture} 
\begin{axis}[xmin=-10, xmax=10, ymin=-10, ymax=10, axis x line=middle, axis y line=middle]\addplot[domain=-10:10]{-8*x+-5};
\addplot[mark=*] coordinates {(0,-5)};
\addplot[mark=*] coordinates {(-0.62,0)};
\end{axis}
\end{tikzpicture}%
\newline%
\newline%
\newline%
3) Find the equation of the line with the points (0, 10), (1.11, 0). Round your slope to the nearest whole number:

\begin{tikzpicture} 
\begin{axis}[xmin=-10, xmax=10, ymin=-10, ymax=10, axis x line=middle, axis y line=middle]\addplot[domain=-10:10]{-9*x+10};
\addplot[mark=*] coordinates {(0,10)};
\addplot[mark=*] coordinates {(1.11,0)};
\end{axis}
\end{tikzpicture}%
\newline%
\newline%
\newline%
4) Find the equation of the line with the points (0, 4), (0.4, 0). Round your slope to the nearest whole number:

\begin{tikzpicture} 
\begin{axis}[xmin=-10, xmax=10, ymin=-10, ymax=10, axis x line=middle, axis y line=middle]\addplot[domain=-10:10]{-10*x+4};
\addplot[mark=*] coordinates {(0,4)};
\addplot[mark=*] coordinates {(0.4,0)};
\end{axis}
\end{tikzpicture}%
\newline%
\newline%
\newline%
5) Find the equation of the line with the points (0, 4), (-4.0, 0). Round your slope to the nearest whole number:

\begin{tikzpicture} 
\begin{axis}[xmin=-10, xmax=10, ymin=-10, ymax=10, axis x line=middle, axis y line=middle]\addplot[domain=-10:10]{1*x+4};
\addplot[mark=*] coordinates {(0,4)};
\addplot[mark=*] coordinates {(-4.0,0)};
\end{axis}
\end{tikzpicture}%
\newline%
\newline%
\newline%
6) Find the equation of the line with the points (0, 5), (1.67, 0). Round your slope to the nearest whole number:

\begin{tikzpicture} 
\begin{axis}[xmin=-10, xmax=10, ymin=-10, ymax=10, axis x line=middle, axis y line=middle]\addplot[domain=-10:10]{-3*x+5};
\addplot[mark=*] coordinates {(0,5)};
\addplot[mark=*] coordinates {(1.67,0)};
\end{axis}
\end{tikzpicture}%
\newline%
\newline%
\newline%
7) Find the equation of the line with the points (0, 10), (-2.0, 0). Round your slope to the nearest whole number:

\begin{tikzpicture} 
\begin{axis}[xmin=-10, xmax=10, ymin=-10, ymax=10, axis x line=middle, axis y line=middle]\addplot[domain=-10:10]{5*x+10};
\addplot[mark=*] coordinates {(0,10)};
\addplot[mark=*] coordinates {(-2.0,0)};
\end{axis}
\end{tikzpicture}%
\newline%
\newline%
\newline%
8) Find the equation of the line with the points (0, 9), (2.25, 0). Round your slope to the nearest whole number:

\begin{tikzpicture} 
\begin{axis}[xmin=-10, xmax=10, ymin=-10, ymax=10, axis x line=middle, axis y line=middle]\addplot[domain=-10:10]{-4*x+9};
\addplot[mark=*] coordinates {(0,9)};
\addplot[mark=*] coordinates {(2.25,0)};
\end{axis}
\end{tikzpicture}%
\newline%
\newline%
\newline%
9) Find the equation of the line with the points (0, 1), (0.33, 0). Round your slope to the nearest whole number:

\begin{tikzpicture} 
\begin{axis}[xmin=-10, xmax=10, ymin=-10, ymax=10, axis x line=middle, axis y line=middle]\addplot[domain=-10:10]{-3*x+1};
\addplot[mark=*] coordinates {(0,1)};
\addplot[mark=*] coordinates {(0.33,0)};
\end{axis}
\end{tikzpicture}%
\newline%
\newline%
\newline%
10) Find the equation of the line with the points (0, -1), (-0.12, 0). Round your slope to the nearest whole number:

\begin{tikzpicture} 
\begin{axis}[xmin=-10, xmax=10, ymin=-10, ymax=10, axis x line=middle, axis y line=middle]\addplot[domain=-10:10]{-8*x+-1};
\addplot[mark=*] coordinates {(0,-1)};
\addplot[mark=*] coordinates {(-0.12,0)};
\end{axis}
\end{tikzpicture}%
\newline%
\newline%
\newline%
\pagebreak%
\huge%
\vspace*{\fill}%
\begin{center}%
Solutions%
\end{center}%
\vspace*{\fill}%
\normalsize%
\pagebreak%
\large%
\begin{center}%
\textbf{Solving Basic Algebraic Equations- Solution 1}%
\newline%
\end{center} \normalsize%
1) x=5%
\newline%
2) x=5%
\newline%
3) x=6%
\newline%
4) x=3%
\newline%
5) x=6%
\newline%
6) x=2%
\newline%
7) x=8%
\newline%
8) x=5%
\newline%
9) x=3%
\newline%
10) x=8%
\newline%
11) x=4%
\newline%
12) x=5%
\newline%
13) x=9%
\newline%
14) x=9%
\newline%
15) x=4%
\newline%
16) x=1%
\newline%
17) x=6%
\newline%
18) x=7%
\newline%
19) x=4%
\newline%
20) x=1%
\newline%
21) x=8%
\newline%
22) x=1%
\newline%
23) x=3%
\newline%
24) x=2%
\newline%
25) x=4%
\newline%
26) Laura started with 1 books.%
\newline%
27) Laura started with 1 pencils.%
\newline%
28) Alex started with 2 marbles.%
\newline%
29) Laura started with 7 bottles.%
\newline%
30) Sally started with 5 pens.%
\newline%
\newpage%
\large%
\begin{center}%
\textbf{Solving Basic Algebraic Equations- Solution 2}%
\newline%
\end{center} \normalsize%
1) x=5%
\newline%
2) x=5%
\newline%
3) x=6%
\newline%
4) x=3%
\newline%
5) x=6%
\newline%
6) x=2%
\newline%
7) x=8%
\newline%
8) x=5%
\newline%
9) x=3%
\newline%
10) x=8%
\newline%
11) x=4%
\newline%
12) x=5%
\newline%
13) x=9%
\newline%
14) x=9%
\newline%
15) x=4%
\newline%
16) x=1%
\newline%
17) x=6%
\newline%
18) x=7%
\newline%
19) x=4%
\newline%
20) x=1%
\newline%
21) x=8%
\newline%
22) x=1%
\newline%
23) x=3%
\newline%
24) x=2%
\newline%
25) x=4%
\newline%
26) Laura started with 1 books.%
\newline%
27) Laura started with 1 pencils.%
\newline%
28) Alex started with 2 marbles.%
\newline%
29) Laura started with 7 bottles.%
\newline%
30) Sally started with 5 pens.%
\newline%
\newpage%
\large%
\begin{center}%
\textbf{Solving Basic Algebraic Equations- Solution 3}%
\newline%
\end{center} \normalsize%
1) x=5%
\newline%
2) x=5%
\newline%
3) x=6%
\newline%
4) x=3%
\newline%
5) x=6%
\newline%
6) x=2%
\newline%
7) x=8%
\newline%
8) x=5%
\newline%
9) x=3%
\newline%
10) x=8%
\newline%
11) x=4%
\newline%
12) x=5%
\newline%
13) x=9%
\newline%
14) x=9%
\newline%
15) x=4%
\newline%
16) x=1%
\newline%
17) x=6%
\newline%
18) x=7%
\newline%
19) x=4%
\newline%
20) x=1%
\newline%
21) x=8%
\newline%
22) x=1%
\newline%
23) x=3%
\newline%
24) x=2%
\newline%
25) x=4%
\newline%
26) Laura started with 1 books.%
\newline%
27) Laura started with 1 pencils.%
\newline%
28) Alex started with 2 marbles.%
\newline%
29) Laura started with 7 bottles.%
\newline%
30) Sally started with 5 pens.%
\newline%
\newpage%
\large%
\begin{center}%
\textbf{Solving Basic Algebraic Equations- Solution 4}%
\newline%
\end{center} \normalsize%
1) x=5%
\newline%
2) x=5%
\newline%
3) x=6%
\newline%
4) x=3%
\newline%
5) x=6%
\newline%
6) x=2%
\newline%
7) x=8%
\newline%
8) x=5%
\newline%
9) x=3%
\newline%
10) x=8%
\newline%
11) x=4%
\newline%
12) x=5%
\newline%
13) x=9%
\newline%
14) x=9%
\newline%
15) x=4%
\newline%
16) x=1%
\newline%
17) x=6%
\newline%
18) x=7%
\newline%
19) x=4%
\newline%
20) x=1%
\newline%
21) x=8%
\newline%
22) x=1%
\newline%
23) x=3%
\newline%
24) x=2%
\newline%
25) x=4%
\newline%
26) Laura started with 1 books.%
\newline%
27) Laura started with 1 pencils.%
\newline%
28) Alex started with 2 marbles.%
\newline%
29) Laura started with 7 bottles.%
\newline%
30) Sally started with 5 pens.%
\newline%
\newpage%
\large%
\begin{center}%
\textbf{Solving Inequalities- Solution 1}%
\newline%
\end{center} \normalsize%
1) x>3%
\newline%
2) x>5%
\newline%
3) x<7%
\newline%
4) x$\geq$3%
\newline%
5) x<4%
\newline%
6) x$\leq$6%
\newline%
7) x>1%
\newline%
8) x>8%
\newline%
9) x>3%
\newline%
10) x$\geq$8%
\newline%
11) x>6%
\newline%
12) x$\leq$3%
\newline%
13) x<6%
\newline%
14) x$\geq$5%
\newline%
15) x$\leq$9%
\newline%
16) x>2%
\newline%
17) x$\geq$4%
\newline%
18) x$\geq$4%
\newline%
19) x$\leq$4%
\newline%
20) x$\leq$1%
\newline%
21) x>4%
\newline%
22) x<5%
\newline%
23) x$\geq$5%
\newline%
24) x$\geq$8%
\newline%
25) x$\geq$1%
\newline%
26) x$\leq$9%
\newline%
27) x$\leq$7%
\newline%
28) x<3%
\newline%
29) x<4%
\newline%
30) x$\geq$8%
\newline%
\newpage%
\large%
\begin{center}%
\textbf{Solving Inequalities- Solution 2}%
\newline%
\end{center} \normalsize%
1) x>3%
\newline%
2) x>5%
\newline%
3) x<7%
\newline%
4) x$\geq$3%
\newline%
5) x<4%
\newline%
6) x$\leq$6%
\newline%
7) x>1%
\newline%
8) x>8%
\newline%
9) x>3%
\newline%
10) x$\geq$8%
\newline%
11) x>6%
\newline%
12) x$\leq$3%
\newline%
13) x<6%
\newline%
14) x$\geq$5%
\newline%
15) x$\leq$9%
\newline%
16) x>2%
\newline%
17) x$\geq$4%
\newline%
18) x$\geq$4%
\newline%
19) x$\leq$4%
\newline%
20) x$\leq$1%
\newline%
21) x>4%
\newline%
22) x<5%
\newline%
23) x$\geq$5%
\newline%
24) x$\geq$8%
\newline%
25) x$\geq$1%
\newline%
26) x$\leq$9%
\newline%
27) x$\leq$7%
\newline%
28) x<3%
\newline%
29) x<4%
\newline%
30) x$\geq$8%
\newline%
\newpage%
\large%
\begin{center}%
\textbf{Solving Inequalities- Solution 3}%
\newline%
\end{center} \normalsize%
1) x>3%
\newline%
2) x>5%
\newline%
3) x<7%
\newline%
4) x$\geq$3%
\newline%
5) x<4%
\newline%
6) x$\leq$6%
\newline%
7) x>1%
\newline%
8) x>8%
\newline%
9) x>3%
\newline%
10) x$\geq$8%
\newline%
11) x>6%
\newline%
12) x$\leq$3%
\newline%
13) x<6%
\newline%
14) x$\geq$5%
\newline%
15) x$\leq$9%
\newline%
16) x>2%
\newline%
17) x$\geq$4%
\newline%
18) x$\geq$4%
\newline%
19) x$\leq$4%
\newline%
20) x$\leq$1%
\newline%
21) x>4%
\newline%
22) x<5%
\newline%
23) x$\geq$5%
\newline%
24) x$\geq$8%
\newline%
25) x$\geq$1%
\newline%
26) x$\leq$9%
\newline%
27) x$\leq$7%
\newline%
28) x<3%
\newline%
29) x<4%
\newline%
30) x$\geq$8%
\newline%
\newpage%
\large%
\begin{center}%
\textbf{Solving Inequalities- Solution 4}%
\newline%
\end{center} \normalsize%
1) x>3%
\newline%
2) x>5%
\newline%
3) x<7%
\newline%
4) x$\geq$3%
\newline%
5) x<4%
\newline%
6) x$\leq$6%
\newline%
7) x>1%
\newline%
8) x>8%
\newline%
9) x>3%
\newline%
10) x$\geq$8%
\newline%
11) x>6%
\newline%
12) x$\leq$3%
\newline%
13) x<6%
\newline%
14) x$\geq$5%
\newline%
15) x$\leq$9%
\newline%
16) x>2%
\newline%
17) x$\geq$4%
\newline%
18) x$\geq$4%
\newline%
19) x$\leq$4%
\newline%
20) x$\leq$1%
\newline%
21) x>4%
\newline%
22) x<5%
\newline%
23) x$\geq$5%
\newline%
24) x$\geq$8%
\newline%
25) x$\geq$1%
\newline%
26) x$\leq$9%
\newline%
27) x$\leq$7%
\newline%
28) x<3%
\newline%
29) x<4%
\newline%
30) x$\geq$8%
\newline%
\newpage%
\large%
\begin{center}%
\textbf{Factoring Basic Quadratic Equations- Solution 1}%
\newline%
\end{center} \normalsize%
1) (x - 4)(x - 7)%
\newline%
2) (x - 2)(x + 6)%
\newline%
3) (x - 10)(x - 3)%
\newline%
4) (x + 1)(x - 9)%
\newline%
5) (x - 8)(x - 5)%
\newline%
6) (x - 10)(x + 6)%
\newline%
7) (x + 10)(x - 7)%
\newline%
8) (x - 7)(x + 7)%
\newline%
9) (x + 10)(x - 7)%
\newline%
10) (x - 3)(x + 7)%
\newline%
11) (x + 9)(x + 4)%
\newline%
12) (x - 8)(x + 5)%
\newline%
13) (x - 5)(x + 10)%
\newline%
14) (x - 10)(x - 9)%
\newline%
15) (x - 6)(x - 4)%
\newline%
16) (x + 5)(x + 4)%
\newline%
17) (x + 3)(x - 4)%
\newline%
18) (x + 4)(x - 2)%
\newline%
19) (x - 1)(x - 5)%
\newline%
20) (x + 4)(x - 8)%
\newline%
21) (x - 2)(x + 4)%
\newline%
22) (x - 2)(x - 2)%
\newline%
23) (x - 3)(x - 5)%
\newline%
24) (x - 4)(x - 4)%
\newline%
25) (x + 4)(x - 4)%
\newline%
26) (x - 7)(x + 8)%
\newline%
27) (x + 7)(x - 6)%
\newline%
28) (x + 1)(x - 6)%
\newline%
29) (x + 8)(x - 3)%
\newline%
30) (x + 5)(x - 5)%
\newline%
\newpage%
\large%
\begin{center}%
\textbf{Factoring Basic Quadratic Equations- Solution 2}%
\newline%
\end{center} \normalsize%
1) (x - 4)(x - 7)%
\newline%
2) (x - 2)(x + 6)%
\newline%
3) (x - 10)(x - 3)%
\newline%
4) (x + 1)(x - 9)%
\newline%
5) (x - 8)(x - 5)%
\newline%
6) (x - 10)(x + 6)%
\newline%
7) (x + 10)(x - 7)%
\newline%
8) (x - 7)(x + 7)%
\newline%
9) (x + 10)(x - 7)%
\newline%
10) (x - 3)(x + 7)%
\newline%
11) (x + 9)(x + 4)%
\newline%
12) (x - 8)(x + 5)%
\newline%
13) (x - 5)(x + 10)%
\newline%
14) (x - 10)(x - 9)%
\newline%
15) (x - 6)(x - 4)%
\newline%
16) (x + 5)(x + 4)%
\newline%
17) (x + 3)(x - 4)%
\newline%
18) (x + 4)(x - 2)%
\newline%
19) (x - 1)(x - 5)%
\newline%
20) (x + 4)(x - 8)%
\newline%
21) (x - 2)(x + 4)%
\newline%
22) (x - 2)(x - 2)%
\newline%
23) (x - 3)(x - 5)%
\newline%
24) (x - 4)(x - 4)%
\newline%
25) (x + 4)(x - 4)%
\newline%
26) (x - 7)(x + 8)%
\newline%
27) (x + 7)(x - 6)%
\newline%
28) (x + 1)(x - 6)%
\newline%
29) (x + 8)(x - 3)%
\newline%
30) (x + 5)(x - 5)%
\newline%
\newpage%
\large%
\begin{center}%
\textbf{Factoring Basic Quadratic Equations- Solution 3}%
\newline%
\end{center} \normalsize%
1) (x - 4)(x - 7)%
\newline%
2) (x - 2)(x + 6)%
\newline%
3) (x - 10)(x - 3)%
\newline%
4) (x + 1)(x - 9)%
\newline%
5) (x - 8)(x - 5)%
\newline%
6) (x - 10)(x + 6)%
\newline%
7) (x + 10)(x - 7)%
\newline%
8) (x - 7)(x + 7)%
\newline%
9) (x + 10)(x - 7)%
\newline%
10) (x - 3)(x + 7)%
\newline%
11) (x + 9)(x + 4)%
\newline%
12) (x - 8)(x + 5)%
\newline%
13) (x - 5)(x + 10)%
\newline%
14) (x - 10)(x - 9)%
\newline%
15) (x - 6)(x - 4)%
\newline%
16) (x + 5)(x + 4)%
\newline%
17) (x + 3)(x - 4)%
\newline%
18) (x + 4)(x - 2)%
\newline%
19) (x - 1)(x - 5)%
\newline%
20) (x + 4)(x - 8)%
\newline%
21) (x - 2)(x + 4)%
\newline%
22) (x - 2)(x - 2)%
\newline%
23) (x - 3)(x - 5)%
\newline%
24) (x - 4)(x - 4)%
\newline%
25) (x + 4)(x - 4)%
\newline%
26) (x - 7)(x + 8)%
\newline%
27) (x + 7)(x - 6)%
\newline%
28) (x + 1)(x - 6)%
\newline%
29) (x + 8)(x - 3)%
\newline%
30) (x + 5)(x - 5)%
\newline%
\newpage%
\large%
\begin{center}%
\textbf{Factoring Basic Quadratic Equations- Solution 4}%
\newline%
\end{center} \normalsize%
1) (x - 4)(x - 7)%
\newline%
2) (x - 2)(x + 6)%
\newline%
3) (x - 10)(x - 3)%
\newline%
4) (x + 1)(x - 9)%
\newline%
5) (x - 8)(x - 5)%
\newline%
6) (x - 10)(x + 6)%
\newline%
7) (x + 10)(x - 7)%
\newline%
8) (x - 7)(x + 7)%
\newline%
9) (x + 10)(x - 7)%
\newline%
10) (x - 3)(x + 7)%
\newline%
11) (x + 9)(x + 4)%
\newline%
12) (x - 8)(x + 5)%
\newline%
13) (x - 5)(x + 10)%
\newline%
14) (x - 10)(x - 9)%
\newline%
15) (x - 6)(x - 4)%
\newline%
16) (x + 5)(x + 4)%
\newline%
17) (x + 3)(x - 4)%
\newline%
18) (x + 4)(x - 2)%
\newline%
19) (x - 1)(x - 5)%
\newline%
20) (x + 4)(x - 8)%
\newline%
21) (x - 2)(x + 4)%
\newline%
22) (x - 2)(x - 2)%
\newline%
23) (x - 3)(x - 5)%
\newline%
24) (x - 4)(x - 4)%
\newline%
25) (x + 4)(x - 4)%
\newline%
26) (x - 7)(x + 8)%
\newline%
27) (x + 7)(x - 6)%
\newline%
28) (x + 1)(x - 6)%
\newline%
29) (x + 8)(x - 3)%
\newline%
30) (x + 5)(x - 5)%
\newline%
\newpage%
\large%
\begin{center}%
\textbf{Factoring Advanced Quadratic Equations- Solution 1}%
\newline%
\end{center} \normalsize%
1) (6x- 2)(4x - 8)%
\newline%
2) (-1x + 3)(4x + 2)%
\newline%
3) (5x + 5)(-1x - 8)%
\newline%
4) (2x- 8)(-5x - 10)%
\newline%
5) (6x + 8)(-5x - 5)%
\newline%
6) (3x- 3)(5x + 1)%
\newline%
7) (3x- 7)(-2x - 8)%
\newline%
8) (-3x- 1)(-2x + 1)%
\newline%
9) (-6x + 8)(1x - 2)%
\newline%
10) (6x + 9)(-5x - 1)%
\newline%
11) (-3x + 1)(-2x - 5)%
\newline%
12) (3x- 8)(-1x - 8)%
\newline%
13) (-3x- 2)(6x - 7)%
\newline%
14) (1x- 2)(6x - 7)%
\newline%
15) (-3x + 5)(-5x + 8)%
\newline%
16) (-1x + 3)(3x - 6)%
\newline%
17) (-3x + 3)(-5x - 7)%
\newline%
18) (3x- 1)(5x - 2)%
\newline%
19) (5x + 3)(2x + 9)%
\newline%
20) (6x- 5)(-4x + 9)%
\newline%
21) (2x + 7)(6x + 3)%
\newline%
22) (2x- 2)(-5x + 9)%
\newline%
23) (-1x- 4)(-3x - 4)%
\newline%
24) (-1x- 9)(2x + 8)%
\newline%
25) (6x- 1)(2x + 6)%
\newline%
26) (1x- 3)(-3x - 7)%
\newline%
27) (-3x + 3)(-1x + 5)%
\newline%
28) (-6x + 10)(-2x + 10)%
\newline%
29) (6x- 8)(2x + 3)%
\newline%
30) (4x + 7)(-1x + 6)%
\newline%
\newpage%
\large%
\begin{center}%
\textbf{Factoring Advanced Quadratic Equations- Solution 2}%
\newline%
\end{center} \normalsize%
1) (6x- 2)(4x - 8)%
\newline%
2) (-1x + 3)(4x + 2)%
\newline%
3) (5x + 5)(-1x - 8)%
\newline%
4) (2x- 8)(-5x - 10)%
\newline%
5) (6x + 8)(-5x - 5)%
\newline%
6) (3x- 3)(5x + 1)%
\newline%
7) (3x- 7)(-2x - 8)%
\newline%
8) (-3x- 1)(-2x + 1)%
\newline%
9) (-6x + 8)(1x - 2)%
\newline%
10) (6x + 9)(-5x - 1)%
\newline%
11) (-3x + 1)(-2x - 5)%
\newline%
12) (3x- 8)(-1x - 8)%
\newline%
13) (-3x- 2)(6x - 7)%
\newline%
14) (1x- 2)(6x - 7)%
\newline%
15) (-3x + 5)(-5x + 8)%
\newline%
16) (-1x + 3)(3x - 6)%
\newline%
17) (-3x + 3)(-5x - 7)%
\newline%
18) (3x- 1)(5x - 2)%
\newline%
19) (5x + 3)(2x + 9)%
\newline%
20) (6x- 5)(-4x + 9)%
\newline%
21) (2x + 7)(6x + 3)%
\newline%
22) (2x- 2)(-5x + 9)%
\newline%
23) (-1x- 4)(-3x - 4)%
\newline%
24) (-1x- 9)(2x + 8)%
\newline%
25) (6x- 1)(2x + 6)%
\newline%
26) (1x- 3)(-3x - 7)%
\newline%
27) (-3x + 3)(-1x + 5)%
\newline%
28) (-6x + 10)(-2x + 10)%
\newline%
29) (6x- 8)(2x + 3)%
\newline%
30) (4x + 7)(-1x + 6)%
\newline%
\newpage%
\large%
\begin{center}%
\textbf{Factoring Advanced Quadratic Equations- Solution 3}%
\newline%
\end{center} \normalsize%
1) (6x- 2)(4x - 8)%
\newline%
2) (-1x + 3)(4x + 2)%
\newline%
3) (5x + 5)(-1x - 8)%
\newline%
4) (2x- 8)(-5x - 10)%
\newline%
5) (6x + 8)(-5x - 5)%
\newline%
6) (3x- 3)(5x + 1)%
\newline%
7) (3x- 7)(-2x - 8)%
\newline%
8) (-3x- 1)(-2x + 1)%
\newline%
9) (-6x + 8)(1x - 2)%
\newline%
10) (6x + 9)(-5x - 1)%
\newline%
11) (-3x + 1)(-2x - 5)%
\newline%
12) (3x- 8)(-1x - 8)%
\newline%
13) (-3x- 2)(6x - 7)%
\newline%
14) (1x- 2)(6x - 7)%
\newline%
15) (-3x + 5)(-5x + 8)%
\newline%
16) (-1x + 3)(3x - 6)%
\newline%
17) (-3x + 3)(-5x - 7)%
\newline%
18) (3x- 1)(5x - 2)%
\newline%
19) (5x + 3)(2x + 9)%
\newline%
20) (6x- 5)(-4x + 9)%
\newline%
21) (2x + 7)(6x + 3)%
\newline%
22) (2x- 2)(-5x + 9)%
\newline%
23) (-1x- 4)(-3x - 4)%
\newline%
24) (-1x- 9)(2x + 8)%
\newline%
25) (6x- 1)(2x + 6)%
\newline%
26) (1x- 3)(-3x - 7)%
\newline%
27) (-3x + 3)(-1x + 5)%
\newline%
28) (-6x + 10)(-2x + 10)%
\newline%
29) (6x- 8)(2x + 3)%
\newline%
30) (4x + 7)(-1x + 6)%
\newline%
\newpage%
\large%
\begin{center}%
\textbf{Factoring Advanced Quadratic Equations- Solution 4}%
\newline%
\end{center} \normalsize%
1) (6x- 2)(4x - 8)%
\newline%
2) (-1x + 3)(4x + 2)%
\newline%
3) (5x + 5)(-1x - 8)%
\newline%
4) (2x- 8)(-5x - 10)%
\newline%
5) (6x + 8)(-5x - 5)%
\newline%
6) (3x- 3)(5x + 1)%
\newline%
7) (3x- 7)(-2x - 8)%
\newline%
8) (-3x- 1)(-2x + 1)%
\newline%
9) (-6x + 8)(1x - 2)%
\newline%
10) (6x + 9)(-5x - 1)%
\newline%
11) (-3x + 1)(-2x - 5)%
\newline%
12) (3x- 8)(-1x - 8)%
\newline%
13) (-3x- 2)(6x - 7)%
\newline%
14) (1x- 2)(6x - 7)%
\newline%
15) (-3x + 5)(-5x + 8)%
\newline%
16) (-1x + 3)(3x - 6)%
\newline%
17) (-3x + 3)(-5x - 7)%
\newline%
18) (3x- 1)(5x - 2)%
\newline%
19) (5x + 3)(2x + 9)%
\newline%
20) (6x- 5)(-4x + 9)%
\newline%
21) (2x + 7)(6x + 3)%
\newline%
22) (2x- 2)(-5x + 9)%
\newline%
23) (-1x- 4)(-3x - 4)%
\newline%
24) (-1x- 9)(2x + 8)%
\newline%
25) (6x- 1)(2x + 6)%
\newline%
26) (1x- 3)(-3x - 7)%
\newline%
27) (-3x + 3)(-1x + 5)%
\newline%
28) (-6x + 10)(-2x + 10)%
\newline%
29) (6x- 8)(2x + 3)%
\newline%
30) (4x + 7)(-1x + 6)%
\newline%
\newpage%
\large%
\begin{center}%
\textbf{Finding the Equation of a Line- Solution 1}%
\newline%
\end{center} \normalsize%
1) y=-5x + 0%
\newline%
2) y=-8x + -5%
\newline%
3) y=-9x + 10%
\newline%
4) y=-10x + 4%
\newline%
5) y=1x + 4%
\newline%
6) y=-3x + 5%
\newline%
7) y=5x + 10%
\newline%
8) y=-4x + 9%
\newline%
9) y=-3x + 1%
\newline%
10) y=-8x + -1%
\newline%
\newpage%
\large%
\begin{center}%
\textbf{Finding the Equation of a Line- Solution 2}%
\newline%
\end{center} \normalsize%
1) y=-5x + 0%
\newline%
2) y=-8x + -5%
\newline%
3) y=-9x + 10%
\newline%
4) y=-10x + 4%
\newline%
5) y=1x + 4%
\newline%
6) y=-3x + 5%
\newline%
7) y=5x + 10%
\newline%
8) y=-4x + 9%
\newline%
9) y=-3x + 1%
\newline%
10) y=-8x + -1%
\newline%
\newpage%
\large%
\begin{center}%
\textbf{Finding the Equation of a Line- Solution 3}%
\newline%
\end{center} \normalsize%
1) y=-5x + 0%
\newline%
2) y=-8x + -5%
\newline%
3) y=-9x + 10%
\newline%
4) y=-10x + 4%
\newline%
5) y=1x + 4%
\newline%
6) y=-3x + 5%
\newline%
7) y=5x + 10%
\newline%
8) y=-4x + 9%
\newline%
9) y=-3x + 1%
\newline%
10) y=-8x + -1%
\newline%
\newpage%
\large%
\begin{center}%
\textbf{Finding the Equation of a Line- Solution 4}%
\newline%
\end{center} \normalsize%
1) y=-5x + 0%
\newline%
2) y=-8x + -5%
\newline%
3) y=-9x + 10%
\newline%
4) y=-10x + 4%
\newline%
5) y=1x + 4%
\newline%
6) y=-3x + 5%
\newline%
7) y=5x + 10%
\newline%
8) y=-4x + 9%
\newline%
9) y=-3x + 1%
\newline%
10) y=-8x + -1%
\newline%
\newpage%
\end{document}