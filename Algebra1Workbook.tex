\documentclass{article}%
\usepackage[T1]{fontenc}%
\usepackage[utf8]{inputenc}%
\usepackage{lmodern}%
\usepackage{textcomp}%
\usepackage{lastpage}%
\usepackage{tikz}%
\usepackage{pgfplots}%
%
\title{Algebra 1 Math Workbook}%
\author{Akshai Srinivasan, Teja Koripella, Skye Tyrrell, Angellou Sutharsan}%
\date{}%
%
\begin{document}%
\normalsize%
\maketitle%
\vfill%
\begin{center}%
ISBN: 9798848825312%
\linebreak%
\copyright%
MathMaestro.org 2022%
\end{center}%
\newpage%
\section{Table of Contents}%
\label{sec:TableofContents}%
Problems...........................................................................................Page 3%
\newline%
Solving Basic Algebraic Equations%
.%
.%
.%
.%
.%
.%
.%
.%
.%
.%
.%
.%
.%
.%
.%
.%
.%
.%
.%
.%
.%
.%
.%
.%
.%
.%
.%
.%
.%
.%
.%
.%
.%
.%
.%
.%
.%
.%
.%
.%
.%
.%
.%
.%
.%
Page 4{-}15%
\newline%
Solving Inequalities%
.%
.%
.%
.%
.%
.%
.%
.%
.%
.%
.%
.%
.%
.%
.%
.%
.%
.%
.%
.%
.%
.%
.%
.%
.%
.%
.%
.%
.%
.%
.%
.%
.%
.%
.%
.%
.%
.%
.%
.%
.%
.%
.%
.%
.%
.%
.%
.%
.%
.%
.%
.%
.%
.%
.%
.%
.%
.%
.%
.%
.%
.%
.%
.%
.%
.%
.%
.%
.%
Page 16{-}27%
\newline%
Factoring Basic Quadratic Equations%
.%
.%
.%
.%
.%
.%
.%
.%
.%
.%
.%
.%
.%
.%
.%
.%
.%
.%
.%
.%
.%
.%
.%
.%
.%
.%
.%
.%
.%
.%
.%
.%
.%
.%
.%
.%
.%
.%
.%
.%
.%
.%
Page 28{-}39%
\newline%
Factoring Advanced Quadratic Equations%
.%
.%
.%
.%
.%
.%
.%
.%
.%
.%
.%
.%
.%
.%
.%
.%
.%
.%
.%
.%
.%
.%
.%
.%
.%
.%
.%
.%
.%
.%
.%
.%
.%
.%
.%
.%
Page 40{-}51%
\newline%
Finding the Equation of a Line%
.%
.%
.%
.%
.%
.%
.%
.%
.%
.%
.%
.%
.%
.%
.%
.%
.%
.%
.%
.%
.%
.%
.%
.%
.%
.%
.%
.%
.%
.%
.%
.%
.%
.%
.%
.%
.%
.%
.%
.%
.%
.%
.%
.%
.%
.%
.%
.%
.%
.%
.%
Page 52{-}71%
\newline%
Solutions......................................................................................Page 72%
\newline%
Solving Basic Algebraic Equations Solutions%
.%
.%
.%
.%
.%
.%
.%
.%
.%
.%
.%
.%
.%
.%
.%
.%
.%
.%
.%
.%
.%
.%
.%
.%
.%
.%
.%
Page 73{-}76%
\newline%
Solving Inequalities Solutions%
.%
.%
.%
.%
.%
.%
.%
.%
.%
.%
.%
.%
.%
.%
.%
.%
.%
.%
.%
.%
.%
.%
.%
.%
.%
.%
.%
.%
.%
.%
.%
.%
.%
.%
.%
.%
.%
.%
.%
.%
.%
.%
.%
.%
.%
.%
.%
.%
.%
.%
.%
Page 77{-}80%
\newline%
Factoring Basic Quadratic Equations Solutions%
.%
.%
.%
.%
.%
.%
.%
.%
.%
.%
.%
.%
.%
.%
.%
.%
.%
.%
.%
.%
.%
.%
.%
.%
Page 81{-}84%
\newline%
Factoring Advanced Quadratic Equations Solutions%
.%
.%
.%
.%
.%
.%
.%
.%
.%
.%
.%
.%
.%
.%
.%
.%
.%
.%
Page 85{-}88%
\newline%
Finding the Equation of a Line Solutions%
.%
.%
.%
.%
.%
.%
.%
.%
.%
.%
.%
.%
.%
.%
.%
.%
.%
.%
.%
.%
.%
.%
.%
.%
.%
.%
.%
.%
.%
.%
.%
.%
.%
Page 89{-}92%
\newline%
\newpage

%
\huge%
\vspace*{\fill}%
\begin{center}%
Problems%
\end{center}%
\vspace*{\fill}%
\pagebreak%
\normalsize%
\large%
\begin{center}%
\textbf{Solving Basic Algebraic Equations- Worksheet 1}%
\newline%
\newline%
\newline%
\end{center} \normalsize%
1) 9x - 2 = 61%
\newline%
\newline%
\newline%
2) 9x - 5 = 49%
\newline%
\newline%
\newline%
3) 6x - 5 = 13%
\newline%
\newline%
\newline%
4) 4x + 9 = 41%
\newline%
\newline%
\newline%
5) 8x - 5 = 35%
\newline%
\newline%
\newline%
6) x - 5 = -2%
\newline%
\newline%
\newline%
7) 2x - 5 = 5%
\newline%
\newline%
\newline%
8) 9x + 4 = 58%
\newline%
\newline%
\newline%
9) x - 5 = -3%
\newline%
\newline%
\newline%
10) 7x - 3 = 53%
\newline%
\newline%
\newline%
11) 3x - 3 = 6%
\newline%
\newline%
\newline%
12) 5x - 5 = 30%
\newline%
\newline%
\newline%
13) 6x - 1 = 11%
\newline%
\newline%
\newline%
14) 7x + 6 = 13%
\newline%
\newline%
\newline%
15) x + 1 = 3%
\newline%
\newline%
\newline%
16) 6x - 7 = 5%
\newline%
\newline%
\newline%
17) 5x - 2 = 13%
\newline%
\newline%
\newline%
18) 4x - 4 = 12%
\newline%
\newline%
\newline%
19) 3x + 4 = 19%
\newline%
\newline%
\newline%
20) 2x - 1 = 5%
\newline%
\newline%
\newline%
21) x + 8 = 10%
\newline%
\newline%
\newline%
22) 6x - 5 = 13%
\newline%
\newline%
\newline%
23) 7x - 9 = 33%
\newline%
\newline%
\newline%
24) 2x - 3 = 11%
\newline%
\newline%
\newline%
25) 7x + 5 = 40%
\newline%
\newline%
\newline%
26) Bob had an unknown number of bottles. They got 4 times as many bottles before losing 9 more. In total, they now have 3 bottles. How many objects did Bob start with?%
\newline%
\newline%
\newline%
27) Sally had an unknown number of books. They got 7 times as many books before losing 2 more. In total, they now have 33 books. How many objects did Sally start with?%
\newline%
\newline%
\newline%
28) Rachel had an unknown number of marbles. They got 7 times as many marbles before losing 9 more. In total, they now have 5 marbles. How many objects did Rachel start with?%
\newline%
\newline%
\newline%
29) Michael had an unknown number of toys. They got 5 times as many toys before losing 9 more. In total, they now have 1 toys. How many objects did Michael start with?%
\newline%
\newline%
\newline%
30) Alex had an unknown number of marbles. They got 6 times as many marbles before losing 1 more. In total, they now have 29 marbles. How many objects did Alex start with?%
\newline%
\newline%
\newline%
\pagebreak%
\large%
\begin{center}%
\textbf{Solving Basic Algebraic Equations- Worksheet 2}%
\newline%
\newline%
\newline%
\end{center} \normalsize%
1) 9x - 2 = 61%
\newline%
\newline%
\newline%
2) 9x - 5 = 49%
\newline%
\newline%
\newline%
3) 6x - 5 = 13%
\newline%
\newline%
\newline%
4) 4x + 9 = 41%
\newline%
\newline%
\newline%
5) 8x - 5 = 35%
\newline%
\newline%
\newline%
6) x - 5 = -2%
\newline%
\newline%
\newline%
7) 2x - 5 = 5%
\newline%
\newline%
\newline%
8) 9x + 4 = 58%
\newline%
\newline%
\newline%
9) x - 5 = -3%
\newline%
\newline%
\newline%
10) 7x - 3 = 53%
\newline%
\newline%
\newline%
11) 3x - 3 = 6%
\newline%
\newline%
\newline%
12) 5x - 5 = 30%
\newline%
\newline%
\newline%
13) 6x - 1 = 11%
\newline%
\newline%
\newline%
14) 7x + 6 = 13%
\newline%
\newline%
\newline%
15) x + 1 = 3%
\newline%
\newline%
\newline%
16) 6x - 7 = 5%
\newline%
\newline%
\newline%
17) 5x - 2 = 13%
\newline%
\newline%
\newline%
18) 4x - 4 = 12%
\newline%
\newline%
\newline%
19) 3x + 4 = 19%
\newline%
\newline%
\newline%
20) 2x - 1 = 5%
\newline%
\newline%
\newline%
21) x + 8 = 10%
\newline%
\newline%
\newline%
22) 6x - 5 = 13%
\newline%
\newline%
\newline%
23) 7x - 9 = 33%
\newline%
\newline%
\newline%
24) 2x - 3 = 11%
\newline%
\newline%
\newline%
25) 7x + 5 = 40%
\newline%
\newline%
\newline%
26) Bob had an unknown number of bottles. They got 4 times as many bottles before losing 9 more. In total, they now have 3 bottles. How many objects did Bob start with?%
\newline%
\newline%
\newline%
27) Sally had an unknown number of books. They got 7 times as many books before losing 2 more. In total, they now have 33 books. How many objects did Sally start with?%
\newline%
\newline%
\newline%
28) Rachel had an unknown number of marbles. They got 7 times as many marbles before losing 9 more. In total, they now have 5 marbles. How many objects did Rachel start with?%
\newline%
\newline%
\newline%
29) Michael had an unknown number of toys. They got 5 times as many toys before losing 9 more. In total, they now have 1 toys. How many objects did Michael start with?%
\newline%
\newline%
\newline%
30) Alex had an unknown number of marbles. They got 6 times as many marbles before losing 1 more. In total, they now have 29 marbles. How many objects did Alex start with?%
\newline%
\newline%
\newline%
\pagebreak%
\large%
\begin{center}%
\textbf{Solving Basic Algebraic Equations- Worksheet 3}%
\newline%
\newline%
\newline%
\end{center} \normalsize%
1) 9x - 2 = 61%
\newline%
\newline%
\newline%
2) 9x - 5 = 49%
\newline%
\newline%
\newline%
3) 6x - 5 = 13%
\newline%
\newline%
\newline%
4) 4x + 9 = 41%
\newline%
\newline%
\newline%
5) 8x - 5 = 35%
\newline%
\newline%
\newline%
6) x - 5 = -2%
\newline%
\newline%
\newline%
7) 2x - 5 = 5%
\newline%
\newline%
\newline%
8) 9x + 4 = 58%
\newline%
\newline%
\newline%
9) x - 5 = -3%
\newline%
\newline%
\newline%
10) 7x - 3 = 53%
\newline%
\newline%
\newline%
11) 3x - 3 = 6%
\newline%
\newline%
\newline%
12) 5x - 5 = 30%
\newline%
\newline%
\newline%
13) 6x - 1 = 11%
\newline%
\newline%
\newline%
14) 7x + 6 = 13%
\newline%
\newline%
\newline%
15) x + 1 = 3%
\newline%
\newline%
\newline%
16) 6x - 7 = 5%
\newline%
\newline%
\newline%
17) 5x - 2 = 13%
\newline%
\newline%
\newline%
18) 4x - 4 = 12%
\newline%
\newline%
\newline%
19) 3x + 4 = 19%
\newline%
\newline%
\newline%
20) 2x - 1 = 5%
\newline%
\newline%
\newline%
21) x + 8 = 10%
\newline%
\newline%
\newline%
22) 6x - 5 = 13%
\newline%
\newline%
\newline%
23) 7x - 9 = 33%
\newline%
\newline%
\newline%
24) 2x - 3 = 11%
\newline%
\newline%
\newline%
25) 7x + 5 = 40%
\newline%
\newline%
\newline%
26) Bob had an unknown number of bottles. They got 4 times as many bottles before losing 9 more. In total, they now have 3 bottles. How many objects did Bob start with?%
\newline%
\newline%
\newline%
27) Sally had an unknown number of books. They got 7 times as many books before losing 2 more. In total, they now have 33 books. How many objects did Sally start with?%
\newline%
\newline%
\newline%
28) Rachel had an unknown number of marbles. They got 7 times as many marbles before losing 9 more. In total, they now have 5 marbles. How many objects did Rachel start with?%
\newline%
\newline%
\newline%
29) Michael had an unknown number of toys. They got 5 times as many toys before losing 9 more. In total, they now have 1 toys. How many objects did Michael start with?%
\newline%
\newline%
\newline%
30) Alex had an unknown number of marbles. They got 6 times as many marbles before losing 1 more. In total, they now have 29 marbles. How many objects did Alex start with?%
\newline%
\newline%
\newline%
\pagebreak%
\large%
\begin{center}%
\textbf{Solving Basic Algebraic Equations- Worksheet 4}%
\newline%
\newline%
\newline%
\end{center} \normalsize%
1) 9x - 2 = 61%
\newline%
\newline%
\newline%
2) 9x - 5 = 49%
\newline%
\newline%
\newline%
3) 6x - 5 = 13%
\newline%
\newline%
\newline%
4) 4x + 9 = 41%
\newline%
\newline%
\newline%
5) 8x - 5 = 35%
\newline%
\newline%
\newline%
6) x - 5 = -2%
\newline%
\newline%
\newline%
7) 2x - 5 = 5%
\newline%
\newline%
\newline%
8) 9x + 4 = 58%
\newline%
\newline%
\newline%
9) x - 5 = -3%
\newline%
\newline%
\newline%
10) 7x - 3 = 53%
\newline%
\newline%
\newline%
11) 3x - 3 = 6%
\newline%
\newline%
\newline%
12) 5x - 5 = 30%
\newline%
\newline%
\newline%
13) 6x - 1 = 11%
\newline%
\newline%
\newline%
14) 7x + 6 = 13%
\newline%
\newline%
\newline%
15) x + 1 = 3%
\newline%
\newline%
\newline%
16) 6x - 7 = 5%
\newline%
\newline%
\newline%
17) 5x - 2 = 13%
\newline%
\newline%
\newline%
18) 4x - 4 = 12%
\newline%
\newline%
\newline%
19) 3x + 4 = 19%
\newline%
\newline%
\newline%
20) 2x - 1 = 5%
\newline%
\newline%
\newline%
21) x + 8 = 10%
\newline%
\newline%
\newline%
22) 6x - 5 = 13%
\newline%
\newline%
\newline%
23) 7x - 9 = 33%
\newline%
\newline%
\newline%
24) 2x - 3 = 11%
\newline%
\newline%
\newline%
25) 7x + 5 = 40%
\newline%
\newline%
\newline%
26) Bob had an unknown number of bottles. They got 4 times as many bottles before losing 9 more. In total, they now have 3 bottles. How many objects did Bob start with?%
\newline%
\newline%
\newline%
27) Sally had an unknown number of books. They got 7 times as many books before losing 2 more. In total, they now have 33 books. How many objects did Sally start with?%
\newline%
\newline%
\newline%
28) Rachel had an unknown number of marbles. They got 7 times as many marbles before losing 9 more. In total, they now have 5 marbles. How many objects did Rachel start with?%
\newline%
\newline%
\newline%
29) Michael had an unknown number of toys. They got 5 times as many toys before losing 9 more. In total, they now have 1 toys. How many objects did Michael start with?%
\newline%
\newline%
\newline%
30) Alex had an unknown number of marbles. They got 6 times as many marbles before losing 1 more. In total, they now have 29 marbles. How many objects did Alex start with?%
\newline%
\newline%
\newline%
\pagebreak%
\large%
\begin{center}%
\textbf{Solving Inequalities- Worksheet 1}%
\newline%
\newline%
\newline%
\end{center} \normalsize%
1) Solve: 9x + 2 < 38%
\newline%
\newline%
\newline%
2) Solve: 2x + 7 > 21%
\newline%
\newline%
\newline%
3) Solve: -1x + 9 $\geq$ 6%
\newline%
\newline%
\newline%
4) Solve: 2x + 1 > 15%
\newline%
\newline%
\newline%
5) Solve: 5x + 9 < 24%
\newline%
\newline%
\newline%
6) Solve: -6x + 7 $\geq$ -17%
\newline%
\newline%
\newline%
7) Solve: 3x + 5 $\leq$ 17%
\newline%
\newline%
\newline%
8) Solve: -6x - 3 $\leq$ -9%
\newline%
\newline%
\newline%
9) Solve: -7x + 9 $\leq$ -26%
\newline%
\newline%
\newline%
10) Solve: 6x + 4 > 58%
\newline%
\newline%
\newline%
11) Solve: 4x - 1 $\geq$ 27%
\newline%
\newline%
\newline%
12) Solve: -2x - 9 $\geq$ -17%
\newline%
\newline%
\newline%
13) Solve: 6x + 2 < 38%
\newline%
\newline%
\newline%
14) Solve: 3x - 8 < 16%
\newline%
\newline%
\newline%
15) Solve: 8x - 5 $\geq$ 19%
\newline%
\newline%
\newline%
16) Solve: 8x + 2 $\leq$ 66%
\newline%
\newline%
\newline%
17) Solve: 6x + 6 > 18%
\newline%
\newline%
\newline%
18) Solve: 4x - 2 $\geq$ 2%
\newline%
\newline%
\newline%
19) Solve: 9x - 6 $\leq$ 57%
\newline%
\newline%
\newline%
20) Solve: -4x + 1 $\geq$ -23%
\newline%
\newline%
\newline%
21) Solve: 6x - 7 > -1%
\newline%
\newline%
\newline%
22) Solve: -7x + 1 > -62%
\newline%
\newline%
\newline%
23) Solve: 5x + 8 $\leq$ 18%
\newline%
\newline%
\newline%
24) Solve: -2x + 2 $\leq$ -2%
\newline%
\newline%
\newline%
25) Solve: -6x - 9 < -21%
\newline%
\newline%
\newline%
26) Solve: -8x + 9 $\geq$ -63%
\newline%
\newline%
\newline%
27) Solve: 4x - 7 > 1%
\newline%
\newline%
\newline%
28) Solve: -8x + 3 > -61%
\newline%
\newline%
\newline%
29) Solve: -3x + 5 < 2%
\newline%
\newline%
\newline%
30) Solve: -6x - 4 > -46%
\newline%
\newline%
\newline%
\pagebreak%
\large%
\begin{center}%
\textbf{Solving Inequalities- Worksheet 2}%
\newline%
\newline%
\newline%
\end{center} \normalsize%
1) Solve: 9x + 2 < 38%
\newline%
\newline%
\newline%
2) Solve: 2x + 7 > 21%
\newline%
\newline%
\newline%
3) Solve: -1x + 9 $\geq$ 6%
\newline%
\newline%
\newline%
4) Solve: 2x + 1 > 15%
\newline%
\newline%
\newline%
5) Solve: 5x + 9 < 24%
\newline%
\newline%
\newline%
6) Solve: -6x + 7 $\geq$ -17%
\newline%
\newline%
\newline%
7) Solve: 3x + 5 $\leq$ 17%
\newline%
\newline%
\newline%
8) Solve: -6x - 3 $\leq$ -9%
\newline%
\newline%
\newline%
9) Solve: -7x + 9 $\leq$ -26%
\newline%
\newline%
\newline%
10) Solve: 6x + 4 > 58%
\newline%
\newline%
\newline%
11) Solve: 4x - 1 $\geq$ 27%
\newline%
\newline%
\newline%
12) Solve: -2x - 9 $\geq$ -17%
\newline%
\newline%
\newline%
13) Solve: 6x + 2 < 38%
\newline%
\newline%
\newline%
14) Solve: 3x - 8 < 16%
\newline%
\newline%
\newline%
15) Solve: 8x - 5 $\geq$ 19%
\newline%
\newline%
\newline%
16) Solve: 8x + 2 $\leq$ 66%
\newline%
\newline%
\newline%
17) Solve: 6x + 6 > 18%
\newline%
\newline%
\newline%
18) Solve: 4x - 2 $\geq$ 2%
\newline%
\newline%
\newline%
19) Solve: 9x - 6 $\leq$ 57%
\newline%
\newline%
\newline%
20) Solve: -4x + 1 $\geq$ -23%
\newline%
\newline%
\newline%
21) Solve: 6x - 7 > -1%
\newline%
\newline%
\newline%
22) Solve: -7x + 1 > -62%
\newline%
\newline%
\newline%
23) Solve: 5x + 8 $\leq$ 18%
\newline%
\newline%
\newline%
24) Solve: -2x + 2 $\leq$ -2%
\newline%
\newline%
\newline%
25) Solve: -6x - 9 < -21%
\newline%
\newline%
\newline%
26) Solve: -8x + 9 $\geq$ -63%
\newline%
\newline%
\newline%
27) Solve: 4x - 7 > 1%
\newline%
\newline%
\newline%
28) Solve: -8x + 3 > -61%
\newline%
\newline%
\newline%
29) Solve: -3x + 5 < 2%
\newline%
\newline%
\newline%
30) Solve: -6x - 4 > -46%
\newline%
\newline%
\newline%
\pagebreak%
\large%
\begin{center}%
\textbf{Solving Inequalities- Worksheet 3}%
\newline%
\newline%
\newline%
\end{center} \normalsize%
1) Solve: 9x + 2 < 38%
\newline%
\newline%
\newline%
2) Solve: 2x + 7 > 21%
\newline%
\newline%
\newline%
3) Solve: -1x + 9 $\geq$ 6%
\newline%
\newline%
\newline%
4) Solve: 2x + 1 > 15%
\newline%
\newline%
\newline%
5) Solve: 5x + 9 < 24%
\newline%
\newline%
\newline%
6) Solve: -6x + 7 $\geq$ -17%
\newline%
\newline%
\newline%
7) Solve: 3x + 5 $\leq$ 17%
\newline%
\newline%
\newline%
8) Solve: -6x - 3 $\leq$ -9%
\newline%
\newline%
\newline%
9) Solve: -7x + 9 $\leq$ -26%
\newline%
\newline%
\newline%
10) Solve: 6x + 4 > 58%
\newline%
\newline%
\newline%
11) Solve: 4x - 1 $\geq$ 27%
\newline%
\newline%
\newline%
12) Solve: -2x - 9 $\geq$ -17%
\newline%
\newline%
\newline%
13) Solve: 6x + 2 < 38%
\newline%
\newline%
\newline%
14) Solve: 3x - 8 < 16%
\newline%
\newline%
\newline%
15) Solve: 8x - 5 $\geq$ 19%
\newline%
\newline%
\newline%
16) Solve: 8x + 2 $\leq$ 66%
\newline%
\newline%
\newline%
17) Solve: 6x + 6 > 18%
\newline%
\newline%
\newline%
18) Solve: 4x - 2 $\geq$ 2%
\newline%
\newline%
\newline%
19) Solve: 9x - 6 $\leq$ 57%
\newline%
\newline%
\newline%
20) Solve: -4x + 1 $\geq$ -23%
\newline%
\newline%
\newline%
21) Solve: 6x - 7 > -1%
\newline%
\newline%
\newline%
22) Solve: -7x + 1 > -62%
\newline%
\newline%
\newline%
23) Solve: 5x + 8 $\leq$ 18%
\newline%
\newline%
\newline%
24) Solve: -2x + 2 $\leq$ -2%
\newline%
\newline%
\newline%
25) Solve: -6x - 9 < -21%
\newline%
\newline%
\newline%
26) Solve: -8x + 9 $\geq$ -63%
\newline%
\newline%
\newline%
27) Solve: 4x - 7 > 1%
\newline%
\newline%
\newline%
28) Solve: -8x + 3 > -61%
\newline%
\newline%
\newline%
29) Solve: -3x + 5 < 2%
\newline%
\newline%
\newline%
30) Solve: -6x - 4 > -46%
\newline%
\newline%
\newline%
\pagebreak%
\large%
\begin{center}%
\textbf{Solving Inequalities- Worksheet 4}%
\newline%
\newline%
\newline%
\end{center} \normalsize%
1) Solve: 9x + 2 < 38%
\newline%
\newline%
\newline%
2) Solve: 2x + 7 > 21%
\newline%
\newline%
\newline%
3) Solve: -1x + 9 $\geq$ 6%
\newline%
\newline%
\newline%
4) Solve: 2x + 1 > 15%
\newline%
\newline%
\newline%
5) Solve: 5x + 9 < 24%
\newline%
\newline%
\newline%
6) Solve: -6x + 7 $\geq$ -17%
\newline%
\newline%
\newline%
7) Solve: 3x + 5 $\leq$ 17%
\newline%
\newline%
\newline%
8) Solve: -6x - 3 $\leq$ -9%
\newline%
\newline%
\newline%
9) Solve: -7x + 9 $\leq$ -26%
\newline%
\newline%
\newline%
10) Solve: 6x + 4 > 58%
\newline%
\newline%
\newline%
11) Solve: 4x - 1 $\geq$ 27%
\newline%
\newline%
\newline%
12) Solve: -2x - 9 $\geq$ -17%
\newline%
\newline%
\newline%
13) Solve: 6x + 2 < 38%
\newline%
\newline%
\newline%
14) Solve: 3x - 8 < 16%
\newline%
\newline%
\newline%
15) Solve: 8x - 5 $\geq$ 19%
\newline%
\newline%
\newline%
16) Solve: 8x + 2 $\leq$ 66%
\newline%
\newline%
\newline%
17) Solve: 6x + 6 > 18%
\newline%
\newline%
\newline%
18) Solve: 4x - 2 $\geq$ 2%
\newline%
\newline%
\newline%
19) Solve: 9x - 6 $\leq$ 57%
\newline%
\newline%
\newline%
20) Solve: -4x + 1 $\geq$ -23%
\newline%
\newline%
\newline%
21) Solve: 6x - 7 > -1%
\newline%
\newline%
\newline%
22) Solve: -7x + 1 > -62%
\newline%
\newline%
\newline%
23) Solve: 5x + 8 $\leq$ 18%
\newline%
\newline%
\newline%
24) Solve: -2x + 2 $\leq$ -2%
\newline%
\newline%
\newline%
25) Solve: -6x - 9 < -21%
\newline%
\newline%
\newline%
26) Solve: -8x + 9 $\geq$ -63%
\newline%
\newline%
\newline%
27) Solve: 4x - 7 > 1%
\newline%
\newline%
\newline%
28) Solve: -8x + 3 > -61%
\newline%
\newline%
\newline%
29) Solve: -3x + 5 < 2%
\newline%
\newline%
\newline%
30) Solve: -6x - 4 > -46%
\newline%
\newline%
\newline%
\pagebreak%
\large%
\begin{center}%
\textbf{Factoring Basic Quadratic Equations- Worksheet 1}%
\newline%
\newline%
\newline%
\end{center} \normalsize%
1) Factorize: $x^2$ + 13x + 42%
\newline%
\newline%
\newline%
2) Factorize: $x^2$ + 13x + 40%
\newline%
\newline%
\newline%
3) Factorize: $x^2$ + 11x + 10%
\newline%
\newline%
\newline%
4) Factorize: $x^2$ - 2x - 3%
\newline%
\newline%
\newline%
5) Factorize: $x^2$ + 2x - 35%
\newline%
\newline%
\newline%
6) Factorize: $x^2$ + 4x - 45%
\newline%
\newline%
\newline%
7) Factorize: $x^2$ - 2x - 24%
\newline%
\newline%
\newline%
8) Factorize: $x^2$ + 12x + 35%
\newline%
\newline%
\newline%
9) Factorize: $x^2$ + 5x - 50%
\newline%
\newline%
\newline%
10) Factorize: $x^2$ + 5x + 4%
\newline%
\newline%
\newline%
11) Factorize: $x^2$ + 4x - 21%
\newline%
\newline%
\newline%
12) Factorize: $x^2$ + 4x + 3%
\newline%
\newline%
\newline%
13) Factorize: $x^2$ - 12x + 35%
\newline%
\newline%
\newline%
14) Factorize: $x^2$ - 49%
\newline%
\newline%
\newline%
15) Factorize: $x^2$ - 19x + 90%
\newline%
\newline%
\newline%
16) Factorize: $x^2$ - 1x - 30%
\newline%
\newline%
\newline%
17) Factorize: $x^2$ + 11x + 10%
\newline%
\newline%
\newline%
18) Factorize: $x^2$ - 7x - 30%
\newline%
\newline%
\newline%
19) Factorize: $x^2$ - 3x - 54%
\newline%
\newline%
\newline%
20) Factorize: $x^2$ + 6x + 8%
\newline%
\newline%
\newline%
21) Factorize: $x^2$ - 3x - 70%
\newline%
\newline%
\newline%
22) Factorize: $x^2$ + 3x + 2%
\newline%
\newline%
\newline%
23) Factorize: $x^2$ + 2x - 24%
\newline%
\newline%
\newline%
24) Factorize: $x^2$ - 3x - 18%
\newline%
\newline%
\newline%
25) Factorize: $x^2$ + 12x + 32%
\newline%
\newline%
\newline%
26) Factorize: $x^2$ - 8x - 20%
\newline%
\newline%
\newline%
27) Factorize: $x^2$ + 2x - 3%
\newline%
\newline%
\newline%
28) Factorize: $x^2$ + 20x + 100%
\newline%
\newline%
\newline%
29) Factorize: $x^2$ + 8x + 7%
\newline%
\newline%
\newline%
30) Factorize: $x^2$ - 13x + 42%
\newline%
\newline%
\newline%
\pagebreak%
\large%
\begin{center}%
\textbf{Factoring Basic Quadratic Equations- Worksheet 2}%
\newline%
\newline%
\newline%
\end{center} \normalsize%
1) Factorize: $x^2$ + 13x + 42%
\newline%
\newline%
\newline%
2) Factorize: $x^2$ + 13x + 40%
\newline%
\newline%
\newline%
3) Factorize: $x^2$ + 11x + 10%
\newline%
\newline%
\newline%
4) Factorize: $x^2$ - 2x - 3%
\newline%
\newline%
\newline%
5) Factorize: $x^2$ + 2x - 35%
\newline%
\newline%
\newline%
6) Factorize: $x^2$ + 4x - 45%
\newline%
\newline%
\newline%
7) Factorize: $x^2$ - 2x - 24%
\newline%
\newline%
\newline%
8) Factorize: $x^2$ + 12x + 35%
\newline%
\newline%
\newline%
9) Factorize: $x^2$ + 5x - 50%
\newline%
\newline%
\newline%
10) Factorize: $x^2$ + 5x + 4%
\newline%
\newline%
\newline%
11) Factorize: $x^2$ + 4x - 21%
\newline%
\newline%
\newline%
12) Factorize: $x^2$ + 4x + 3%
\newline%
\newline%
\newline%
13) Factorize: $x^2$ - 12x + 35%
\newline%
\newline%
\newline%
14) Factorize: $x^2$ - 49%
\newline%
\newline%
\newline%
15) Factorize: $x^2$ - 19x + 90%
\newline%
\newline%
\newline%
16) Factorize: $x^2$ - 1x - 30%
\newline%
\newline%
\newline%
17) Factorize: $x^2$ + 11x + 10%
\newline%
\newline%
\newline%
18) Factorize: $x^2$ - 7x - 30%
\newline%
\newline%
\newline%
19) Factorize: $x^2$ - 3x - 54%
\newline%
\newline%
\newline%
20) Factorize: $x^2$ + 6x + 8%
\newline%
\newline%
\newline%
21) Factorize: $x^2$ - 3x - 70%
\newline%
\newline%
\newline%
22) Factorize: $x^2$ + 3x + 2%
\newline%
\newline%
\newline%
23) Factorize: $x^2$ + 2x - 24%
\newline%
\newline%
\newline%
24) Factorize: $x^2$ - 3x - 18%
\newline%
\newline%
\newline%
25) Factorize: $x^2$ + 12x + 32%
\newline%
\newline%
\newline%
26) Factorize: $x^2$ - 8x - 20%
\newline%
\newline%
\newline%
27) Factorize: $x^2$ + 2x - 3%
\newline%
\newline%
\newline%
28) Factorize: $x^2$ + 20x + 100%
\newline%
\newline%
\newline%
29) Factorize: $x^2$ + 8x + 7%
\newline%
\newline%
\newline%
30) Factorize: $x^2$ - 13x + 42%
\newline%
\newline%
\newline%
\pagebreak%
\large%
\begin{center}%
\textbf{Factoring Basic Quadratic Equations- Worksheet 3}%
\newline%
\newline%
\newline%
\end{center} \normalsize%
1) Factorize: $x^2$ + 13x + 42%
\newline%
\newline%
\newline%
2) Factorize: $x^2$ + 13x + 40%
\newline%
\newline%
\newline%
3) Factorize: $x^2$ + 11x + 10%
\newline%
\newline%
\newline%
4) Factorize: $x^2$ - 2x - 3%
\newline%
\newline%
\newline%
5) Factorize: $x^2$ + 2x - 35%
\newline%
\newline%
\newline%
6) Factorize: $x^2$ + 4x - 45%
\newline%
\newline%
\newline%
7) Factorize: $x^2$ - 2x - 24%
\newline%
\newline%
\newline%
8) Factorize: $x^2$ + 12x + 35%
\newline%
\newline%
\newline%
9) Factorize: $x^2$ + 5x - 50%
\newline%
\newline%
\newline%
10) Factorize: $x^2$ + 5x + 4%
\newline%
\newline%
\newline%
11) Factorize: $x^2$ + 4x - 21%
\newline%
\newline%
\newline%
12) Factorize: $x^2$ + 4x + 3%
\newline%
\newline%
\newline%
13) Factorize: $x^2$ - 12x + 35%
\newline%
\newline%
\newline%
14) Factorize: $x^2$ - 49%
\newline%
\newline%
\newline%
15) Factorize: $x^2$ - 19x + 90%
\newline%
\newline%
\newline%
16) Factorize: $x^2$ - 1x - 30%
\newline%
\newline%
\newline%
17) Factorize: $x^2$ + 11x + 10%
\newline%
\newline%
\newline%
18) Factorize: $x^2$ - 7x - 30%
\newline%
\newline%
\newline%
19) Factorize: $x^2$ - 3x - 54%
\newline%
\newline%
\newline%
20) Factorize: $x^2$ + 6x + 8%
\newline%
\newline%
\newline%
21) Factorize: $x^2$ - 3x - 70%
\newline%
\newline%
\newline%
22) Factorize: $x^2$ + 3x + 2%
\newline%
\newline%
\newline%
23) Factorize: $x^2$ + 2x - 24%
\newline%
\newline%
\newline%
24) Factorize: $x^2$ - 3x - 18%
\newline%
\newline%
\newline%
25) Factorize: $x^2$ + 12x + 32%
\newline%
\newline%
\newline%
26) Factorize: $x^2$ - 8x - 20%
\newline%
\newline%
\newline%
27) Factorize: $x^2$ + 2x - 3%
\newline%
\newline%
\newline%
28) Factorize: $x^2$ + 20x + 100%
\newline%
\newline%
\newline%
29) Factorize: $x^2$ + 8x + 7%
\newline%
\newline%
\newline%
30) Factorize: $x^2$ - 13x + 42%
\newline%
\newline%
\newline%
\pagebreak%
\large%
\begin{center}%
\textbf{Factoring Basic Quadratic Equations- Worksheet 4}%
\newline%
\newline%
\newline%
\end{center} \normalsize%
1) Factorize: $x^2$ + 13x + 42%
\newline%
\newline%
\newline%
2) Factorize: $x^2$ + 13x + 40%
\newline%
\newline%
\newline%
3) Factorize: $x^2$ + 11x + 10%
\newline%
\newline%
\newline%
4) Factorize: $x^2$ - 2x - 3%
\newline%
\newline%
\newline%
5) Factorize: $x^2$ + 2x - 35%
\newline%
\newline%
\newline%
6) Factorize: $x^2$ + 4x - 45%
\newline%
\newline%
\newline%
7) Factorize: $x^2$ - 2x - 24%
\newline%
\newline%
\newline%
8) Factorize: $x^2$ + 12x + 35%
\newline%
\newline%
\newline%
9) Factorize: $x^2$ + 5x - 50%
\newline%
\newline%
\newline%
10) Factorize: $x^2$ + 5x + 4%
\newline%
\newline%
\newline%
11) Factorize: $x^2$ + 4x - 21%
\newline%
\newline%
\newline%
12) Factorize: $x^2$ + 4x + 3%
\newline%
\newline%
\newline%
13) Factorize: $x^2$ - 12x + 35%
\newline%
\newline%
\newline%
14) Factorize: $x^2$ - 49%
\newline%
\newline%
\newline%
15) Factorize: $x^2$ - 19x + 90%
\newline%
\newline%
\newline%
16) Factorize: $x^2$ - 1x - 30%
\newline%
\newline%
\newline%
17) Factorize: $x^2$ + 11x + 10%
\newline%
\newline%
\newline%
18) Factorize: $x^2$ - 7x - 30%
\newline%
\newline%
\newline%
19) Factorize: $x^2$ - 3x - 54%
\newline%
\newline%
\newline%
20) Factorize: $x^2$ + 6x + 8%
\newline%
\newline%
\newline%
21) Factorize: $x^2$ - 3x - 70%
\newline%
\newline%
\newline%
22) Factorize: $x^2$ + 3x + 2%
\newline%
\newline%
\newline%
23) Factorize: $x^2$ + 2x - 24%
\newline%
\newline%
\newline%
24) Factorize: $x^2$ - 3x - 18%
\newline%
\newline%
\newline%
25) Factorize: $x^2$ + 12x + 32%
\newline%
\newline%
\newline%
26) Factorize: $x^2$ - 8x - 20%
\newline%
\newline%
\newline%
27) Factorize: $x^2$ + 2x - 3%
\newline%
\newline%
\newline%
28) Factorize: $x^2$ + 20x + 100%
\newline%
\newline%
\newline%
29) Factorize: $x^2$ + 8x + 7%
\newline%
\newline%
\newline%
30) Factorize: $x^2$ - 13x + 42%
\newline%
\newline%
\newline%
\pagebreak%
\large%
\begin{center}%
\textbf{Factoring Advanced Quadratic Equations- Worksheet 1}%
\newline%
\newline%
\newline%
\end{center} \normalsize%
1) Factorize: $6x^2$ + 37x + 35%
\newline%
\newline%
\newline%
2) Factorize: $4x^2$ + 20x + 16%
\newline%
\newline%
\newline%
3) Factorize: $8x^2$ - 14x - 72%
\newline%
\newline%
\newline%
4) Factorize: $10x^2$ - 39x + 35%
\newline%
\newline%
\newline%
5) Factorize: $-6x^2$ + 19x - 15%
\newline%
\newline%
\newline%
6) Factorize: $20x^2$ - 49x + 30%
\newline%
\newline%
\newline%
7) Factorize: $10x^2$ - 1x - 24%
\newline%
\newline%
\newline%
8) Factorize: $-6x^2$ - 31x - 35%
\newline%
\newline%
\newline%
9) Factorize: $20x^2$ + 38x - 30%
\newline%
\newline%
\newline%
10) Factorize: $10x^2$ + 15x - 10%
\newline%
\newline%
\newline%
11) Factorize: $-16x^2$ + 48x - 27%
\newline%
\newline%
\newline%
12) Factorize: $-4x^2$ - 21x - 5%
\newline%
\newline%
\newline%
13) Factorize: $12x^2$ + 31x + 9%
\newline%
\newline%
\newline%
14) Factorize: $-2x^2$ + 12x + 14%
\newline%
\newline%
\newline%
15) Factorize: $10x^2$ - 39x + 14%
\newline%
\newline%
\newline%
16) Factorize: $-3x^2$ - 32x - 45%
\newline%
\newline%
\newline%
17) Factorize: $2x^2$ - 11x - 21%
\newline%
\newline%
\newline%
18) Factorize: $12x^2$ - 14x - 10%
\newline%
\newline%
\newline%
19) Factorize: $-3x^2$ + 18x - 24%
\newline%
\newline%
\newline%
20) Factorize: $36x^2$ + 12x - 80%
\newline%
\newline%
\newline%
21) Factorize: $4x^2$ + 16x + 12%
\newline%
\newline%
\newline%
22) Factorize: $12x^2$ + 47x + 45%
\newline%
\newline%
\newline%
23) Factorize: $-10x^2$ + 28x - 18%
\newline%
\newline%
\newline%
24) Factorize: $-15x^2$ + 25x + 10%
\newline%
\newline%
\newline%
25) Factorize: $-4x^2$ - 27x - 18%
\newline%
\newline%
\newline%
26) Factorize: $4x^2$ - 10x + 6%
\newline%
\newline%
\newline%
27) Factorize: $24x^2$ + 14x - 49%
\newline%
\newline%
\newline%
28) Factorize: $-30x^2$ - 10x + 20%
\newline%
\newline%
\newline%
29) Factorize: $-4x^2$ - 11x - 6%
\newline%
\newline%
\newline%
30) Factorize: $-4x^2$ - 4x + 35%
\newline%
\newline%
\newline%
\pagebreak%
\large%
\begin{center}%
\textbf{Factoring Advanced Quadratic Equations- Worksheet 2}%
\newline%
\newline%
\newline%
\end{center} \normalsize%
1) Factorize: $6x^2$ + 37x + 35%
\newline%
\newline%
\newline%
2) Factorize: $4x^2$ + 20x + 16%
\newline%
\newline%
\newline%
3) Factorize: $8x^2$ - 14x - 72%
\newline%
\newline%
\newline%
4) Factorize: $10x^2$ - 39x + 35%
\newline%
\newline%
\newline%
5) Factorize: $-6x^2$ + 19x - 15%
\newline%
\newline%
\newline%
6) Factorize: $20x^2$ - 49x + 30%
\newline%
\newline%
\newline%
7) Factorize: $10x^2$ - 1x - 24%
\newline%
\newline%
\newline%
8) Factorize: $-6x^2$ - 31x - 35%
\newline%
\newline%
\newline%
9) Factorize: $20x^2$ + 38x - 30%
\newline%
\newline%
\newline%
10) Factorize: $10x^2$ + 15x - 10%
\newline%
\newline%
\newline%
11) Factorize: $-16x^2$ + 48x - 27%
\newline%
\newline%
\newline%
12) Factorize: $-4x^2$ - 21x - 5%
\newline%
\newline%
\newline%
13) Factorize: $12x^2$ + 31x + 9%
\newline%
\newline%
\newline%
14) Factorize: $-2x^2$ + 12x + 14%
\newline%
\newline%
\newline%
15) Factorize: $10x^2$ - 39x + 14%
\newline%
\newline%
\newline%
16) Factorize: $-3x^2$ - 32x - 45%
\newline%
\newline%
\newline%
17) Factorize: $2x^2$ - 11x - 21%
\newline%
\newline%
\newline%
18) Factorize: $12x^2$ - 14x - 10%
\newline%
\newline%
\newline%
19) Factorize: $-3x^2$ + 18x - 24%
\newline%
\newline%
\newline%
20) Factorize: $36x^2$ + 12x - 80%
\newline%
\newline%
\newline%
21) Factorize: $4x^2$ + 16x + 12%
\newline%
\newline%
\newline%
22) Factorize: $12x^2$ + 47x + 45%
\newline%
\newline%
\newline%
23) Factorize: $-10x^2$ + 28x - 18%
\newline%
\newline%
\newline%
24) Factorize: $-15x^2$ + 25x + 10%
\newline%
\newline%
\newline%
25) Factorize: $-4x^2$ - 27x - 18%
\newline%
\newline%
\newline%
26) Factorize: $4x^2$ - 10x + 6%
\newline%
\newline%
\newline%
27) Factorize: $24x^2$ + 14x - 49%
\newline%
\newline%
\newline%
28) Factorize: $-30x^2$ - 10x + 20%
\newline%
\newline%
\newline%
29) Factorize: $-4x^2$ - 11x - 6%
\newline%
\newline%
\newline%
30) Factorize: $-4x^2$ - 4x + 35%
\newline%
\newline%
\newline%
\pagebreak%
\large%
\begin{center}%
\textbf{Factoring Advanced Quadratic Equations- Worksheet 3}%
\newline%
\newline%
\newline%
\end{center} \normalsize%
1) Factorize: $6x^2$ + 37x + 35%
\newline%
\newline%
\newline%
2) Factorize: $4x^2$ + 20x + 16%
\newline%
\newline%
\newline%
3) Factorize: $8x^2$ - 14x - 72%
\newline%
\newline%
\newline%
4) Factorize: $10x^2$ - 39x + 35%
\newline%
\newline%
\newline%
5) Factorize: $-6x^2$ + 19x - 15%
\newline%
\newline%
\newline%
6) Factorize: $20x^2$ - 49x + 30%
\newline%
\newline%
\newline%
7) Factorize: $10x^2$ - 1x - 24%
\newline%
\newline%
\newline%
8) Factorize: $-6x^2$ - 31x - 35%
\newline%
\newline%
\newline%
9) Factorize: $20x^2$ + 38x - 30%
\newline%
\newline%
\newline%
10) Factorize: $10x^2$ + 15x - 10%
\newline%
\newline%
\newline%
11) Factorize: $-16x^2$ + 48x - 27%
\newline%
\newline%
\newline%
12) Factorize: $-4x^2$ - 21x - 5%
\newline%
\newline%
\newline%
13) Factorize: $12x^2$ + 31x + 9%
\newline%
\newline%
\newline%
14) Factorize: $-2x^2$ + 12x + 14%
\newline%
\newline%
\newline%
15) Factorize: $10x^2$ - 39x + 14%
\newline%
\newline%
\newline%
16) Factorize: $-3x^2$ - 32x - 45%
\newline%
\newline%
\newline%
17) Factorize: $2x^2$ - 11x - 21%
\newline%
\newline%
\newline%
18) Factorize: $12x^2$ - 14x - 10%
\newline%
\newline%
\newline%
19) Factorize: $-3x^2$ + 18x - 24%
\newline%
\newline%
\newline%
20) Factorize: $36x^2$ + 12x - 80%
\newline%
\newline%
\newline%
21) Factorize: $4x^2$ + 16x + 12%
\newline%
\newline%
\newline%
22) Factorize: $12x^2$ + 47x + 45%
\newline%
\newline%
\newline%
23) Factorize: $-10x^2$ + 28x - 18%
\newline%
\newline%
\newline%
24) Factorize: $-15x^2$ + 25x + 10%
\newline%
\newline%
\newline%
25) Factorize: $-4x^2$ - 27x - 18%
\newline%
\newline%
\newline%
26) Factorize: $4x^2$ - 10x + 6%
\newline%
\newline%
\newline%
27) Factorize: $24x^2$ + 14x - 49%
\newline%
\newline%
\newline%
28) Factorize: $-30x^2$ - 10x + 20%
\newline%
\newline%
\newline%
29) Factorize: $-4x^2$ - 11x - 6%
\newline%
\newline%
\newline%
30) Factorize: $-4x^2$ - 4x + 35%
\newline%
\newline%
\newline%
\pagebreak%
\large%
\begin{center}%
\textbf{Factoring Advanced Quadratic Equations- Worksheet 4}%
\newline%
\newline%
\newline%
\end{center} \normalsize%
1) Factorize: $6x^2$ + 37x + 35%
\newline%
\newline%
\newline%
2) Factorize: $4x^2$ + 20x + 16%
\newline%
\newline%
\newline%
3) Factorize: $8x^2$ - 14x - 72%
\newline%
\newline%
\newline%
4) Factorize: $10x^2$ - 39x + 35%
\newline%
\newline%
\newline%
5) Factorize: $-6x^2$ + 19x - 15%
\newline%
\newline%
\newline%
6) Factorize: $20x^2$ - 49x + 30%
\newline%
\newline%
\newline%
7) Factorize: $10x^2$ - 1x - 24%
\newline%
\newline%
\newline%
8) Factorize: $-6x^2$ - 31x - 35%
\newline%
\newline%
\newline%
9) Factorize: $20x^2$ + 38x - 30%
\newline%
\newline%
\newline%
10) Factorize: $10x^2$ + 15x - 10%
\newline%
\newline%
\newline%
11) Factorize: $-16x^2$ + 48x - 27%
\newline%
\newline%
\newline%
12) Factorize: $-4x^2$ - 21x - 5%
\newline%
\newline%
\newline%
13) Factorize: $12x^2$ + 31x + 9%
\newline%
\newline%
\newline%
14) Factorize: $-2x^2$ + 12x + 14%
\newline%
\newline%
\newline%
15) Factorize: $10x^2$ - 39x + 14%
\newline%
\newline%
\newline%
16) Factorize: $-3x^2$ - 32x - 45%
\newline%
\newline%
\newline%
17) Factorize: $2x^2$ - 11x - 21%
\newline%
\newline%
\newline%
18) Factorize: $12x^2$ - 14x - 10%
\newline%
\newline%
\newline%
19) Factorize: $-3x^2$ + 18x - 24%
\newline%
\newline%
\newline%
20) Factorize: $36x^2$ + 12x - 80%
\newline%
\newline%
\newline%
21) Factorize: $4x^2$ + 16x + 12%
\newline%
\newline%
\newline%
22) Factorize: $12x^2$ + 47x + 45%
\newline%
\newline%
\newline%
23) Factorize: $-10x^2$ + 28x - 18%
\newline%
\newline%
\newline%
24) Factorize: $-15x^2$ + 25x + 10%
\newline%
\newline%
\newline%
25) Factorize: $-4x^2$ - 27x - 18%
\newline%
\newline%
\newline%
26) Factorize: $4x^2$ - 10x + 6%
\newline%
\newline%
\newline%
27) Factorize: $24x^2$ + 14x - 49%
\newline%
\newline%
\newline%
28) Factorize: $-30x^2$ - 10x + 20%
\newline%
\newline%
\newline%
29) Factorize: $-4x^2$ - 11x - 6%
\newline%
\newline%
\newline%
30) Factorize: $-4x^2$ - 4x + 35%
\newline%
\newline%
\newline%
\pagebreak%
\large%
\begin{center}%
\textbf{Finding the Equation of a Line- Worksheet 1}%
\newline%
\newline%
\newline%
\end{center} \normalsize%
1) Find the equation of the line with the points (0, -1), (0.25, 0). Round your slope to the nearest whole number:

\begin{tikzpicture} 
\begin{axis}[xmin=-10, xmax=10, ymin=-10, ymax=10, axis x line=middle, axis y line=middle]\addplot[domain=-10:10]{4*x+-1};
\addplot[mark=*] coordinates {(0,-1)};
\addplot[mark=*] coordinates {(0.25,0)};
\end{axis}
\end{tikzpicture}%
\newline%
\newline%
\newline%
2) Find the equation of the line with the points (0, 5), (0.83, 0). Round your slope to the nearest whole number:

\begin{tikzpicture} 
\begin{axis}[xmin=-10, xmax=10, ymin=-10, ymax=10, axis x line=middle, axis y line=middle]\addplot[domain=-10:10]{-6*x+5};
\addplot[mark=*] coordinates {(0,5)};
\addplot[mark=*] coordinates {(0.83,0)};
\end{axis}
\end{tikzpicture}%
\newline%
\newline%
\newline%
3) Find the equation of the line with the points (0, -5), (1.25, 0). Round your slope to the nearest whole number:

\begin{tikzpicture} 
\begin{axis}[xmin=-10, xmax=10, ymin=-10, ymax=10, axis x line=middle, axis y line=middle]\addplot[domain=-10:10]{4*x+-5};
\addplot[mark=*] coordinates {(0,-5)};
\addplot[mark=*] coordinates {(1.25,0)};
\end{axis}
\end{tikzpicture}%
\newline%
\newline%
\newline%
4) Find the equation of the line with the points (0, -6), (0.75, 0). Round your slope to the nearest whole number:

\begin{tikzpicture} 
\begin{axis}[xmin=-10, xmax=10, ymin=-10, ymax=10, axis x line=middle, axis y line=middle]\addplot[domain=-10:10]{8*x+-6};
\addplot[mark=*] coordinates {(0,-6)};
\addplot[mark=*] coordinates {(0.75,0)};
\end{axis}
\end{tikzpicture}%
\newline%
\newline%
\newline%
5) Find the equation of the line with the points (0, -4), (1.0, 0). Round your slope to the nearest whole number:

\begin{tikzpicture} 
\begin{axis}[xmin=-10, xmax=10, ymin=-10, ymax=10, axis x line=middle, axis y line=middle]\addplot[domain=-10:10]{4*x+-4};
\addplot[mark=*] coordinates {(0,-4)};
\addplot[mark=*] coordinates {(1.0,0)};
\end{axis}
\end{tikzpicture}%
\newline%
\newline%
\newline%
6) Find the equation of the line with the points (0, 1), (-0.12, 0). Round your slope to the nearest whole number:

\begin{tikzpicture} 
\begin{axis}[xmin=-10, xmax=10, ymin=-10, ymax=10, axis x line=middle, axis y line=middle]\addplot[domain=-10:10]{8*x+1};
\addplot[mark=*] coordinates {(0,1)};
\addplot[mark=*] coordinates {(-0.12,0)};
\end{axis}
\end{tikzpicture}%
\newline%
\newline%
\newline%
7) Find the equation of the line with the points (0, 7), (3.5, 0). Round your slope to the nearest whole number:

\begin{tikzpicture} 
\begin{axis}[xmin=-10, xmax=10, ymin=-10, ymax=10, axis x line=middle, axis y line=middle]\addplot[domain=-10:10]{-2*x+7};
\addplot[mark=*] coordinates {(0,7)};
\addplot[mark=*] coordinates {(3.5,0)};
\end{axis}
\end{tikzpicture}%
\newline%
\newline%
\newline%
8) Find the equation of the line with the points (0, -3), (-0.6, 0). Round your slope to the nearest whole number:

\begin{tikzpicture} 
\begin{axis}[xmin=-10, xmax=10, ymin=-10, ymax=10, axis x line=middle, axis y line=middle]\addplot[domain=-10:10]{-5*x+-3};
\addplot[mark=*] coordinates {(0,-3)};
\addplot[mark=*] coordinates {(-0.6,0)};
\end{axis}
\end{tikzpicture}%
\newline%
\newline%
\newline%
9) Find the equation of the line with the points (0, -4), (0.44, 0). Round your slope to the nearest whole number:

\begin{tikzpicture} 
\begin{axis}[xmin=-10, xmax=10, ymin=-10, ymax=10, axis x line=middle, axis y line=middle]\addplot[domain=-10:10]{9*x+-4};
\addplot[mark=*] coordinates {(0,-4)};
\addplot[mark=*] coordinates {(0.44,0)};
\end{axis}
\end{tikzpicture}%
\newline%
\newline%
\newline%
10) Find the equation of the line with the points (0, 2), (-0.4, 0). Round your slope to the nearest whole number:

\begin{tikzpicture} 
\begin{axis}[xmin=-10, xmax=10, ymin=-10, ymax=10, axis x line=middle, axis y line=middle]\addplot[domain=-10:10]{5*x+2};
\addplot[mark=*] coordinates {(0,2)};
\addplot[mark=*] coordinates {(-0.4,0)};
\end{axis}
\end{tikzpicture}%
\newline%
\newline%
\newline%
\pagebreak%
\large%
\begin{center}%
\textbf{Finding the Equation of a Line- Worksheet 2}%
\newline%
\newline%
\newline%
\end{center} \normalsize%
1) Find the equation of the line with the points (0, -1), (0.25, 0). Round your slope to the nearest whole number:

\begin{tikzpicture} 
\begin{axis}[xmin=-10, xmax=10, ymin=-10, ymax=10, axis x line=middle, axis y line=middle]\addplot[domain=-10:10]{4*x+-1};
\addplot[mark=*] coordinates {(0,-1)};
\addplot[mark=*] coordinates {(0.25,0)};
\end{axis}
\end{tikzpicture}%
\newline%
\newline%
\newline%
2) Find the equation of the line with the points (0, 5), (0.83, 0). Round your slope to the nearest whole number:

\begin{tikzpicture} 
\begin{axis}[xmin=-10, xmax=10, ymin=-10, ymax=10, axis x line=middle, axis y line=middle]\addplot[domain=-10:10]{-6*x+5};
\addplot[mark=*] coordinates {(0,5)};
\addplot[mark=*] coordinates {(0.83,0)};
\end{axis}
\end{tikzpicture}%
\newline%
\newline%
\newline%
3) Find the equation of the line with the points (0, -5), (1.25, 0). Round your slope to the nearest whole number:

\begin{tikzpicture} 
\begin{axis}[xmin=-10, xmax=10, ymin=-10, ymax=10, axis x line=middle, axis y line=middle]\addplot[domain=-10:10]{4*x+-5};
\addplot[mark=*] coordinates {(0,-5)};
\addplot[mark=*] coordinates {(1.25,0)};
\end{axis}
\end{tikzpicture}%
\newline%
\newline%
\newline%
4) Find the equation of the line with the points (0, -6), (0.75, 0). Round your slope to the nearest whole number:

\begin{tikzpicture} 
\begin{axis}[xmin=-10, xmax=10, ymin=-10, ymax=10, axis x line=middle, axis y line=middle]\addplot[domain=-10:10]{8*x+-6};
\addplot[mark=*] coordinates {(0,-6)};
\addplot[mark=*] coordinates {(0.75,0)};
\end{axis}
\end{tikzpicture}%
\newline%
\newline%
\newline%
5) Find the equation of the line with the points (0, -4), (1.0, 0). Round your slope to the nearest whole number:

\begin{tikzpicture} 
\begin{axis}[xmin=-10, xmax=10, ymin=-10, ymax=10, axis x line=middle, axis y line=middle]\addplot[domain=-10:10]{4*x+-4};
\addplot[mark=*] coordinates {(0,-4)};
\addplot[mark=*] coordinates {(1.0,0)};
\end{axis}
\end{tikzpicture}%
\newline%
\newline%
\newline%
6) Find the equation of the line with the points (0, 1), (-0.12, 0). Round your slope to the nearest whole number:

\begin{tikzpicture} 
\begin{axis}[xmin=-10, xmax=10, ymin=-10, ymax=10, axis x line=middle, axis y line=middle]\addplot[domain=-10:10]{8*x+1};
\addplot[mark=*] coordinates {(0,1)};
\addplot[mark=*] coordinates {(-0.12,0)};
\end{axis}
\end{tikzpicture}%
\newline%
\newline%
\newline%
7) Find the equation of the line with the points (0, 7), (3.5, 0). Round your slope to the nearest whole number:

\begin{tikzpicture} 
\begin{axis}[xmin=-10, xmax=10, ymin=-10, ymax=10, axis x line=middle, axis y line=middle]\addplot[domain=-10:10]{-2*x+7};
\addplot[mark=*] coordinates {(0,7)};
\addplot[mark=*] coordinates {(3.5,0)};
\end{axis}
\end{tikzpicture}%
\newline%
\newline%
\newline%
8) Find the equation of the line with the points (0, -3), (-0.6, 0). Round your slope to the nearest whole number:

\begin{tikzpicture} 
\begin{axis}[xmin=-10, xmax=10, ymin=-10, ymax=10, axis x line=middle, axis y line=middle]\addplot[domain=-10:10]{-5*x+-3};
\addplot[mark=*] coordinates {(0,-3)};
\addplot[mark=*] coordinates {(-0.6,0)};
\end{axis}
\end{tikzpicture}%
\newline%
\newline%
\newline%
9) Find the equation of the line with the points (0, -4), (0.44, 0). Round your slope to the nearest whole number:

\begin{tikzpicture} 
\begin{axis}[xmin=-10, xmax=10, ymin=-10, ymax=10, axis x line=middle, axis y line=middle]\addplot[domain=-10:10]{9*x+-4};
\addplot[mark=*] coordinates {(0,-4)};
\addplot[mark=*] coordinates {(0.44,0)};
\end{axis}
\end{tikzpicture}%
\newline%
\newline%
\newline%
10) Find the equation of the line with the points (0, 2), (-0.4, 0). Round your slope to the nearest whole number:

\begin{tikzpicture} 
\begin{axis}[xmin=-10, xmax=10, ymin=-10, ymax=10, axis x line=middle, axis y line=middle]\addplot[domain=-10:10]{5*x+2};
\addplot[mark=*] coordinates {(0,2)};
\addplot[mark=*] coordinates {(-0.4,0)};
\end{axis}
\end{tikzpicture}%
\newline%
\newline%
\newline%
\pagebreak%
\large%
\begin{center}%
\textbf{Finding the Equation of a Line- Worksheet 3}%
\newline%
\newline%
\newline%
\end{center} \normalsize%
1) Find the equation of the line with the points (0, -1), (0.25, 0). Round your slope to the nearest whole number:

\begin{tikzpicture} 
\begin{axis}[xmin=-10, xmax=10, ymin=-10, ymax=10, axis x line=middle, axis y line=middle]\addplot[domain=-10:10]{4*x+-1};
\addplot[mark=*] coordinates {(0,-1)};
\addplot[mark=*] coordinates {(0.25,0)};
\end{axis}
\end{tikzpicture}%
\newline%
\newline%
\newline%
2) Find the equation of the line with the points (0, 5), (0.83, 0). Round your slope to the nearest whole number:

\begin{tikzpicture} 
\begin{axis}[xmin=-10, xmax=10, ymin=-10, ymax=10, axis x line=middle, axis y line=middle]\addplot[domain=-10:10]{-6*x+5};
\addplot[mark=*] coordinates {(0,5)};
\addplot[mark=*] coordinates {(0.83,0)};
\end{axis}
\end{tikzpicture}%
\newline%
\newline%
\newline%
3) Find the equation of the line with the points (0, -5), (1.25, 0). Round your slope to the nearest whole number:

\begin{tikzpicture} 
\begin{axis}[xmin=-10, xmax=10, ymin=-10, ymax=10, axis x line=middle, axis y line=middle]\addplot[domain=-10:10]{4*x+-5};
\addplot[mark=*] coordinates {(0,-5)};
\addplot[mark=*] coordinates {(1.25,0)};
\end{axis}
\end{tikzpicture}%
\newline%
\newline%
\newline%
4) Find the equation of the line with the points (0, -6), (0.75, 0). Round your slope to the nearest whole number:

\begin{tikzpicture} 
\begin{axis}[xmin=-10, xmax=10, ymin=-10, ymax=10, axis x line=middle, axis y line=middle]\addplot[domain=-10:10]{8*x+-6};
\addplot[mark=*] coordinates {(0,-6)};
\addplot[mark=*] coordinates {(0.75,0)};
\end{axis}
\end{tikzpicture}%
\newline%
\newline%
\newline%
5) Find the equation of the line with the points (0, -4), (1.0, 0). Round your slope to the nearest whole number:

\begin{tikzpicture} 
\begin{axis}[xmin=-10, xmax=10, ymin=-10, ymax=10, axis x line=middle, axis y line=middle]\addplot[domain=-10:10]{4*x+-4};
\addplot[mark=*] coordinates {(0,-4)};
\addplot[mark=*] coordinates {(1.0,0)};
\end{axis}
\end{tikzpicture}%
\newline%
\newline%
\newline%
6) Find the equation of the line with the points (0, 1), (-0.12, 0). Round your slope to the nearest whole number:

\begin{tikzpicture} 
\begin{axis}[xmin=-10, xmax=10, ymin=-10, ymax=10, axis x line=middle, axis y line=middle]\addplot[domain=-10:10]{8*x+1};
\addplot[mark=*] coordinates {(0,1)};
\addplot[mark=*] coordinates {(-0.12,0)};
\end{axis}
\end{tikzpicture}%
\newline%
\newline%
\newline%
7) Find the equation of the line with the points (0, 7), (3.5, 0). Round your slope to the nearest whole number:

\begin{tikzpicture} 
\begin{axis}[xmin=-10, xmax=10, ymin=-10, ymax=10, axis x line=middle, axis y line=middle]\addplot[domain=-10:10]{-2*x+7};
\addplot[mark=*] coordinates {(0,7)};
\addplot[mark=*] coordinates {(3.5,0)};
\end{axis}
\end{tikzpicture}%
\newline%
\newline%
\newline%
8) Find the equation of the line with the points (0, -3), (-0.6, 0). Round your slope to the nearest whole number:

\begin{tikzpicture} 
\begin{axis}[xmin=-10, xmax=10, ymin=-10, ymax=10, axis x line=middle, axis y line=middle]\addplot[domain=-10:10]{-5*x+-3};
\addplot[mark=*] coordinates {(0,-3)};
\addplot[mark=*] coordinates {(-0.6,0)};
\end{axis}
\end{tikzpicture}%
\newline%
\newline%
\newline%
9) Find the equation of the line with the points (0, -4), (0.44, 0). Round your slope to the nearest whole number:

\begin{tikzpicture} 
\begin{axis}[xmin=-10, xmax=10, ymin=-10, ymax=10, axis x line=middle, axis y line=middle]\addplot[domain=-10:10]{9*x+-4};
\addplot[mark=*] coordinates {(0,-4)};
\addplot[mark=*] coordinates {(0.44,0)};
\end{axis}
\end{tikzpicture}%
\newline%
\newline%
\newline%
10) Find the equation of the line with the points (0, 2), (-0.4, 0). Round your slope to the nearest whole number:

\begin{tikzpicture} 
\begin{axis}[xmin=-10, xmax=10, ymin=-10, ymax=10, axis x line=middle, axis y line=middle]\addplot[domain=-10:10]{5*x+2};
\addplot[mark=*] coordinates {(0,2)};
\addplot[mark=*] coordinates {(-0.4,0)};
\end{axis}
\end{tikzpicture}%
\newline%
\newline%
\newline%
\pagebreak%
\large%
\begin{center}%
\textbf{Finding the Equation of a Line- Worksheet 4}%
\newline%
\newline%
\newline%
\end{center} \normalsize%
1) Find the equation of the line with the points (0, -1), (0.25, 0). Round your slope to the nearest whole number:

\begin{tikzpicture} 
\begin{axis}[xmin=-10, xmax=10, ymin=-10, ymax=10, axis x line=middle, axis y line=middle]\addplot[domain=-10:10]{4*x+-1};
\addplot[mark=*] coordinates {(0,-1)};
\addplot[mark=*] coordinates {(0.25,0)};
\end{axis}
\end{tikzpicture}%
\newline%
\newline%
\newline%
2) Find the equation of the line with the points (0, 5), (0.83, 0). Round your slope to the nearest whole number:

\begin{tikzpicture} 
\begin{axis}[xmin=-10, xmax=10, ymin=-10, ymax=10, axis x line=middle, axis y line=middle]\addplot[domain=-10:10]{-6*x+5};
\addplot[mark=*] coordinates {(0,5)};
\addplot[mark=*] coordinates {(0.83,0)};
\end{axis}
\end{tikzpicture}%
\newline%
\newline%
\newline%
3) Find the equation of the line with the points (0, -5), (1.25, 0). Round your slope to the nearest whole number:

\begin{tikzpicture} 
\begin{axis}[xmin=-10, xmax=10, ymin=-10, ymax=10, axis x line=middle, axis y line=middle]\addplot[domain=-10:10]{4*x+-5};
\addplot[mark=*] coordinates {(0,-5)};
\addplot[mark=*] coordinates {(1.25,0)};
\end{axis}
\end{tikzpicture}%
\newline%
\newline%
\newline%
4) Find the equation of the line with the points (0, -6), (0.75, 0). Round your slope to the nearest whole number:

\begin{tikzpicture} 
\begin{axis}[xmin=-10, xmax=10, ymin=-10, ymax=10, axis x line=middle, axis y line=middle]\addplot[domain=-10:10]{8*x+-6};
\addplot[mark=*] coordinates {(0,-6)};
\addplot[mark=*] coordinates {(0.75,0)};
\end{axis}
\end{tikzpicture}%
\newline%
\newline%
\newline%
5) Find the equation of the line with the points (0, -4), (1.0, 0). Round your slope to the nearest whole number:

\begin{tikzpicture} 
\begin{axis}[xmin=-10, xmax=10, ymin=-10, ymax=10, axis x line=middle, axis y line=middle]\addplot[domain=-10:10]{4*x+-4};
\addplot[mark=*] coordinates {(0,-4)};
\addplot[mark=*] coordinates {(1.0,0)};
\end{axis}
\end{tikzpicture}%
\newline%
\newline%
\newline%
6) Find the equation of the line with the points (0, 1), (-0.12, 0). Round your slope to the nearest whole number:

\begin{tikzpicture} 
\begin{axis}[xmin=-10, xmax=10, ymin=-10, ymax=10, axis x line=middle, axis y line=middle]\addplot[domain=-10:10]{8*x+1};
\addplot[mark=*] coordinates {(0,1)};
\addplot[mark=*] coordinates {(-0.12,0)};
\end{axis}
\end{tikzpicture}%
\newline%
\newline%
\newline%
7) Find the equation of the line with the points (0, 7), (3.5, 0). Round your slope to the nearest whole number:

\begin{tikzpicture} 
\begin{axis}[xmin=-10, xmax=10, ymin=-10, ymax=10, axis x line=middle, axis y line=middle]\addplot[domain=-10:10]{-2*x+7};
\addplot[mark=*] coordinates {(0,7)};
\addplot[mark=*] coordinates {(3.5,0)};
\end{axis}
\end{tikzpicture}%
\newline%
\newline%
\newline%
8) Find the equation of the line with the points (0, -3), (-0.6, 0). Round your slope to the nearest whole number:

\begin{tikzpicture} 
\begin{axis}[xmin=-10, xmax=10, ymin=-10, ymax=10, axis x line=middle, axis y line=middle]\addplot[domain=-10:10]{-5*x+-3};
\addplot[mark=*] coordinates {(0,-3)};
\addplot[mark=*] coordinates {(-0.6,0)};
\end{axis}
\end{tikzpicture}%
\newline%
\newline%
\newline%
9) Find the equation of the line with the points (0, -4), (0.44, 0). Round your slope to the nearest whole number:

\begin{tikzpicture} 
\begin{axis}[xmin=-10, xmax=10, ymin=-10, ymax=10, axis x line=middle, axis y line=middle]\addplot[domain=-10:10]{9*x+-4};
\addplot[mark=*] coordinates {(0,-4)};
\addplot[mark=*] coordinates {(0.44,0)};
\end{axis}
\end{tikzpicture}%
\newline%
\newline%
\newline%
10) Find the equation of the line with the points (0, 2), (-0.4, 0). Round your slope to the nearest whole number:

\begin{tikzpicture} 
\begin{axis}[xmin=-10, xmax=10, ymin=-10, ymax=10, axis x line=middle, axis y line=middle]\addplot[domain=-10:10]{5*x+2};
\addplot[mark=*] coordinates {(0,2)};
\addplot[mark=*] coordinates {(-0.4,0)};
\end{axis}
\end{tikzpicture}%
\newline%
\newline%
\newline%
\pagebreak%
\huge%
\vspace*{\fill}%
\begin{center}%
Solutions%
\end{center}%
\vspace*{\fill}%
\normalsize%
\pagebreak%
\large%
\begin{center}%
\textbf{Solving Basic Algebraic Equations- Solution 1}%
\newline%
\end{center} \normalsize%
1) x=7%
\newline%
2) x=6%
\newline%
3) x=3%
\newline%
4) x=8%
\newline%
5) x=5%
\newline%
6) x=3%
\newline%
7) x=5%
\newline%
8) x=6%
\newline%
9) x=2%
\newline%
10) x=8%
\newline%
11) x=3%
\newline%
12) x=7%
\newline%
13) x=2%
\newline%
14) x=1%
\newline%
15) x=2%
\newline%
16) x=2%
\newline%
17) x=3%
\newline%
18) x=4%
\newline%
19) x=5%
\newline%
20) x=3%
\newline%
21) x=2%
\newline%
22) x=3%
\newline%
23) x=6%
\newline%
24) x=7%
\newline%
25) x=5%
\newline%
26) Bob started with 3 bottles.%
\newline%
27) Sally started with 5 books.%
\newline%
28) Rachel started with 2 marbles.%
\newline%
29) Michael started with 2 toys.%
\newline%
30) Alex started with 5 marbles.%
\newline%
\newpage%
\large%
\begin{center}%
\textbf{Solving Basic Algebraic Equations- Solution 2}%
\newline%
\end{center} \normalsize%
1) x=7%
\newline%
2) x=6%
\newline%
3) x=3%
\newline%
4) x=8%
\newline%
5) x=5%
\newline%
6) x=3%
\newline%
7) x=5%
\newline%
8) x=6%
\newline%
9) x=2%
\newline%
10) x=8%
\newline%
11) x=3%
\newline%
12) x=7%
\newline%
13) x=2%
\newline%
14) x=1%
\newline%
15) x=2%
\newline%
16) x=2%
\newline%
17) x=3%
\newline%
18) x=4%
\newline%
19) x=5%
\newline%
20) x=3%
\newline%
21) x=2%
\newline%
22) x=3%
\newline%
23) x=6%
\newline%
24) x=7%
\newline%
25) x=5%
\newline%
26) Bob started with 3 bottles.%
\newline%
27) Sally started with 5 books.%
\newline%
28) Rachel started with 2 marbles.%
\newline%
29) Michael started with 2 toys.%
\newline%
30) Alex started with 5 marbles.%
\newline%
\newpage%
\large%
\begin{center}%
\textbf{Solving Basic Algebraic Equations- Solution 3}%
\newline%
\end{center} \normalsize%
1) x=7%
\newline%
2) x=6%
\newline%
3) x=3%
\newline%
4) x=8%
\newline%
5) x=5%
\newline%
6) x=3%
\newline%
7) x=5%
\newline%
8) x=6%
\newline%
9) x=2%
\newline%
10) x=8%
\newline%
11) x=3%
\newline%
12) x=7%
\newline%
13) x=2%
\newline%
14) x=1%
\newline%
15) x=2%
\newline%
16) x=2%
\newline%
17) x=3%
\newline%
18) x=4%
\newline%
19) x=5%
\newline%
20) x=3%
\newline%
21) x=2%
\newline%
22) x=3%
\newline%
23) x=6%
\newline%
24) x=7%
\newline%
25) x=5%
\newline%
26) Bob started with 3 bottles.%
\newline%
27) Sally started with 5 books.%
\newline%
28) Rachel started with 2 marbles.%
\newline%
29) Michael started with 2 toys.%
\newline%
30) Alex started with 5 marbles.%
\newline%
\newpage%
\large%
\begin{center}%
\textbf{Solving Basic Algebraic Equations- Solution 4}%
\newline%
\end{center} \normalsize%
1) x=7%
\newline%
2) x=6%
\newline%
3) x=3%
\newline%
4) x=8%
\newline%
5) x=5%
\newline%
6) x=3%
\newline%
7) x=5%
\newline%
8) x=6%
\newline%
9) x=2%
\newline%
10) x=8%
\newline%
11) x=3%
\newline%
12) x=7%
\newline%
13) x=2%
\newline%
14) x=1%
\newline%
15) x=2%
\newline%
16) x=2%
\newline%
17) x=3%
\newline%
18) x=4%
\newline%
19) x=5%
\newline%
20) x=3%
\newline%
21) x=2%
\newline%
22) x=3%
\newline%
23) x=6%
\newline%
24) x=7%
\newline%
25) x=5%
\newline%
26) Bob started with 3 bottles.%
\newline%
27) Sally started with 5 books.%
\newline%
28) Rachel started with 2 marbles.%
\newline%
29) Michael started with 2 toys.%
\newline%
30) Alex started with 5 marbles.%
\newline%
\newpage%
\large%
\begin{center}%
\textbf{Solving Inequalities- Solution 1}%
\newline%
\end{center} \normalsize%
1) x<4%
\newline%
2) x>7%
\newline%
3) x$\leq$3%
\newline%
4) x>7%
\newline%
5) x<3%
\newline%
6) x$\leq$4%
\newline%
7) x$\leq$4%
\newline%
8) x$\geq$1%
\newline%
9) x$\geq$5%
\newline%
10) x>9%
\newline%
11) x$\geq$7%
\newline%
12) x$\leq$4%
\newline%
13) x<6%
\newline%
14) x<8%
\newline%
15) x$\geq$3%
\newline%
16) x$\leq$8%
\newline%
17) x>2%
\newline%
18) x$\geq$1%
\newline%
19) x$\leq$7%
\newline%
20) x$\leq$6%
\newline%
21) x>1%
\newline%
22) x<9%
\newline%
23) x$\leq$2%
\newline%
24) x$\geq$2%
\newline%
25) x>2%
\newline%
26) x$\leq$9%
\newline%
27) x>2%
\newline%
28) x<8%
\newline%
29) x>1%
\newline%
30) x<7%
\newline%
\newpage%
\large%
\begin{center}%
\textbf{Solving Inequalities- Solution 2}%
\newline%
\end{center} \normalsize%
1) x<4%
\newline%
2) x>7%
\newline%
3) x$\leq$3%
\newline%
4) x>7%
\newline%
5) x<3%
\newline%
6) x$\leq$4%
\newline%
7) x$\leq$4%
\newline%
8) x$\geq$1%
\newline%
9) x$\geq$5%
\newline%
10) x>9%
\newline%
11) x$\geq$7%
\newline%
12) x$\leq$4%
\newline%
13) x<6%
\newline%
14) x<8%
\newline%
15) x$\geq$3%
\newline%
16) x$\leq$8%
\newline%
17) x>2%
\newline%
18) x$\geq$1%
\newline%
19) x$\leq$7%
\newline%
20) x$\leq$6%
\newline%
21) x>1%
\newline%
22) x<9%
\newline%
23) x$\leq$2%
\newline%
24) x$\geq$2%
\newline%
25) x>2%
\newline%
26) x$\leq$9%
\newline%
27) x>2%
\newline%
28) x<8%
\newline%
29) x>1%
\newline%
30) x<7%
\newline%
\newpage%
\large%
\begin{center}%
\textbf{Solving Inequalities- Solution 3}%
\newline%
\end{center} \normalsize%
1) x<4%
\newline%
2) x>7%
\newline%
3) x$\leq$3%
\newline%
4) x>7%
\newline%
5) x<3%
\newline%
6) x$\leq$4%
\newline%
7) x$\leq$4%
\newline%
8) x$\geq$1%
\newline%
9) x$\geq$5%
\newline%
10) x>9%
\newline%
11) x$\geq$7%
\newline%
12) x$\leq$4%
\newline%
13) x<6%
\newline%
14) x<8%
\newline%
15) x$\geq$3%
\newline%
16) x$\leq$8%
\newline%
17) x>2%
\newline%
18) x$\geq$1%
\newline%
19) x$\leq$7%
\newline%
20) x$\leq$6%
\newline%
21) x>1%
\newline%
22) x<9%
\newline%
23) x$\leq$2%
\newline%
24) x$\geq$2%
\newline%
25) x>2%
\newline%
26) x$\leq$9%
\newline%
27) x>2%
\newline%
28) x<8%
\newline%
29) x>1%
\newline%
30) x<7%
\newline%
\newpage%
\large%
\begin{center}%
\textbf{Solving Inequalities- Solution 4}%
\newline%
\end{center} \normalsize%
1) x<4%
\newline%
2) x>7%
\newline%
3) x$\leq$3%
\newline%
4) x>7%
\newline%
5) x<3%
\newline%
6) x$\leq$4%
\newline%
7) x$\leq$4%
\newline%
8) x$\geq$1%
\newline%
9) x$\geq$5%
\newline%
10) x>9%
\newline%
11) x$\geq$7%
\newline%
12) x$\leq$4%
\newline%
13) x<6%
\newline%
14) x<8%
\newline%
15) x$\geq$3%
\newline%
16) x$\leq$8%
\newline%
17) x>2%
\newline%
18) x$\geq$1%
\newline%
19) x$\leq$7%
\newline%
20) x$\leq$6%
\newline%
21) x>1%
\newline%
22) x<9%
\newline%
23) x$\leq$2%
\newline%
24) x$\geq$2%
\newline%
25) x>2%
\newline%
26) x$\leq$9%
\newline%
27) x>2%
\newline%
28) x<8%
\newline%
29) x>1%
\newline%
30) x<7%
\newline%
\newpage%
\large%
\begin{center}%
\textbf{Factoring Basic Quadratic Equations- Solution 1}%
\newline%
\end{center} \normalsize%
1) (x + 7)(x + 6)%
\newline%
2) (x + 5)(x + 8)%
\newline%
3) (x + 10)(x + 1)%
\newline%
4) (x + 1)(x - 3)%
\newline%
5) (x - 5)(x + 7)%
\newline%
6) (x + 9)(x - 5)%
\newline%
7) (x - 6)(x + 4)%
\newline%
8) (x + 5)(x + 7)%
\newline%
9) (x - 5)(x + 10)%
\newline%
10) (x + 4)(x + 1)%
\newline%
11) (x + 7)(x - 3)%
\newline%
12) (x + 1)(x + 3)%
\newline%
13) (x - 7)(x - 5)%
\newline%
14) (x + 7)(x - 7)%
\newline%
15) (x - 10)(x - 9)%
\newline%
16) (x - 6)(x + 5)%
\newline%
17) (x + 1)(x + 10)%
\newline%
18) (x - 10)(x + 3)%
\newline%
19) (x + 6)(x - 9)%
\newline%
20) (x + 2)(x + 4)%
\newline%
21) (x - 10)(x + 7)%
\newline%
22) (x + 2)(x + 1)%
\newline%
23) (x - 4)(x + 6)%
\newline%
24) (x + 3)(x - 6)%
\newline%
25) (x + 4)(x + 8)%
\newline%
26) (x + 2)(x - 10)%
\newline%
27) (x - 1)(x + 3)%
\newline%
28) (x + 10)(x + 10)%
\newline%
29) (x + 1)(x + 7)%
\newline%
30) (x - 6)(x - 7)%
\newline%
\newpage%
\large%
\begin{center}%
\textbf{Factoring Basic Quadratic Equations- Solution 2}%
\newline%
\end{center} \normalsize%
1) (x + 7)(x + 6)%
\newline%
2) (x + 5)(x + 8)%
\newline%
3) (x + 10)(x + 1)%
\newline%
4) (x + 1)(x - 3)%
\newline%
5) (x - 5)(x + 7)%
\newline%
6) (x + 9)(x - 5)%
\newline%
7) (x - 6)(x + 4)%
\newline%
8) (x + 5)(x + 7)%
\newline%
9) (x - 5)(x + 10)%
\newline%
10) (x + 4)(x + 1)%
\newline%
11) (x + 7)(x - 3)%
\newline%
12) (x + 1)(x + 3)%
\newline%
13) (x - 7)(x - 5)%
\newline%
14) (x + 7)(x - 7)%
\newline%
15) (x - 10)(x - 9)%
\newline%
16) (x - 6)(x + 5)%
\newline%
17) (x + 1)(x + 10)%
\newline%
18) (x - 10)(x + 3)%
\newline%
19) (x + 6)(x - 9)%
\newline%
20) (x + 2)(x + 4)%
\newline%
21) (x - 10)(x + 7)%
\newline%
22) (x + 2)(x + 1)%
\newline%
23) (x - 4)(x + 6)%
\newline%
24) (x + 3)(x - 6)%
\newline%
25) (x + 4)(x + 8)%
\newline%
26) (x + 2)(x - 10)%
\newline%
27) (x - 1)(x + 3)%
\newline%
28) (x + 10)(x + 10)%
\newline%
29) (x + 1)(x + 7)%
\newline%
30) (x - 6)(x - 7)%
\newline%
\newpage%
\large%
\begin{center}%
\textbf{Factoring Basic Quadratic Equations- Solution 3}%
\newline%
\end{center} \normalsize%
1) (x + 7)(x + 6)%
\newline%
2) (x + 5)(x + 8)%
\newline%
3) (x + 10)(x + 1)%
\newline%
4) (x + 1)(x - 3)%
\newline%
5) (x - 5)(x + 7)%
\newline%
6) (x + 9)(x - 5)%
\newline%
7) (x - 6)(x + 4)%
\newline%
8) (x + 5)(x + 7)%
\newline%
9) (x - 5)(x + 10)%
\newline%
10) (x + 4)(x + 1)%
\newline%
11) (x + 7)(x - 3)%
\newline%
12) (x + 1)(x + 3)%
\newline%
13) (x - 7)(x - 5)%
\newline%
14) (x + 7)(x - 7)%
\newline%
15) (x - 10)(x - 9)%
\newline%
16) (x - 6)(x + 5)%
\newline%
17) (x + 1)(x + 10)%
\newline%
18) (x - 10)(x + 3)%
\newline%
19) (x + 6)(x - 9)%
\newline%
20) (x + 2)(x + 4)%
\newline%
21) (x - 10)(x + 7)%
\newline%
22) (x + 2)(x + 1)%
\newline%
23) (x - 4)(x + 6)%
\newline%
24) (x + 3)(x - 6)%
\newline%
25) (x + 4)(x + 8)%
\newline%
26) (x + 2)(x - 10)%
\newline%
27) (x - 1)(x + 3)%
\newline%
28) (x + 10)(x + 10)%
\newline%
29) (x + 1)(x + 7)%
\newline%
30) (x - 6)(x - 7)%
\newline%
\newpage%
\large%
\begin{center}%
\textbf{Factoring Basic Quadratic Equations- Solution 4}%
\newline%
\end{center} \normalsize%
1) (x + 7)(x + 6)%
\newline%
2) (x + 5)(x + 8)%
\newline%
3) (x + 10)(x + 1)%
\newline%
4) (x + 1)(x - 3)%
\newline%
5) (x - 5)(x + 7)%
\newline%
6) (x + 9)(x - 5)%
\newline%
7) (x - 6)(x + 4)%
\newline%
8) (x + 5)(x + 7)%
\newline%
9) (x - 5)(x + 10)%
\newline%
10) (x + 4)(x + 1)%
\newline%
11) (x + 7)(x - 3)%
\newline%
12) (x + 1)(x + 3)%
\newline%
13) (x - 7)(x - 5)%
\newline%
14) (x + 7)(x - 7)%
\newline%
15) (x - 10)(x - 9)%
\newline%
16) (x - 6)(x + 5)%
\newline%
17) (x + 1)(x + 10)%
\newline%
18) (x - 10)(x + 3)%
\newline%
19) (x + 6)(x - 9)%
\newline%
20) (x + 2)(x + 4)%
\newline%
21) (x - 10)(x + 7)%
\newline%
22) (x + 2)(x + 1)%
\newline%
23) (x - 4)(x + 6)%
\newline%
24) (x + 3)(x - 6)%
\newline%
25) (x + 4)(x + 8)%
\newline%
26) (x + 2)(x - 10)%
\newline%
27) (x - 1)(x + 3)%
\newline%
28) (x + 10)(x + 10)%
\newline%
29) (x + 1)(x + 7)%
\newline%
30) (x - 6)(x - 7)%
\newline%
\newpage%
\large%
\begin{center}%
\textbf{Factoring Advanced Quadratic Equations- Solution 1}%
\newline%
\end{center} \normalsize%
1) (-6x- 7)(-1x - 5)%
\newline%
2) (-4x- 4)(-1x - 4)%
\newline%
3) (4x + 9)(2x - 8)%
\newline%
4) (5x- 7)(2x - 5)%
\newline%
5) (3x- 5)(-2x + 3)%
\newline%
6) (4x- 5)(5x - 6)%
\newline%
7) (-2x- 3)(-5x + 8)%
\newline%
8) (-3x- 5)(2x + 7)%
\newline%
9) (-5x + 3)(-4x - 10)%
\newline%
10) (-5x- 10)(-2x + 1)%
\newline%
11) (4x- 3)(-4x + 9)%
\newline%
12) (-4x- 1)(1x + 5)%
\newline%
13) (-4x- 9)(-3x - 1)%
\newline%
14) (-2x- 2)(1x - 7)%
\newline%
15) (-2x + 7)(-5x + 2)%
\newline%
16) (-1x- 9)(3x + 5)%
\newline%
17) (2x + 3)(1x - 7)%
\newline%
18) (3x- 5)(4x + 2)%
\newline%
19) (-1x + 4)(3x - 6)%
\newline%
20) (-6x- 10)(-6x + 8)%
\newline%
21) (-4x- 4)(-1x - 3)%
\newline%
22) (4x + 9)(3x + 5)%
\newline%
23) (2x- 2)(-5x + 9)%
\newline%
24) (5x- 10)(-3x - 1)%
\newline%
25) (1x + 6)(-4x - 3)%
\newline%
26) (-4x + 6)(-1x + 1)%
\newline%
27) (4x + 7)(6x - 7)%
\newline%
28) (5x + 5)(-6x + 4)%
\newline%
29) (-1x- 2)(4x + 3)%
\newline%
30) (-2x- 7)(2x - 5)%
\newline%
\newpage%
\large%
\begin{center}%
\textbf{Factoring Advanced Quadratic Equations- Solution 2}%
\newline%
\end{center} \normalsize%
1) (-6x- 7)(-1x - 5)%
\newline%
2) (-4x- 4)(-1x - 4)%
\newline%
3) (4x + 9)(2x - 8)%
\newline%
4) (5x- 7)(2x - 5)%
\newline%
5) (3x- 5)(-2x + 3)%
\newline%
6) (4x- 5)(5x - 6)%
\newline%
7) (-2x- 3)(-5x + 8)%
\newline%
8) (-3x- 5)(2x + 7)%
\newline%
9) (-5x + 3)(-4x - 10)%
\newline%
10) (-5x- 10)(-2x + 1)%
\newline%
11) (4x- 3)(-4x + 9)%
\newline%
12) (-4x- 1)(1x + 5)%
\newline%
13) (-4x- 9)(-3x - 1)%
\newline%
14) (-2x- 2)(1x - 7)%
\newline%
15) (-2x + 7)(-5x + 2)%
\newline%
16) (-1x- 9)(3x + 5)%
\newline%
17) (2x + 3)(1x - 7)%
\newline%
18) (3x- 5)(4x + 2)%
\newline%
19) (-1x + 4)(3x - 6)%
\newline%
20) (-6x- 10)(-6x + 8)%
\newline%
21) (-4x- 4)(-1x - 3)%
\newline%
22) (4x + 9)(3x + 5)%
\newline%
23) (2x- 2)(-5x + 9)%
\newline%
24) (5x- 10)(-3x - 1)%
\newline%
25) (1x + 6)(-4x - 3)%
\newline%
26) (-4x + 6)(-1x + 1)%
\newline%
27) (4x + 7)(6x - 7)%
\newline%
28) (5x + 5)(-6x + 4)%
\newline%
29) (-1x- 2)(4x + 3)%
\newline%
30) (-2x- 7)(2x - 5)%
\newline%
\newpage%
\large%
\begin{center}%
\textbf{Factoring Advanced Quadratic Equations- Solution 3}%
\newline%
\end{center} \normalsize%
1) (-6x- 7)(-1x - 5)%
\newline%
2) (-4x- 4)(-1x - 4)%
\newline%
3) (4x + 9)(2x - 8)%
\newline%
4) (5x- 7)(2x - 5)%
\newline%
5) (3x- 5)(-2x + 3)%
\newline%
6) (4x- 5)(5x - 6)%
\newline%
7) (-2x- 3)(-5x + 8)%
\newline%
8) (-3x- 5)(2x + 7)%
\newline%
9) (-5x + 3)(-4x - 10)%
\newline%
10) (-5x- 10)(-2x + 1)%
\newline%
11) (4x- 3)(-4x + 9)%
\newline%
12) (-4x- 1)(1x + 5)%
\newline%
13) (-4x- 9)(-3x - 1)%
\newline%
14) (-2x- 2)(1x - 7)%
\newline%
15) (-2x + 7)(-5x + 2)%
\newline%
16) (-1x- 9)(3x + 5)%
\newline%
17) (2x + 3)(1x - 7)%
\newline%
18) (3x- 5)(4x + 2)%
\newline%
19) (-1x + 4)(3x - 6)%
\newline%
20) (-6x- 10)(-6x + 8)%
\newline%
21) (-4x- 4)(-1x - 3)%
\newline%
22) (4x + 9)(3x + 5)%
\newline%
23) (2x- 2)(-5x + 9)%
\newline%
24) (5x- 10)(-3x - 1)%
\newline%
25) (1x + 6)(-4x - 3)%
\newline%
26) (-4x + 6)(-1x + 1)%
\newline%
27) (4x + 7)(6x - 7)%
\newline%
28) (5x + 5)(-6x + 4)%
\newline%
29) (-1x- 2)(4x + 3)%
\newline%
30) (-2x- 7)(2x - 5)%
\newline%
\newpage%
\large%
\begin{center}%
\textbf{Factoring Advanced Quadratic Equations- Solution 4}%
\newline%
\end{center} \normalsize%
1) (-6x- 7)(-1x - 5)%
\newline%
2) (-4x- 4)(-1x - 4)%
\newline%
3) (4x + 9)(2x - 8)%
\newline%
4) (5x- 7)(2x - 5)%
\newline%
5) (3x- 5)(-2x + 3)%
\newline%
6) (4x- 5)(5x - 6)%
\newline%
7) (-2x- 3)(-5x + 8)%
\newline%
8) (-3x- 5)(2x + 7)%
\newline%
9) (-5x + 3)(-4x - 10)%
\newline%
10) (-5x- 10)(-2x + 1)%
\newline%
11) (4x- 3)(-4x + 9)%
\newline%
12) (-4x- 1)(1x + 5)%
\newline%
13) (-4x- 9)(-3x - 1)%
\newline%
14) (-2x- 2)(1x - 7)%
\newline%
15) (-2x + 7)(-5x + 2)%
\newline%
16) (-1x- 9)(3x + 5)%
\newline%
17) (2x + 3)(1x - 7)%
\newline%
18) (3x- 5)(4x + 2)%
\newline%
19) (-1x + 4)(3x - 6)%
\newline%
20) (-6x- 10)(-6x + 8)%
\newline%
21) (-4x- 4)(-1x - 3)%
\newline%
22) (4x + 9)(3x + 5)%
\newline%
23) (2x- 2)(-5x + 9)%
\newline%
24) (5x- 10)(-3x - 1)%
\newline%
25) (1x + 6)(-4x - 3)%
\newline%
26) (-4x + 6)(-1x + 1)%
\newline%
27) (4x + 7)(6x - 7)%
\newline%
28) (5x + 5)(-6x + 4)%
\newline%
29) (-1x- 2)(4x + 3)%
\newline%
30) (-2x- 7)(2x - 5)%
\newline%
\newpage%
\large%
\begin{center}%
\textbf{Finding the Equation of a Line- Solution 1}%
\newline%
\end{center} \normalsize%
1) y=4x + -1%
\newline%
2) y=-6x + 5%
\newline%
3) y=4x + -5%
\newline%
4) y=8x + -6%
\newline%
5) y=4x + -4%
\newline%
6) y=8x + 1%
\newline%
7) y=-2x + 7%
\newline%
8) y=-5x + -3%
\newline%
9) y=9x + -4%
\newline%
10) y=5x + 2%
\newline%
\newpage%
\large%
\begin{center}%
\textbf{Finding the Equation of a Line- Solution 2}%
\newline%
\end{center} \normalsize%
1) y=4x + -1%
\newline%
2) y=-6x + 5%
\newline%
3) y=4x + -5%
\newline%
4) y=8x + -6%
\newline%
5) y=4x + -4%
\newline%
6) y=8x + 1%
\newline%
7) y=-2x + 7%
\newline%
8) y=-5x + -3%
\newline%
9) y=9x + -4%
\newline%
10) y=5x + 2%
\newline%
\newpage%
\large%
\begin{center}%
\textbf{Finding the Equation of a Line- Solution 3}%
\newline%
\end{center} \normalsize%
1) y=4x + -1%
\newline%
2) y=-6x + 5%
\newline%
3) y=4x + -5%
\newline%
4) y=8x + -6%
\newline%
5) y=4x + -4%
\newline%
6) y=8x + 1%
\newline%
7) y=-2x + 7%
\newline%
8) y=-5x + -3%
\newline%
9) y=9x + -4%
\newline%
10) y=5x + 2%
\newline%
\newpage%
\large%
\begin{center}%
\textbf{Finding the Equation of a Line- Solution 4}%
\newline%
\end{center} \normalsize%
1) y=4x + -1%
\newline%
2) y=-6x + 5%
\newline%
3) y=4x + -5%
\newline%
4) y=8x + -6%
\newline%
5) y=4x + -4%
\newline%
6) y=8x + 1%
\newline%
7) y=-2x + 7%
\newline%
8) y=-5x + -3%
\newline%
9) y=9x + -4%
\newline%
10) y=5x + 2%
\newline%
\newpage%
\end{document}